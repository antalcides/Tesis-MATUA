
\chapter*{\textcolor{blue}{Abstract}}

\addstarredchapter{Abstract}

\thispagestyle{empty} \begin{small} 


\lettrine[lines=4,loversize=-0.1,lraise=0.1,lhang=.2]{T}{}{\vskip -2exhe sequences are very versatile data structures. In
a straightforward manner, a sequence of symbols can store any type
of information. Systematic analysis of sequences is a very rich area
of algorithmics, with lots of successful applications. The comparison
by sequence alignment is a very powerful analysis tool. Dynamic programming
is one of the most popular and efficient approaches to align two sequences.
However, despite their utility, alignments are not always the best
option for characterizing the function of two sequences. Sequences
often encode information in different levels of organization (meta-information).
In these cases, direct sequence comparison is not able to unveil those
higher-order structures that can actually explain the relationship
between the sequences.}

We have contributed with the work presented here to improve the way
in which two sequences can be compared, developing a new family of
algorithms that align high level information encoded in biological
sequences (meta-alignment). Initially, we have redesigned an existent
algorithm, based in dynamic programming, to align two sequences of
meta-information, introducing later several improvements for a better
performance. Next, we have developed a multiple meta-alignment algorithm,
by combining the general algorithm with the progressive schema. In
addition, we have studied the properties of the resulting meta-alignments,
modifying the algorithm to identify non-collinear or permuted configurations.

Molecular life is a great example of the sequence versatility. Comparative
genomics provide the identification of numerous biologically functional
elements. The nucleotide sequence of many genes, for example, is relatively
well conserved between different species. In contrast, the sequences
that regulate the gene expression are shorter and weaker. Thus, the
simultaneous activation of a set of genes only can be explained in
terms of conservation between configurations of higher-order regulatory
elements, that can not be detected at the sequence level. We, therefore,
have trained our meta-alignment programs in several datasets of regulatory
regions collected from the literature. Then, we have tested the accuracy
of our approximation to successfully characterize the promoter regions
of human genes and their orthologs in other species. \textbf{Palabras
Claves:} polinomios tipo Apostol generalizados; relaci�n de recurrencia;
ecuaci�n diferencial; polinomios de Apostol-Bernoulli; polinomios
de Genocchi; polinomios de Jacobi; polinomios de Hermite; polinomios
de Laguerre; polinomios de Charlier; polinomios de Bessel; polinomios
de Bernoulli generalizados; n�meros de Stirling de segunda clase.
\end{small}

\clearemptydoublepage


\chapter*{\textcolor{blue}{Resumen}}

\addstarredchapter{Resumen}\thispagestyle{empty}\begin{small} 


\lettrine[lines=4,loversize=-0.1,lraise=0.1,lhang=.2]{E}{}{n este trabajo estudiamos una nueva clase de polinomios tipo Apostol
generalizados $\mathcal{Q}_{n}^{[m-1,\alpha]}(x,b,c;\lambda;u,v)$
con $(\alpha,u,v\in\CC$ y $b,c\in\RR^{+})$ los cuales est�n definidos
en un entorno adecuado de $t=0$ por la siguiente funci�n generatriz:
\begin{equation}
{\displaystyle \left(\frac{(2^{u}t^{v})^{m}}{\lambda b^{t}+\sum\limits _{l=0}^{m-1}\frac{(t\log b)^{l}}{l!}}\right)^{\alpha}c^{xt}={\displaystyle \sum\limits _{n=0}^{\infty}\mathcal{Q}_{n}^{[m-1,\alpha]}(x,b,c;\lambda;u,v)\frac{t^{n}}{n!};\quad|t\log b|<|log(-\lambda)|,}}
\end{equation}
estos polinomios generan a las nuevas clases de todos los polinomios
conocidos de Apostol, as\'{i}:\\
 
\begin{align}
 & \mathcal{B}_{n}^{[m-1,\alpha]}(x;b,c;\lambda)=(-1)^{\alpha}\mathcal{Q}_{n}^{[m-1,\alpha]}(x;b,c;-\lambda;0,1),\\
 & \mathcal{E}_{n}^{[m-1,\alpha]}(x;b,c;\lambda)=\mathcal{Q}_{n}^{[m-1,\alpha]}(x;b,c;\lambda;1,0),\\
 & \mathcal{G}_{n}^{[m-1,\alpha]}(x;b,c;\lambda)=\mathcal{Q}_{n}^{[m-1,\alpha]}(x;b,c;\lambda;1,1).
\end{align}
}

Establecemos algunas propiedades b�sicas para estos polinomios, incluyendo
relaci�n de recurrencia y la ecuaci�n diferencial que estos polinomios
satisfacen. Finalmente se determinan f�rmulas de conexi�n entre los
polinomios y los polinomios de Genocchi, los polinomios de Jacobi,
los polinomios de Hermite, los polinomios de Laguerre, los polinomios
de Charlier, los polinomios de Bessel, los polinomios de Bernoulli
generalizados $B_{n}^{[m-1]}(x)$ y los n�meros de Stirling de segunda
clase, lo cual extiende algunos resultados conocidos. Finalmente introducimos
una nueva clase de polinomios tipo Apostol generalizados basados en
los polinomios de Hermite de dos variables ${_{H}}\mathcal{Q}_{n}^{[m-1,\alpha]}(x,b,c;\lambda;u,v)$
y mencionamos propiedades b�sicas. \\


\noindent \textbf{Palabras Claves:} polinomios tipo Apostol generalizados;
relaci�n de recurrencia; ecuaci�n diferencial; polinomios de Apostol-Bernoulli;
polinomios de Genocchi; polinomios de Jacobi; polinomios de Hermite;
polinomios de Laguerre; polinomios de Charlier; polinomios de Bessel;
polinomios de Bernoulli generalizados; n�meros de Stirling de segunda
clase. \end{small}

\clearemptydoublepage
