
\chapter{CONCLUSIONES Y RECOMENDACIONES}
\markboth{CAP\'ITULO 4\quad CONCLUSIONES Y RECOMENDACIONES}{CAP\'ITULO 4\quad CONCLUSIONES Y RECOMENDACIONES}
\label{AB4}

%recuperar numeracion arabica%
\def\thechapter{\arabic{chapter}}

En este \'ultimo cap\'itulo haremos un resumen de las t\'ecnicas y resultados fundamentales de nuestro trabajo y expondremos problemas abiertos y futuras l\'ineas de investigaci\'on relativos a *****.

\section{Conclusiones}
Podemos decir que el resultado m\'as importante de este trabajo es el relacionado con *****, (ver teorema ****).



\section{Problemas Abiertos y Futuras L\'{\i}neas de Investigaci\'on}

De lo expuesto en la secci\'on precedente se desprende que algunos  problemas interesantes que permitir\'{\i}an  continuar con investigaciones relacionadas, ser\'{\i}an:


\textbf{Problema 1.}
Series de Fourier y representaci\'on integral de los polinomios $\mathcal{Q}_n^{[m-1,\alpha]}(x,b,c;\lambda;u,v)$.

\textbf{Problema 2.}
Estudio de las propiedades de los polinomios $\mathcal{Q}_n^{(\alpha)}(x;\lambda;a,b,c;u,v)$ definidos en (\ref{eqnewpol2}) y relaci\'on con otros polinomios y n\'umeros.

\textbf{Problema 3.}
Series de Fourier y representaci\'on integral de los polinomios $\mathcal{Q}_n^{(\alpha)}(x;\lambda;a,b,c;u,v)$ definidos en (\ref{eqnewpol2}).

\textbf{Problema 4.}
Estudio de la nueva clase de polinomios tipo Apostol generalizados basados en Hermite ${_{H}}\mathcal{Q}_n^{[m-1,\alpha]}(x,b,c;\lambda;u,v)$ que est\'an definidos en (\ref{new787}).

\textbf{Problema 5.}
Estudio de otra nueva clase de polinomios tipo Apostol generalizados basados en Hermite ${_{H}}\mathcal{Q}_n^{(\alpha)}(x;\lambda;a,b,c;u,v)$ que est\'an definidos por la siguiente funci\'on generatriz:
\begin{equation}\label{eqnewpol23}
\displaystyle\left(\frac{2^uz^v}{\lambda b^z+a^z}\right)^{\alpha}c^{xz+(\log c)yz^2} =\sum\limits_{n=0}^{\infty}
{_{H}}\mathcal{Q}_n^{(\alpha)}(x;\lambda;a,b,c;u,v)\frac{z^n}{n!} 
\end{equation}



