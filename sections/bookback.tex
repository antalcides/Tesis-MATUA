\begin{center}
\begin{minipage}[t][0.965\textheight][c]{0.8\textwidth}%
\color{coverwhite}

\begin{center}
\avantgarboldhuge\mytitle \vskip 2.5ex {\avantgarboldLarge\thyauthor}
 \colorbox{covergreen}{%
\begin{minipage}[c]{0.925\textwidth}%
\setlength{\parindent}{0.5cm}\color{coverwhite}\avantgarsmall\begin{small}
Al interior de los experimentos estad�sticos la teor�a de los dise�os
�ptimos ha sido desarrollada. En general el tema de esta teor�a es
que para un apropiado modelo, si queremos poner �nfasis sobre una
cualidad particular de los par�metros a estimar, entonces la configuraci�n
experimental deber\'{i}a ser elegida de acuerdo a ciertos criterios
con sentido estad�stico. En la literatura relacionada con los dise�os
�ptimos, un prominente autor fue Kiefer (1959), el cu�l present� los
principales conceptos, tales como dise�os aproximados y una variedad
de criterios de �ptimalidad para esta rama de los dise�os de experimentos;
Kiefer, en particular dio el nombre $D-$optimalidad al criterio introducido
por Wald (1943), este criterio es el m�s comunmente aplicado y est�
definido en funci�n del determinante de la matriz de covarianza.\\


M�s recientemente son reconocidos los libros de Atkinson y Donev (1992)
y Pukelsheim (1993), donde los autores hacen una presentaci�n estad�stica
formal de los dise�os �ptimos. El presente trabajo se ha organizado
en tres cap�tulos: el cap�tulo uno (Preliminares) contiene conceptos
generales que sirven de apoyo y base a la teor�a que se desarrolla
en los siguientes dos cap�tulos. El cap�tulo dos trata sobre dise�os
�ptimos en la presencia de efectos de bloques aleatorios, y en el
cap�tulo tres se daran a conocer las conclusiones y una serie de problemas
abiertos para futuras investigaciones relacionadas con el tema central
de este trabajo de investigaci�n. \end{small} %
\end{minipage}}
\par\end{center}

\vskip 2.5ex

\begin{center}
{\avantgarboldLarge Programa de Matem�ticas}\\

\par\end{center}

\begin{center}
{\avantgarLarge Universidad del Atl�ntico}\\[3ex]
\par\end{center}

\begin{center}

\par\end{center}%
\end{minipage}
\par\end{center}

%}%centering
