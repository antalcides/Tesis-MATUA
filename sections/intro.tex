
\chapter*{\textcolor{blue}{Introducci�n}}

\addstarredchapter{Introducci�n}

%\iflettrine


\lettrine[lines=4,loversize=-0.1,lraise=0.1,lhang=.2]{A}{}{l interior de los experimentos estad\'{i}sticos la teor\'{i}a de
los dise�os �ptimos ha sido desarrollada. En general el tema de esta
teor\'{i}a es que para un apropiado modelo, si queremos poner �nfasis
sobre una cualidad particular de los par�metros a estimar, entonces
la configuraci�n experimental deber\'{i}a ser elegida de acuerdo a
ciertos criterios con sentido estad\'{i}stico. En la literatura relacionada
con los dise�os �ptimos, un prominente autor fue Kiefer (1959), el
cu�l present� los principales conceptos, tales como dise�os aproximados
y una variedad de criterios de �ptimalidad para esta rama de los dise�os
de experimentos; Kiefer, en particular dio el nombre $D-$optimalidad
al criterio introducido por Wald (1943), este criterio es el m�s comunmente
aplicado y est� definido en funci�n del determinante de la matriz
de covarianza.\\
}

M�s recientemente son reconocidos los libros de Atkinson y Donev (1992)
y Pukelsheim (1993), donde los autores hacen una presentaci�n estad\'{i}stica
formal de los dise�os �ptimos. El presente trabajo se ha organizado
en tres cap\'{i}tulos: el cap\'{i}tulo uno (Preliminares) contiene
conceptos generales que sirven de apoyo y base a la teor\'{i}a que
se desarrolla en los siguientes dos cap\'{i}tulos. El cap\'{i}tulo
dos trata sobre dise�os �ptimos en la presencia de efectos de bloques
aleatorios, y en el cap\'{i}tulo tres se daran a conocer las conclusiones
y una serie de problemas abiertos para futuras investigaciones relacionadas
con el tema central de este trabajo de investigaci�n. 
