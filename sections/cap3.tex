
\chapter{\textcolor{blue}{Conclusiones y trabajos futuros}}

\label{AB3}

%recuperar numeracion arabica%\global\long\def\thechapter{\arabic{chapter}}



\lettrine[lines=4,loversize=-0.1,lraise=0.1,lhang=.2]{E}{}{n el presente trabajo se desarrolla la matriz de informaci�n de
los criterios fijos en un modelo de regresi�n lineal en la presencia
de efectos de bloques aleatorios como una combinaci�n convexa de las
matrices de informaci�n de los modelos l�mites cuando la varianza
del efecto de bloque es cero o tiende a infinito.\\
}

%recuperar numeracion arabica%\global\long\def\thechapter{\arabic{chapter}}
 Por otro lado, se muestra que los dise�os �ptimos para modelos de
efectos fijos tambi�n son �ptimos para modelos en la presencia de
efectos de bloques aleatorios siempre que los bloques sean uniformes.\\


$D$ y $D_{\boldsymbol{\beta}}$-optimalidad coinciden tambi�n en
modelos en la presencia de efectos de bloques aleatorios, un hecho
ya conocido en los escenarios con efectos fijos.\\


Para trabajos futuros se pueden considerar modelos en la presencia
de efectos de bloques aleatorios donde los par�metros de regresi�n
interact�an con diferentes grupos o tratamientos. 
