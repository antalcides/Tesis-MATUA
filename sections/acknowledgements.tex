\chapter*{\textcolor{blue}{\textbf{A}gradecimientos}}
\addstarredchapter{Agradecimientos}

\lettrine[lines=4,loversize=-0.1,lraise=0.1,lhang=.2]{T}{his section is usually the part of the thesis}
in which the authors mention those people that have decisively contributed to the
presented work. As I am a generous person but my gratitude is not infinite,
I want to express the following acknowledgments only to those that really deserve
the reward of being cited here.

I am totally convinced about how to begin and to end this
section. Honestly, there is only one person that deserves the honor
of appearing in the first place of this section: myself. This thesis
has not been an easy work at all. In our society, most computer scientists
are working on the private sector, so that the orientation of their careers
to investigation is a rare fact nowadays. And I have learned to live with
this pressure as well. For a computer engineer like me, it has been a rich
experience to work in a research environment
devoted to the biological discovery. However, it has also been very demanding,
because this thesis is not only about the development of new theoretical
algorithms. It was also an exercise of application of such methods in real data
to obtain novel biological conclusions. To sum up, it was like doing two thesis:
one about computer science, and one about molecular biology. And I am very proud
to have fulfilled both aspects of my work. Therefore, I want to thank myself
for not abandoning, for supporting myself, for carrying on when the main
objectives of the thesis seemed to be very far, when things were going too slow, or
when the adaptation to the academic world was difficult because of its
competitiveness.

I want to warmly thank you my two PhD advisors, Xavier Messeguer and Roderic Guig\'o
for the correct direction of my work. We started in 1999 with the program
\prog{geneid} to successfully obtain my degree in computer science the summer
of 2000. A few years later, I am very happy to see that the majority of people
in my lab have used it in their investigations with a lot of success, and some of
them have even been able to modify some of its modules without difficulty. Thanks
then to Xavier, for your calm, for your wisdom, for your patience with me, specially
when continuous communication was sometimes difficult because I was physically
working outside the department, and for believing in my work in many occasions that
I will never forget. Also thanks to Roderic, for the computational facilities,
for the opportunity to work with so good people in the IMIM lab, for let me
learning involuntarily from your experience, for the funding, for the international
meetings that increased a lot my knowledge and for being always ambitious in whatever
task you are doing.

Also thanks to my colleagues at work from which I learn a lot of useful things.
Many of them have also decisively contributed to improve the quality of this thesis.
Specially thanks to Josep Francesc Abril that have assisted me in uncountable
occasions with his priceless help. Thanks also to those that were in the lab when I
arrive over there: Mois\'es Burset, Sergi Castellano and Gen\'{\i}s Parra. To those
that arrived later, many thanks as well: Robert Castelo, Jan-Jaap Wesselink,
Mar Alb\`a, Eduardo Eyras, Charles Chapple, Nicol\'as Bellora and Miguel Pignatelli.
Further thanks to our system administrators, Alfons Gonz\'alez, Xavier Fustero
and \`Oscar Gonz\'alez. I want to specially acknowledge Robert Castelo for
the excellent template in \LaTeX{} from his PhD thesis. This template was
later adapted by Sergi Castellano and Gen\'{\i}s Parra, and substantially
improved by Josep Francesc Abril, for their theses. This manuscript is an
adaptation of such templates following my own style.

At this point I want to remember those teachers from many disciplines that have
contributed positively to my education throughout my life. First of all, thanks to
those in the teaching staff that positively contributed to my education at my school
Hermanos Maristas de Les Corts. Second, to those good teachers I have found in my
university Facultad d'Inform\`atica de Barcelona. Finally, many thanks to my teachers
at Escola Oficial d'Idiomes de Barcelona, that help me to speak and to write correctly
in English and Italian.

During this time, I have been involved in many educational activities related
to teach about Bioinformatics in Masters and other programs. Specially thanks among others
for your cooperation and your advice to Manuel G\'omez (Centro Nacional de
Astrobiolog\'ia, Madrid), Silvia Atriain (Universitat de Barcelona, Barcelona) and
Alfonso Valencia (Centro Nacional de Biotecnolog\'ia, Madrid).

Many thanks also to Dr. Montserrat Corominas and Dr. Jorge Ferrer for two fruitful and interesting
collaborations, using the expression data produced during the research performed in their labs.

For formal reasons, I have to thank the Ministerio de Educaci\'on y Ciencia of Spain
and the Institut Municipal d'Investigacions Mediques (IMIM) for the funding for my
thesis. Also thanks to Cold Spring Harbor Labs for several travel grants to attend
their excellent meetings.

Specially during the latest stages of my thesis I have not much time for my friends
so that it is now a good moment to thank you for being there. Specially thanks to David
S\'anchez for your friendship and for your help, and to David Valldosera for your
proximity and wise advice. Also thanks to Josep Vallverd\'u, Roberto Garc\'ia
and Oriol Teixid\'o for conserving our friendship since we first met at university.

As I said before, it was very clear to me how to begin and end this section. Now that
we have arrived at the end, I would like to express my acknowledgments to those that deserve
the honor of closing this section: my parents. What I have reached in my life is
due to your courage and decision. You can be sure that I will never forget my roots.
Thanks to both, for being always my support. Time goes by in my life but you
are always here with me. This work is entirely dedicated to you.

