\documentclass[12pt,oneside]{book}
\usepackage{graphicx}
\usepackage{epsfig}
\usepackage{amssymb}
\usepackage{amsmath}
%\usepackage{amsmath}
\usepackage{amsthm}
\usepackage{longtable}
\usepackage{slashbox}
\usepackage{multirow}
\usepackage{tabularx}
%\usepackage{amssymb}
\usepackage{graphicx}
\usepackage{graphics, t1enc}
\usepackage[spanish]{babel}
\usepackage{makeidx}
%\usepackage{titlesec}
\makeindex
\makeglossary

\usepackage{pstricks}
%\usepackage{showkeys}
\setlength\unitlength{1cm}
\newif\ifAMS
\IfFileExists{amssymb.sty}
{\AMStrue\usepackage{amssymb}}{}
\usepackage{latexsym}



\renewcommand{\baselinestretch}{1.5}
\textwidth=16.5cm
\textheight=21.5cm
\footskip=1.5cm
\oddsidemargin -0.6cm
\voffset =-1cm
%\renewcommand\chaptername{Cap\' {\i}tulo}
%\renewcommand\bibname{Bibliograf\' {\i}a}

\newtheorem{teo}{\sc Teorema}[section]
\newtheorem{ejemplo}{\sc Ejemplo}[section]
\newtheorem{lemma}{\sc Lema}[section]
\newtheorem{obs}{\sc Observaci\'on}[section]
\newtheorem{coro}{\sc Corolario}[section]
\newtheorem{definition}{\sc Definici\'on}[section]
\newtheorem{prop}{\sc Proposici\'on}[section]
\newtheorem{nota}{\sc Nota}[section]
\newcommand{\ese}{\mathcal{S}}

\newtheorem{notacion}{\sc Notaci\'on}[section]

\newcommand{\CC}{\mathbb{C}}
\newcommand{\NN}{\mathbb{N}}
\newcommand{\PP}{\mathbb{P}}
\newcommand{\HH}{\mathbb{H}}
\newcommand{\pp}{\mathbb{\overline{P}}}
\newcommand{\DD}{\mathbb{D}}
\newcommand{\QQ}{\mathbb{Q}}
\newcommand{\RR}{\mathbb{R}}
\newcommand{\ZZ}{\mathbb{Z}}
\newcommand{\EE}{\mathbb{E}}
\newcommand{\TT}{\mathbb{T}}
\newcommand{\XX}{\mathbb{X}}

\newcommand{\II}{\mathbb{I}}

\newcommand{\der}{\mathcal{D}}
\newcommand{\kk}{\mathcal{K}}
\newcommand{\mm}[1]{{\mathcal{M}\/}(#1)}
\newcommand{\nequiv}{{\equiv \hspace*{-3.7mm}/}}
\newcommand{\capa}[1]{\mbox{{\em Cap\/}}(#1)}
\newcommand{\gra}[1]{\mbox{{\em grad\/}}(#1)}
\newcommand{\sop}[1]{\mbox{{\em supp\/}}(#1)}
\newcommand{\esup}[1]{{\mbox{\rm ess\,sup\/}}#1}
\newcommand{\lqqd}{\hfill $\blacksquare$}

\renewcommand{\thefootnote}{\fnsymbol{footnote}}
\newcommand{\ca}[1]{{\em card\/}(#1)}
\newcommand{\dis}[1]{{\em dist\/}(#1)}
\newcommand{\imm}[1]{{\em Im\/}#1}
\newcommand{\ree}[1]{{\em Re\/}#1}
\newcommand{\conv}[1]{{\em conv\/}(#1)}
\newcommand{\uni}{\; \; \overrightarrow{\longrightarrow}\; \;}
\newcommand{\unin}{\overrightarrow{\longrightarrow}}
\newenvironment{ob}{\begin{obs}{\rm} \end{obs}}

\def\bbuildrel#1_#2^#3{\mathrel{
 \mathop{\kern 0pt#1}\limits_{#2}^{#3}}}
\def\bbbuildrel#1_#2{\mathrel{
 \mathop{\kern 0pt#1}\limits_{#2}}}
\def\limsup{\mathop{\overline{\rm lim}}}
\def\liminf{\mathop{\underline{\rm lim}}}
%%%% redise\~nar cap\'itulos
%\usepackage{titlesec}
%\newcommand{\bigrule}{\titlerule[0.3mm]}
%\titleformat{\chapter}[display]
%{\bfseries\Large}
%{
% \titlerule
% \filleft
% \large\chaptertitlename\
% \large\thechapter}
%{0mm}
%{\filleft}
%[\vspace{0.5mm} \bigrule]
%%% formato de encabezado del cap\'itulo
\usepackage[Lenny]{fncychap}
\ChNameVar{\fontsize{12}{13}\usefont{OT1}{phv}{m}{n}\selectfont}
\ChNumVar{\fontsize{50}{52}\usefont{OT1}{ptm}{m}{n}\selectfont}
\ChTitleVar{\normalsize\bfseries}
\ChNameLowerCase
 \ChRuleWidth{1pt}






%capitulos centrados%

%\usepackage{fncychap}
%\renewcommand{\thechapter}{\Roman{chapter}}
%\ChTitleUpperCase
%\ChNameVar{\centering \bfseries}
%\ChNumVar{\centering \bfseries}
%\ChTitleVar{\centering \bfseries}

%numeracion superior%
\usepackage{fancyhdr}
\fancyhf{}
\fancyhead[R]{\thepage}
\renewcommand{\headrulewidth}{0pt}

\usepackage{hyperref}%  no mover de aqu\'i
\begin{document}
%\pagenumbering{roman}%paginas en romano%
%\pagenumbering{roman}
\frontmatter
\thispagestyle{empty}


\vspace*{-0.5cm}

\begin{figure}[h]
\includegraphics[width=3.5cm, height=1.5cm]{./ps/logo3}\ \ \ \ \ \ \ \ \ \ \ \ \ \ \ \ \ \ \ \ \ \ \ \ \ \ \ \ \ \ \ \ \ \ \ \ \ \ \ \ \ \ \ \ \ \ \ \ \ \ \ \ \ \ \ \ \ \ \ \ \ \ \ \ \ \ \ \ \ \ \ \ \ \ \ \ \ \ \ \ \
\includegraphics[width=2.5cm, height=1.5cm]{./ps/escudoua}
  \end{figure}

\vspace{3cm}
\begin{center}
\begin{large}
{\bf\large OPERADORES LINEALES EN ESPACIOS CON M\'ETRICA INDEFINIDA}
\end{large}
\end{center}
\vspace{2.5cm}
\begin{center}
\begin{large}
\textbf{Willian A. Vides Ramos}\\
\end{large}
\end{center}



\vspace{2.5cm}

\begin{center}
\textbf{Trabajo de Maestr\'ia}
\end{center}

\vspace{2.5cm}



%\end{center}



%\hspace{6cm}
%\begin{minipage}[t]{8cm}




\begin{center}
 Marzo, 2014
\end{center}

\newpage 
\thispagestyle{empty}


\begin{center}

\vspace*{-0.5cm}

\begin{figure}[h]
\includegraphics[width=3.5cm, height=1.5cm]{./ps/logo3}\ \ \ \ \ \ \ \ \ \ \ \ \ \ \ \ \ \ \ \ \ \ \ \ \ \ \ \ \ \ \ \ \ \ \ \ \ \ \ \ \ \ \ \ \ \ \ \ \ \ \ \ \ \ \ \ \ \ \ \ \ \ \ \ \ \ \ \ \ \ \ \ \ \ \ \ \ \ \ \ \
\includegraphics[width=1.8cm, height=1.8cm]{./ps/logo2}
  \end{figure}

\vspace{1cm}

\begin{center}
\textbf{UNIVERSIDAD DEL ATL\'ANTICO}

\vspace{-0.1cm}

PROGRAMA DE MATEM�TICAS
\vspace{-0.1cm}

MAESTR\'IA EN CIENCIAS MATEM\'ATICAS

\vspace{-0.1cm}

\end{center}

\vspace{2.5cm}

\begin{large}
{\bf DISE�OS \'OPTIMOS EN LA PRESENCIA DE EFECTOS DE BLOQUES ALEATORIOS}
\end{large}


\vspace{2cm}


\begin{large}
Trabajo Especial de Grado presentado a la Universidad del Atl�ntico por\\
\textbf{EDDIE EDINSON RODRIGUEZ BOSSIO}\\
\vspace{1.5 cm}
como requisito parcial para optar al grado de\\
\textbf{Magister en Ciencias Matem\'aticas}\\

\vspace{1.5cm}

Con la direcci\'on del profesor\\

%\vspace{0.2cm}

\textbf{Dr. rer. nat. Jes�s Alonso Cabrera}
\end{large}


\vspace{1cm}

Septiembre de 2015
\end{center}

\include{jurado}

\chapter*{\textcolor{blue}{Agradecimientos} }

\addstarredchapter{Agradecimientos}

\protect\thispagestyle{empty}


\lettrine[lines=4,loversize=-0.1,lraise=0.1,lhang=.2]{E}{}{n primer lugar doy gracias a Dios dador de la vida y oportunidades;
quien puso en mi camino a personas e instituciones que fueron piezas
claves para este logro en mi vida. Ellos son mi familia, mi tutor
de tesis Dr. rer. nat. Jes�s Alonso Cabrera por compartir sus conocimientos
y consejos, Dr. Jorge Rodriguez, Dr. Alejandro Urieles, profesores
de la maestr\'{i}a, compa�eros de estudio y las instituciones de la
Universidad del Atl�ntico donde curs� mi pregrado y maestr\'{i}a,
y la Universidad del Norte donde realic� mi especializaci�n en matem�ticas
y se llevaron a cabo las asesor\'{i}as de este trabajo de grado, ya
que mi asesor es docente de planta de esta alma mater.}

\vspace{0.2cm}
 \hspace{11cm} Eddie Rodriguez



\thispagestyle{empty}

\ \vfill%

\hfill%
\draftbox{\begin{minipage}[c]{10cm}
%\begin{flushleft}
{%
\sffamily%
\footnotesize
\vspace{2cm}
\noindent 
\chapter*{\color{blue}AGRADECIMIENTOS}
\vspace{1.0cm}
A mis padres y a Diana Patricia, Adriana y Diana Luc\'ia .
%\end{flushleft}
\end{minipage}}

\vfill\ %

\include{resumen}

\renewcommand\indexname{\'INDICE ALFAB\'ETICO}
\renewcommand\bibname{REFERENCIAS BIBLIOGR\'AFICAS}
\renewcommand\contentsname{\'INDICE GENERAL}
\tableofcontents
\markboth{\'INDICE GENERAL}{\'INDICE GENERAL}
\chapter*{LISTA DE S\'IMBOLOS}

\addcontentsline{toc}{chapter}{LISTA DE S\'IMBOLOS}


\vspace{1.0cm}

\begin{tabular}{lll}
$\ZZ, \NN, \RR, \CC $ & & n\'umeros enteros, naturales, reales y complejos respectivamente.\\
$\mathcal{P}$ & & espacio vectorial de los polinomios con coeficientes complejos.\\
$\mathcal{P}_n$& & subespacio vectorial de los polinomios de grado a lo sumo $n$.\\
$\langle\,\cdot,\cdot\,\rangle_{\mu}$ & & producto interno asociado a la medida $\mu$.\\
$\mu$& &medida de Borel.\\
$(X,\beta,\mu)$& & espacio de medida.\\
$\sop{\mu}\;$ & & sorporte de la medida $\mu$.\\
$H_n(x)$ & & $n$-\'esimo polinomio de Hermite.\\
$L_n^{(\alpha)}(x)$ & & $n$-\'esimo polinomio de Laguerre.\\
$P_n^{(\alpha,\beta)}(x)$ & & $n$-\'esimo polinomio de Jacobi.\\
$\mathfrak{B}_n(\lambda;x)$ & & $n$-\'esimo polinomio de Apostol-Bernoulli.\\
$\mathfrak{E}_n(\lambda;x)$ & & $n$-\'esimo polinomio de Apostol-Euler.\\
$\mathfrak{G}_n(\lambda;x)$ & & $n$-\'esimo polinomio de Apostol-Genocchi.\\
$S(m,n)$ & & n\'umeros de Stirling de primera clase.\\
$\mathfrak{S}(n,k)$ & & n\'umeros de Stirling de segunda clase.\\
$C_n^{(a)}(x)$ & & $n$-\'esimo polinomio de Charlier.\\
$\varphi_n (x,c)$ & & $n$-\'esimo polinomio de Bessel generalizado.\\
$\mathcal{F}_n^{[m-1,\alpha]}(x;\lambda;u,v)$ & & $n$-\'esimo polinomio tipo Apostol generalizado.\\



\end{tabular}














\pagenumbering{arabic}
\setcounter{page}{1}
\mainmatter

\chapter*{\textcolor{blue}{Introducci�n}}

\addstarredchapter{Introducci�n}

%\iflettrine


\lettrine[lines=4,loversize=-0.1,lraise=0.1,lhang=.2]{A}{}{l interior de los experimentos estad\'{i}sticos la teor\'{i}a de
los dise�os �ptimos ha sido desarrollada. En general el tema de esta
teor\'{i}a es que para un apropiado modelo, si queremos poner �nfasis
sobre una cualidad particular de los par�metros a estimar, entonces
la configuraci�n experimental deber\'{i}a ser elegida de acuerdo a
ciertos criterios con sentido estad\'{i}stico. En la literatura relacionada
con los dise�os �ptimos, un prominente autor fue Kiefer (1959), el
cu�l present� los principales conceptos, tales como dise�os aproximados
y una variedad de criterios de �ptimalidad para esta rama de los dise�os
de experimentos; Kiefer, en particular dio el nombre $D-$optimalidad
al criterio introducido por Wald (1943), este criterio es el m�s comunmente
aplicado y est� definido en funci�n del determinante de la matriz
de covarianza.\\
}

M�s recientemente son reconocidos los libros de Atkinson y Donev (1992)
y Pukelsheim (1993), donde los autores hacen una presentaci�n estad\'{i}stica
formal de los dise�os �ptimos. El presente trabajo se ha organizado
en tres cap\'{i}tulos: el cap\'{i}tulo uno (Preliminares) contiene
conceptos generales que sirven de apoyo y base a la teor\'{i}a que
se desarrolla en los siguientes dos cap\'{i}tulos. El cap\'{i}tulo
dos trata sobre dise�os �ptimos en la presencia de efectos de bloques
aleatorios, y en el cap\'{i}tulo tres se daran a conocer las conclusiones
y una serie de problemas abiertos para futuras investigaciones relacionadas
con el tema central de este trabajo de investigaci�n. 

\include{apostolB1}
 \chapter{SOBRE LAS DISTINTAS GENERALIZACIONES DE POLINOMIOS  DE APOSTOL-BERNOULLI, DE APOSTOL-EULER Y DE APOSTOL-GENOCCHI}\markboth{CAP\'ITULO 2\, GENERALIZACIONES DE LOS POL. DE A-B, A-E Y A-G}{CAP\'ITULO 2\, GENERALIZACIONES DE LOS POL. DE A-B, DE A-E Y A-G}




%recuperar numeracion arabica%
\def\thechapter{\arabic{chapter}}

Consideremos los polinomios de Apostol-Bernoulli $\mathfrak{B}_n(x;\lambda)$, los polinomios de Apostol-Euler $\mathfrak{E}_n(x;\lambda)$ y los polinomios de Apostol-Genocchi $\mathfrak{G}_n(x;\lambda)$, definidos en (\ref{eq3}), (\ref{eq300}) y (\ref{eq301}) respectivamente. En este cap\'{\i}tulo mostraremos, las diferentes generalizaciones que han tenido estos polinomios hasta llegar a una de las m�s actuales conocidas. Lo anterior implica que estudiaremos fundamentalmente las definiciones y algunas propiedades de los polinomios de Apostol-Euler generalizados, los polinomios de Apostol-Genocchi generalizados, las nuevas familias y clases de polinomios de Apostol-Bernoulli, de Apostol-Euler y de Apostol-Genocchi generalizados y finalmente los polinomios tipo Apostol generalizados. 

\section{Polinomios de Apostol-Bernoulli, de Apostol-Euler y de Apostol-Genocchi generalizados}
Muchas interesantes generalizaciones de los polinomios de Apostol-Bernoulli, Apostol-Euler y Apostol-Genocchi han sido dadas. En part�cular Luo y Srivastava \cite{gen, REF9} introdujeron los polinomios de Apostol-Bernoulli generalizados
$\mathfrak{B}_{n}^{(\alpha)}(x;\lambda)$ de orden $\alpha \in \CC$; Luo \cite{REF6} invent� los polinomios de Apostol-Euler $\mathfrak{E}_{n}^{(\alpha)}(x;\lambda)$ de orden $\alpha \in \CC$ y los polinomios de Genocchi $\mathfrak{G}_{n}^{(\alpha)}(x;\lambda)$ de orden $\alpha \in \CC$ en \cite{Genpolt}

\begin{definition}\label{abp}
Los polinomios de Apostol-Bernoulli generalizados $\mathfrak{B}_{n}^{(\alpha)}(x;\lambda)$, los polinomios de Apostol Euler generalizados $\mathfrak{E}_{n}^{(\alpha)}(x;\lambda)$ y los polinomios de Apostol-Genocchi generalizados $\mathfrak{G}_{n}^{(\alpha)}(x;\lambda)$ de orden $\alpha \in \CC$ est�n definidos por las siguientes funciones generatrices:
\begin{equation}\label{eqae}
\displaystyle\left(\frac{z}{\lambda e^z-1}\right)^{\alpha}e^{xz} =\displaystyle \sum\limits_{n=0}^{\infty}
\mathfrak{B}_{n}^{(\alpha)}(x;\lambda)\frac{z^n}{n!}; \quad (|z|<|\log(\lambda)|),
\end{equation}

\begin{equation}\label{eqae1}
\displaystyle\left(\frac{2}{\lambda e^z+1}\right)^{\alpha}e^{xz} =\displaystyle \sum\limits_{n=0}^{\infty}
\mathfrak{E}_{n}^{(\alpha)}(x;\lambda)\frac{z^n}{n!}; \quad (|t|<|\log(-\lambda)|),
\end{equation}

\begin{equation}\label{eqae2}
\displaystyle\left(\frac{2z}{\lambda e^z+1}\right)^{\alpha}e^{xz} =\displaystyle \sum\limits_{n=0}^{\infty}
\mathfrak{G}_{n}^{(\alpha)}(x;\lambda)\frac{z^n}{n!}; \quad (|z|<|\log(-\lambda)|).
\end{equation}
\end{definition}

\vspace{0.25cm}

\begin{obs}
Los polinomios de Apostol-Bernoulli $\mathfrak{B}_{n}(x;\lambda) = \mathfrak{B}_{n}^{(1)}(x;\lambda)$ y los polinomios de Apostol-Bernoulli generalizados $B_{n}^{(\alpha)}(x) = \mathfrak{B}_{n}^{(\alpha)}(x;1)$.\\
Los polinomios de Apostol-Euler $\mathfrak{E}_{n}(x;\lambda) = \mathfrak{E}_{n}^{(1)}(x;\lambda)$ y los polinomios de Apostol-Euler generalizados $E_{n}^{(\alpha)}(x) = \mathfrak{E}_{n}^{(\alpha)}(x;1)$.\\
Los polinomios de Apostol-Genocchi $\mathfrak{G}_{n}(x;\lambda) = \mathfrak{G}_{n}^{(1)}(x;\lambda)$ y los polinomios de Apostol-Genocchi generalizados $G_{n}^{(\alpha)}(x) = \mathfrak{G}_{n}^{(\alpha)}(x;1)$.
\end{obs}


\section{Nuevas familias de polinomios de Apostol-Euler y de Apostol-Genocchi generalizados}
En este apartado presentaremos dos familias de polinomios de Apostol-Euler y Apostol-Genocchi generalizados  que fueron introducidas por Srivastava et. al.\cite{Srivasnew}. Ellos investigaron las siguientes formas.

\vspace{0.25cm}
\begin{definition}\label{abs}
Sean $a,b,c \in \RR^{+} \quad (a\neq b)$ y $n \in \NN_0:=\NN \cup \{0\}$. Entonces los polinomios de Apostol-Euler generalizados $\mathfrak{E}_n^{(\alpha)}(x;\lambda;a,b,c)$ de orden $\alpha \in \CC$ est�n definidos por la siguiente funci�n generatriz:
\begin{equation}\label{eqaps}
\displaystyle\left(\frac{2}{\lambda b^z+a^z}\right)^{\alpha}c^{xz} =\sum\limits_{n=0}^{\infty}
\mathfrak{E}_n^{(\alpha)}(x;\lambda;a,b,c)\frac{z^n}{n!} 
\end{equation}

\begin{center}
$\displaystyle\quad \left(\left|z\log \left(\frac{b}{a}\right)\right|<|\log(-\lambda)|; 1^\alpha:=1; x\in \RR\right)$
\end{center}
\end{definition}

\vspace{0.25cm}

\begin{definition}\label{abs1}
Sean $a,b,c \in \RR^{+} \quad (a\neq b)$ y $n \in \NN_0:=\NN \cup \{0\}$. Entonces los polinomios Apostol-Genocchi generalizados $\mathfrak{G}_n^{(\alpha)}(x;\lambda;a,b,c)$ de orden $\alpha \in \CC$ est�n definidos por la siguiente funci�n generatriz:
\begin{equation}\label{eqaps1}
\displaystyle\left(\frac{2z}{\lambda b^z+a^z}\right)^{\alpha}c^{xz} =\sum\limits_{n=0}^{\infty}
\mathfrak{G}_n^{(\alpha)}(x;\lambda;a,b,c)\frac{z^n}{n!} 
\end{equation}

\begin{center}
$\displaystyle\quad \left(\left|z\log \left(\frac{b}{a}\right)\right|<|\log(-\lambda)|; 1^\alpha:=1; x\in \RR\right)$
\end{center}
\end{definition}

\vspace{0.25cm}

\section{Nuevas clases de polinomios de Apostol-Bernoulli, de Apostol-Euler y de Apostol-Genocchi generalizados}
En esta secci�n ilustraremos algunas nuevas clases de polinomios de Apostol-Bernoulli generalizados $\mathcal{B}_n^{[m-1,\alpha]}(x,b,c;\lambda)$ que fueron introducidos por Tremblay, Gaboury y Fug\`{e}re \cite{Tremblay1} y nuevas clases de polinomios de Apostol-Euler $\mathcal{E}_n^{[m-1,\alpha]}(x,b,c;\lambda)$ y Apostol-Genocchi generalizados $\mathcal{G}_n^{[m-1,\alpha]}(x,b,c;\lambda)$ que fueron introducidos por los mismos autores en \cite{Tremblay2}.\\

\begin{definition}\label{Gabory}
Para par�metros complejos o reales arbitrarios $\alpha$ y para $b,c \in \RR^{+}$, la nueva clase de polinomios Apostol-Bernoulli generalizados $\mathcal{B}_n^{[m-1,\alpha]}(x,b,c;\lambda)$, la nueva clase de polinomios Apostol-Euler generalizados $\mathcal{E}_n^{[m-1,\alpha]}(x,b,c;\lambda)$ y la nueva clase de polinomios Apostol-Genocchi generalizados $\mathcal{G}_n^{[m-1,\alpha]}(x,b,c;\lambda)$, $m \in \NN$, $\lambda \in \CC$ est�n definidos en un entorno adecuado de $z=0$, a trav�s de la siguientes funciones generatrices:

\begin{equation}\label{eq5}
\displaystyle\left(\frac{z^{m}}{\lambda b^z-\sum\limits_{l=0}^{m-1}\frac{(z\log b)^l}{l!}}\right)^{\alpha}c^{xz} =\displaystyle\sum\limits_{n=0}^{\infty}
\mathcal{B}_n^{[m-1,\alpha]}(x,b,c;\lambda)\frac{z^n}{n!},
\end{equation}

\begin{equation}\label{eq51}
\displaystyle\left(\frac{2^{m}}{\lambda b^z+\sum\limits_{l=0}^{m-1}\frac{(z\log b)^l}{l!}}\right)^{\alpha}c^{xz} =\displaystyle\sum\limits_{n=0}^{\infty}
\mathcal{E}_n^{[m-1,\alpha]}(x,b,c;\lambda)\frac{z^n}{n!},
\end{equation}

\begin{equation}\label{eq52}
\displaystyle\left(\frac{(2z)^{m}}{\lambda b^z+\sum\limits_{l=0}^{m-1}\frac{(z\log b)^l}{l!}}\right)^{\alpha}c^{xz} =\displaystyle\sum\limits_{n=0}^{\infty}
\mathcal{G}_n^{[m-1,\alpha]}(x,b,c;\lambda)\frac{zs^n}{n!}.
\end{equation}

\end{definition}

\vspace{0.25cm}

\begin{obs}
Si hacemos $b=c=e$ en (\ref{eq5}), (\ref{eq51}) y (\ref{eq52}) obtenemos

\begin{equation}\label{eq50}
\displaystyle\left(\frac{z^{m}}{\lambda e^z-\sum\limits_{l=0}^{m-1}\frac{z^l}{l!}}\right)^{\alpha}e^{xz} =\displaystyle\sum\limits_{n=0}^{\infty}
\mathcal{B}_n^{[m-1,\alpha]}(x;\lambda)\frac{z^n}{n!},
\end{equation}

\begin{equation}\label{eq511}
\displaystyle\left(\frac{2^{m}}{\lambda e^z+\sum\limits_{l=0}^{m-1}\frac{z^l}{l!}}\right)^{\alpha}e^{xz} =\displaystyle\sum\limits_{n=0}^{\infty}
\mathcal{E}_n^{[m-1,\alpha]}(x;\lambda)\frac{z^n}{n!},
\end{equation}

\begin{equation}\label{eq522}
\displaystyle\left(\frac{(2z)^{m}}{\lambda e^z+\sum\limits_{l=0}^{m-1}\frac{z^l}{l!}}\right)^{\alpha}e^{xz} =\displaystyle\sum\limits_{n=0}^{\infty}
\mathcal{G}_n^{[m-1,\alpha]}(x;\lambda)\frac{z^n}{n!}.
\end{equation}

\end{obs}

\vspace{0.25cm}

\begin{obs}
Si hacemos $m=1$ en (\ref{eq50}), (\ref{eq511}) y (\ref{eq522}) obtenemos (\ref{eqae}), (\ref{eqae1}) y (\ref{eqae2}) respectivamente.
\end{obs}

\vspace{0.25cm}

\begin{lemma}
 Los polinomios de Bernoulli generalizados $B_{n}^{[m-1]}(x):=\mathcal{B}_{n}^{[m-1,1]}(x;1)$, $m \in \NN$ verifican la siguiente relaci�n:
\begin{equation}\label{eq60}
x^n=\displaystyle\sum\limits_{k=0}^{n}\binom{n}{k}\frac{k!}{(k+m)!}B_{n-k}^{[m-1]}(x).
\end{equation}
\end{lemma}

\vspace{0.25cm}

Para detalles de la prueba del anterior resultados, ver \cite{REF_14}, p.158, Ec.(2.6).


\vspace{0.25cm}

\section{Polinomios tipo Apostol generalizados}
Consideramos en esta secci�n la generalizaci�n m�s importante y m�s actual para los polinomios de Apostol-Bernoulli, Apostol-Euler y Apostol-Gencocchi, la cual fue introducida en 2013 por Lu y Luo en \cite{atp1}.\\

\begin{definition}\label{abp1}
Los polinomios tipo Apostol generalizados $\mathcal{F}_n^{(\alpha)}(x;\lambda;u,v) \quad \alpha \in \NN_0,\lambda, u, v \in \CC)$ de orden $\alpha$ est�n definidos a trav�s de la siguiente funci�n generatriz:
\begin{equation}\label{eqapt1}
\displaystyle\left(\frac{2^{u}z^{v}}{\lambda e^z+1}\right)^{\alpha}e^{xz} =\sum\limits_{n=0}^{\infty}
\mathcal{F}_n^{(\alpha)}(x;\lambda;u,v)\frac{z^n}{n!}; \quad (|z|<|log(-\lambda)|),
\end{equation}
donde
\begin{equation}
\mathcal{F}_n^{(\alpha)}(\lambda;u,v)=:\mathcal{F}_n^{(\alpha)}(0;\lambda;u,v) 
\end{equation}
denotan los tambi�n llamados n�meros de tipo Apostol de orden $\alpha$.
\end{definition}

\vspace{0.25cm}

So that, by comparing \textit{Definition \ref{abp1}} and \textit{Definition \ref{abp}} 
\begin{align}
& \mathfrak{B}_n^{(\alpha)}(x;\lambda)=(-1)^{\alpha}\mathcal{F}_n^{(\alpha)}(x;-\lambda;0,1),\\
& \mathfrak{E}_n^{(\alpha)}(x;\lambda)=\mathcal{F}_n^{(\alpha)}(x;\lambda;1,0),\\
& \mathfrak{G}_n^{(\alpha)}(x;\lambda)=\mathcal{F}_n^{(\alpha)}(x;\lambda;1,1).
\end{align}

\vspace{0.25cm}


 
\chapter{UNA NUEVA CLASE DE POLINOMIOS TIPO APOSTOL GENERALIZADOS}
\markboth{CAP\'ITULO 3\, UNA NUEVA CLASE DE POL. TIPO APOSTOL GENERALIZADOS}{CAP\'ITULO 3\quad UNA NUEVA CLASE DE POL. TIPO APOSTOL GENERALIZADOS}
\label{AB3}

%recuperar numeracion arabica%
\def\thechapter{\arabic{chapter}}
En este cap�tulo se introducir\'a una nueva clase de polinomios tipo Apostol generalizados denotada por $\mathcal{Q}_n^{[m-1,\alpha]}(x,b,c;\lambda;u,v)$, as\'{\i} como tambi\'en, se estudiar\'an sus propiedades b�sicas incluidas su f�rmula de recurrencia y ecuaci�n diferencial que satisfacen. Para finalizar, se establecer�n f�rmulas de conexi�n entre $\mathcal{Q}_n^{[m-1,\alpha]}(x,b,c;\lambda;u,v)$ y los polinomios ortogonales cl�sicos, con los polinomios de Genocchi, Charlier, Bessel, Bernoulli generalizados y los n�meros de Stirling de segunda clase.\\


La siguiente definici�n que generaliza la \textit{Definici�n \ref{Gabory}} en una sola expresi�n y adem�s proporciona una extensi�n natural para los polinomios tipo Apostol generalizados $\mathcal{F}_n^{(\alpha)}(x;\lambda;u,v) \quad (\alpha \in \NN_0,\lambda, u, v \in \CC)$ de orden $\alpha$ definidos en (\ref{eqapt1}).


\medskip

\begin{definition}\label{abp14}
La nueva clase para los polomios tipo Apostol generalizados $\mathcal{Q}_n^{[m-1,\alpha]}(x,b,c;\lambda;u,v)$ con $(\alpha, u, v\in \CC$ y $b, c \in \RR^{+})$ de orden $\alpha$ est�n definidos por la siguiente funci�n generatriz:
\begin{equation}\label{newpol}
\displaystyle\left(\frac{(2^ut^v)^{m}}{\lambda b^t+\sum\limits_{l=0}^{m-1}\frac{(t\log b)^l}{l!}}\right)^{\alpha}c^{xt} =\displaystyle\sum\limits_{n=0}^{\infty}
\mathcal{Q}_n^{[m-1,\alpha]}(x,b,c;\lambda;u,v)\frac{t^n}{n!}; \quad |t\log b|<|log(-\lambda)|,
\end{equation}
donde
\begin{equation}\label{newnumbers}
\mathcal{Q}_n^{[m-1,\alpha]}(b,c;\lambda;u,v):=\mathcal{Q}_n^{[m-1,\alpha]}(0,b,c;\lambda;u,v)
\end{equation}
denotan los tambi�n llamados la nueva clase de n�meros de tipo Apostol generalizados de orden $\alpha$.
\end{definition}

\vspace{0.25cm}

As� que, comparando la \textit{Definici�n \ref{abp14}} y la \textit{Definici�n \ref{Gabory}} tenemos: 
\begin{align}
& \mathcal{B}_n^{[m-1,\alpha]}(x;b,c;\lambda)=(-1)^{\alpha}\mathcal{Q}_n^{[m-1,\alpha]}(x;b,c;-\lambda;0,1),\\
& \mathcal{E}_n^{[m-1,\alpha]}(x;b,c;\lambda)=\mathcal{Q}_n^{[m-1,\alpha]}(x;b,c;\lambda;1,0),\\
& \mathcal{G}_n^{[m-1,\alpha]}(x;b,c;\lambda)=\mathcal{Q}_n^{[m-1,\alpha]}(x;b,c;\lambda;1,1).
\end{align}

\vspace{0.25cm}

\begin{obs}
$\mathcal{Q}_n^{(\alpha)}(x;\lambda;u,v):=\mathcal{Q}_n^{[0,\alpha]}(x,e,e;\lambda;u,v)$
\end{obs}

\vspace{0.25cm}

Analogamente, usando las ideas de la \textit{Definici�n \ref{abp1}}, podemos escribir las \textit{Definiciones \ref{abs} and \ref{abs1}} en una sola expresi�n, como se sigue:

\vspace{0.25cm}

\begin{definition}\label{newpol2}
Sean $a,b,c \in \RR^{+} \quad (a\neq b)$, $u, v \in \CC$ y $n \in \NN_0:=\NN \cup \{0\}$. Entonces la nueva clase de polinomios tipo Apostol basado en Srivastava $\mathcal{Q}_n^{(\alpha)}(x;\lambda;a,b,c;u,v)$ de orden $\alpha \in \CC$ est� definida a trav�s de la siguiente funci�n generatriz:
\begin{equation}\label{eqnewpol2}
\displaystyle\left(\frac{2^uz^v}{\lambda b^z+a^z}\right)^{\alpha}c^{xz} =\sum\limits_{n=0}^{\infty}
\mathcal{Q}_n^{(\alpha)}(x;\lambda;a,b,c;u,v)\frac{z^n}{n!} 
\end{equation}

\begin{center}
$\displaystyle\quad \left(\left|z\log \left(\frac{b}{a}\right)\right|<|\log(-\lambda)|; 1^\alpha:=1; x\in \RR\right)$
\end{center}
\end{definition}

\vspace{0.25cm}

Los nuevos polinomios en (\ref{eqnewpol2}) tambi�n son una nueva clase de polinomios tipo Apostol al igual los definidos en (\ref{newpol}). Estos verifican lo siguiente:

\begin{align}
& \mathfrak{B}_n^{(\alpha)}(x;\lambda;a,b,c):=(-1)^{\alpha}\mathcal{Q}_n^{(\alpha)}(x;-\lambda;a,b,c;0,1),\\
& \mathfrak{E}_n^{(\alpha)}(x;\lambda;a,b,c)=\mathcal{Q}_n^{(\alpha)}(x;\lambda;a,b,c;1,0),\\
& \mathfrak{G}_n^{(\alpha)}(x;\lambda;a,b,c)=\mathcal{Q}_n^{(\alpha)}(x;\lambda;a,b,c;1,1).
\end{align}


\vspace{0.5cm}
\section{Algunas propiedades b�sicas de $\mathcal{Q}_n^{[m-1,\alpha]}(x,b,c;\lambda;u,v)$}\label{3d1}
En est� secci�n estudiaremos algunas de las propiedades b�sicas de los polinomios $\mathcal{Q}_n^{[m-1,\alpha]}(x,b,c;\lambda;u,v)$.

\medskip

\begin{teo}\label{3cau1}
La nueva clase de polinomios tipo Apostol generalizados $\mathcal{Q}_n^{[m-1,\alpha]}(x,b,c;\lambda;u,v)$ satisface las siguientes relaciones:
\begin{equation}\label{eq8}
\mathcal{Q}_n^{[m-1,\alpha+\beta]}(x+y,b,c;\lambda;u,v)=\displaystyle\sum\limits_{j=0}^{n}
\binom{n}{j}\mathcal{Q}_j^{[m-1,\alpha]}(x,b,c;\lambda;u,v)\mathcal{Q}_{n-j}^{[m-1,\beta]}(y,b,c;\lambda;u,v),
\end{equation}
\begin{equation}\label{eq9}
\mathcal{Q}_n^{[m-1,\alpha]}(x+y,b,c;\lambda;u,v)=\displaystyle\sum\limits_{j=0}^{n}
\binom{n}{j}\mathcal{Q}_j^{[m-1,\alpha]}(y,b,c;\lambda;u,v)(x\log c)^{n-j}.
\end{equation}
\end{teo}

\begin{proof}
 Por (\ref{newpol}), tenemos:

\begin{eqnarray*}
\displaystyle \sum\limits_{n=0}^{\infty} \mathcal{Q}_n^{[m-1,\alpha+\beta]}(x+y,b,c;\lambda;u,v)\frac{t^n}{n!} &=& \displaystyle\left(\frac{(2^ut^v)^{m}}{\lambda b^t+\sum\limits_{l=0}^{m-1}\frac{(t\log b)^l}{l!}}\right)^{\alpha+\beta}c^{(x+y)t}\\
&=& \displaystyle\left(\frac{(2^ut^v)^{m}}{\lambda b^t+\sum\limits_{l=0}^{m-1}\frac{(t\log b)^l}{l!}}\right)^{\alpha}c^{xt}\left(\frac{(2^ut^v)^{m}}{\lambda b^t+\sum\limits_{l=0}^{m-1}\frac{(t\log b)^l}{l!}}\right)^{\beta}c^{yt}\\
&=& \displaystyle\sum\limits_{n=0}^{\infty}
\mathcal{Q}_n^{[m-1,\alpha]}(x,b,c;\lambda;u,v)\frac{t^n}{n!}\sum\limits_{n=0}^{\infty}
\mathcal{Q}_n^{[m-1,\beta]}(y,b,c;\lambda;u,v)\frac{t^n}{n!}\\
&=& \displaystyle \sum\limits_{n=0}^{\infty}\sum\limits_{k=0}^{n}\binom{n}{k}\mathcal{Q}_k^{[m-1,\alpha]}(x,b,c;\lambda;u,v)\mathcal{Q}_{n-k}^{[m-1,\beta]}(y,b,c;\lambda;u,v)\frac{t^n}{n!}.\\
\end{eqnarray*}
Comparando los coeficientes de $\frac{t^n}{n!}$ en ambos miembros de la ecuaci�n anterior, llegamos a (\ref{eq8}) rapidamente.

\vspace{0.25cm}

Haciendo $\beta = 0$ en (\ref{eq8}) tenemos:
\begin{eqnarray*}
\displaystyle \sum\limits_{n=0}^{\infty} \mathcal{Q}_n^{[m-1,\alpha]}(x+y,b,c;\lambda;u,v)\frac{t^n}{n!} &=& \displaystyle\left(\frac{(2^ut^v)^{m}}{\lambda b^t+\sum\limits_{l=0}^{m-1}\frac{(t\log b)^l}{l!}}\right)^{\alpha}c^{(x+y)t}\\
&=& \displaystyle\left(\frac{(2^ut^v)^{m}}{\lambda b^t+\sum\limits_{l=0}^{m-1}\frac{(t\log b)^l}{l!}}\right)^{\alpha}c^{yt}c^{xt}\\
&=& \displaystyle\left(\frac{(2^ut^v)^{m}}{\lambda b^t+\sum\limits_{l=0}^{m-1}\frac{(t\log b)^l}{l!}}\right)^{\alpha}c^{yt}e^{\log c^{xt}}\\
&=& \displaystyle\left(\frac{(2^ut^v)^{m}}{\lambda b^t+\sum\limits_{l=0}^{m-1}\frac{(t\log b)^l}{l!}}\right)^{\alpha}c^{yt}e^{(x\log c)t}\\
&=& \displaystyle\sum\limits_{n=0}^{\infty}
\mathcal{Q}_n^{[m-1,\alpha]}(y,b,c;\lambda;u,v)\frac{t^n}{n!}\sum\limits_{n=0}^{\infty}
(x\log c)^n\frac{t^n}{n!}\\
&=& \displaystyle \sum\limits_{n=0}^{\infty}\sum\limits_{k=0}^{n}\binom{n}{k}\mathcal{Q}_k^{[m-1,\alpha]}(y,b,c;\lambda;u,v)(x\log c)^{n-k}\frac{t^n}{n!}.
\end{eqnarray*}

Comparando los coeficientes de $\frac{t^n}{n!}$ en ambos miembros de la ecuaci�n anterior, llegamos a (\ref{eq9}).
\end{proof}

\vspace{0.25cm}

\begin{coro}\label{3cau12}
La nueva clase de polinomios tipo Apostol generalizados $\mathcal{Q}_n^{[m-1,\alpha]}(x,b,c;\lambda;u,v)$ satisfacen las siguientes relaciones:
\begin{align}
\label{eq823}
\mathcal{Q}_n^{[m-1,\alpha]}(x,b,c;\lambda;u,v)= & \displaystyle\sum\limits_{j=0}^{n}
\binom{n}{j}\mathcal{Q}_{n-j}^{[m-1,\alpha-1]}(b,c;\lambda;u,v)\mathcal{Q}_j^{[m-1]}(x,b,c;\lambda;u,v),\\
\label{eq349}
\mathcal{Q}_n^{[m-1,\alpha]}(x,b,c;\lambda;u,v)= & \displaystyle\sum\limits_{j=0}^{n}
\binom{n}{j}\mathcal{Q}_{n-j}^{[m-1,\alpha]}(b,c;\lambda;u,v)(x\log c)^{j},
\end{align}
donde $\mathcal{Q}_{n}^{[m-1,\alpha]}(b,c;\lambda;u,v)$ son los tambi�n llamados nueva clase de n�meros tipo Apostol generalizados de orden $\alpha$ definidos por (\ref{newnumbers}).
\end{coro}

\medskip

\begin{proof}
Haciendo $\alpha=1$, $\beta=\alpha-1$ y $y=0$ en la Ec.(\ref{eq8}) del \textit{Teorema \ref{3cau1}}, tenemos (\ref{eq823}). Cuando $y=0$ en la Ec.(\ref{eq9}) of the \textit{Teorema \ref{3cau1}} y aplicando (\ref{s4}) podemos obtener (\ref{eq349}).
\end{proof}

\medskip
\begin{obs}
Tomando $m=1$ y $b=c=e$ en el \textit{Corolario \ref{3cau12}}, obtenemos:
\begin{equation}\label{eq8231}
\mathcal{F}_n^{(\alpha)}(x;\lambda;u,v)=\displaystyle\sum\limits_{j=0}^{n}
\binom{n}{j}\mathcal{F}_{n-j}^{(\alpha-1)}(\lambda;u,v)\mathcal{F}_j(x;\lambda;u,v),
\end{equation}
la cual corresponde a la Ec.(3.5) de \cite{atp1},
\begin{equation}\label{eq3491}
\mathcal{F}_n^{(\alpha)}(x;\lambda;u,v)=\displaystyle\sum\limits_{j=0}^{n}
\binom{n}{j}\mathcal{F}_{n-j}^{(\alpha)}(\lambda;u,v)x^{j},
\end{equation}
la cual corresponde a la Ec.(3.1) de \cite{atp1}.
\end{obs}

\medskip

\begin{teo}\label{par}
La nueva clase de polinomios tipo Apostol generalizados $\mathcal{Q}_n^{[m-1,\alpha]}(x,b,c;\lambda;u,v)$ satisfacen la siguiente relaci�n:
\begin{equation}\label{e7}
\displaystyle \frac{\partial^p}{\partial x^p} \mathcal{Q}_n^{[m-1,\alpha]}(x,b,c;\lambda;u,v)= \frac{n!}{(n-p)!}(\log c)^{p}\mathcal{Q}_{n-p}^{[m-1,\alpha]}(x,b,c;\lambda;u,v).
\end{equation}
\end{teo}

\begin{proof}
Usando la ecuaci�n (\ref{eq349}), tenemos\\
\begin{equation}\nonumber
\mathcal{Q}_n^{[m-1,\alpha]}(x,b,c;\lambda;u,v)=\displaystyle\sum\limits_{k=0}^{n}
\binom{n}{k}\mathcal{Q}_k^{[m-1,\alpha]}(b,c;\lambda;u,v)(x\log c)^{n-k}
\end{equation}
diferenciando parcialmente $p$ veces respecto a la variable $x$ en ambos miembros, se sigue que:
\begin{eqnarray*}
\displaystyle \frac{\partial^p}{\partial x^p}\mathcal{Q}_n^{[m-1,\alpha]}(x,b,c;\lambda;u,v) &=& \frac{\partial^p}{\partial x^p}\left[\displaystyle \sum\limits_{k=0}^{n}\binom{n}{k}\mathcal{Q}_k^{[m-1,\alpha]}(b,c;\lambda;u,v)\displaystyle (x\log c)^{n-k}\right]\\
&=& \frac{\partial^p}{\partial x^p}\left[\displaystyle \sum\limits_{k=0}^{n}\binom{n}{k}\mathcal{Q}_k^{[m-1,\alpha]}(b,c;\lambda;u,v)\displaystyle (\log c)^{n-k}x^{n-k}\right]\\
&=& \displaystyle \sum\limits_{k=0}^{n-p}\binom{n}{k}\mathcal{Q}_k^{[m-1,\alpha]}(b,c;\lambda;u,v)(\log c)^{n-k} \frac{\partial^p}{\partial x^p}\left(x^{n-k}\right)\\
&=& \displaystyle \sum\limits_{k=0}^{n-p}\binom{n}{k}\mathcal{Q}_k^{[m-1,\alpha]}(b,c;\lambda;u,v)(\log c)^{n-k} \frac{(n-k)!}{(n-p-k)!} x^{n-p-k}\\
\end{eqnarray*}
desarrollando el combinatorio, obtenemos
\begin{eqnarray*}
\displaystyle \frac{\partial^p}{\partial x^p}\mathcal{Q}_n^{[m-1,\alpha]}(x,b,c;\lambda;u,v) &=& \displaystyle \sum\limits_{k=0}^{n-p}\mathcal{Q}_k^{[m-1,\alpha]}(b,c;\lambda;u,v)(\log c)^{n-k} \frac{(n-k)!}{(n-p-k)!}\frac{n!}{k!(n-k)!} x^{n-p-k}\\
&=& \displaystyle \sum\limits_{k=0}^{n-p}\mathcal{Q}_k^{[m-1,\alpha]}(b,c;\lambda;u,v)(\log c)^{n-k} \frac{(n-p)!}{k!(n-p-k)!}\frac{n!}{(n-p)!} x^{n-p-k}\\
&=& \displaystyle \frac{n!}{(n-p)!}\sum\limits_{k=0}^{n-p}\binom{n-p}{k}\mathcal{Q}_k^{[m-1,\alpha]}(b,c;\lambda;u,v)(\log c)^{p} (x\log c)^{n-p-k}\\
&=& \displaystyle \frac{n!}{(n-p)!}(\log c)^{p}\sum\limits_{k=0}^{n-p}\binom{n-p}{k}\mathcal{Q}_k^{[m-1,\alpha]}(b,c;\lambda;u,v) (x\log c)^{n-p-k}\\
&=& \displaystyle \frac{n!}{(n-p)!}(\log c)^{p}\mathcal{Q}_{n-p}^{[m-1,\alpha]}(x,b,c;\lambda;u,v).
\end{eqnarray*}
\end{proof}

\vspace{0.25cm}

Al establecer $m=1$ y $b=c=e$ en el \textit{Teorema \ref{par}}, tenemos el siguiente corolario.

\vspace{0.25cm}

\begin{coro}\label{par23}
\begin{equation}\label{e723}
\displaystyle \frac{\partial^p}{\partial x^p} \mathcal{F}_n^{(\alpha)}(x;\lambda;u,v)= \frac{n!}{(n-p)!}\mathcal{F}_{n-p}^{(\alpha)}(x;\lambda;u,v).
\end{equation}
la cual es justamente la Ec.(2.10) de \cite{atp1}. 
\end{coro}

\vspace{0.25cm}

\begin{teo}\label{par2}
La nueva clase de polinomios tipo Apostol generalizados $\mathcal{Q}_n^{[m-1,\alpha]}(x,b,c;\lambda;u,v)$ satisfacen la siguiente relaci�n:
\begin{equation}\label{e72}
t^p\left(\frac{(2^ut^v)^{m}}{\lambda b^t+\sum\limits_{l=0}^{m-1}\frac{(t\log b)^l}{l!}}\right)^{\alpha}c^{xt} =\sum\limits_{n=0}^{\infty}\frac{n!}{(n-p)!}\mathcal{Q}_{n-p}^{[m-1,\alpha]}(x,b,c;\lambda;u,v)\frac{t^n}{n!}.
\end{equation}
\end{teo}

\vspace{0.25cm}

\begin{proof}
Consideremos (\ref{newpol}) y derivemos parcialmente $p$ veces en ambos miembros respecto a la variable $x$, entonces tenemos:
\begin{align*}
t^p(\log c)^p\left(\frac{(2^ut^v)^{m}}{\lambda b^t+\sum\limits_{l=0}^{m-1}\frac{(t\log b)^l}{l!}}\right)^{\alpha}c^{xt} & =\sum\limits_{n=0}^{\infty}\frac{\partial^p}{\partial x^p}\left[\mathcal{Q}_{n}^{[m-1,\alpha]}(x,b,c;\lambda;u,v)\right]\frac{t^n}{n!}\\
\intertext{usando el \textit{Teorema \ref{par}} se sigue que}
t^p(\log c)^p\left(\frac{(2^ut^v)^{m}}{\lambda b^t+\sum\limits_{l=0}^{m-1}\frac{(t\log b)^l}{l!}}\right)^{\alpha}c^{xt} & =\sum\limits_{n=0}^{\infty}(\log c)^p\frac{n!}{(n-p)!}\mathcal{Q}_{n}^{[m-1,\alpha]}(x,b,c;\lambda;u,v)\frac{t^n}{n!}\\
\intertext{como $c\in \RR^{+}$ podemos multiplicar por $(\log c)^{-p}$ en ambos miembros y tenemos el final de la prueba:}
t^p\left(\frac{(2^ut^v)^{m}}{\lambda b^t+\sum\limits_{l=0}^{m-1}\frac{(t\log b)^l}{l!}}\right)^{\alpha}c^{xt} & =\sum\limits_{n=0}^{\infty}\frac{n!}{(n-p)!}\mathcal{Q}_{n-p}^{[m-1,\alpha]}(x,b,c;\lambda;u,v)\frac{t^n}{n!}.
\end{align*}
\end{proof}

\vspace{0.25cm}

\begin{lemma}
La nueva clase de polinomios tipo Apostol generalizados $\mathcal{Q}_n^{[m-1,\alpha]}(x,b,c;\lambda;u,v)$ satisface la siguiente relaci�n:
\begin{align}\label{e3}
& \nonumber\lambda \mathcal{Q}_n^{[m-1,\alpha]}(x+1,b,c;\lambda;u,v)+\mathcal{Q}_n^{[m-1,\alpha]}(x,b,c;\lambda;u,v)\\
& =\sum\limits_{k=0}^{n}\binom{n}{k}\binom{k}{v}v!(\log c)^v\mathcal{Q}_{k-v}^{[m-1,\alpha]}(x,b,c;\lambda;u,v)\mathcal{Q}_{n-k}^{(-1)}(0;\lambda;1,c,a;u,v),
\end{align}
donde $\mathcal{Q}_{n-k}^{(-1)}(0;\lambda;1,c,a;u,v)$ son los polinomios tipo Apostol generalizados definidos por (\ref{eqnewpol2}).
\end{lemma}

\vspace{0.25cm}

\begin{proof}
Por (\ref{newpol}) y el \textit{Teorema \ref{par}} cuando $p=1$, tenemos

\begin{align*}
& \displaystyle \sum\limits_{n=0}^{\infty}\left[\lambda \mathcal{Q}_n^{[m-1,\alpha]}(x+1,b,c;\lambda;u,v)+\mathcal{Q}_n^{[m-1,\alpha]}(x,b,c;\lambda;u,v)\right]\frac{t^n}{n!}\\
& = \displaystyle\lambda \sum\limits_{n=0}^{\infty}\mathcal{Q}_n^{[m-1,\alpha]}(x+1,b,c;\lambda;u,v)\frac{t^n}{n!}+\sum\limits_{n=0}^{\infty}\mathcal{Q}_n^{[m-1,\alpha]}(x,b,c;\lambda;u,v)\frac{t^n}{n!}\\
& =\lambda \displaystyle\left(\frac{(2^ut^v)^{m}}{\lambda b^t+\sum\limits_{l=0}^{m-1}\frac{(t\log b)^l}{l!}}\right)^{\alpha}c^{(x+1)t}+ \displaystyle\left(\frac{(2^ut^v)^{m}}{\lambda b^t+\sum\limits_{l=0}^{m-1}\frac{(t\log b)^l}{l!}}\right)^{\alpha}c^{xt}\\ 
& = \displaystyle\left(\frac{(2^ut^v)^{m}}{\lambda b^t+\sum\limits_{l=0}^{m-1}\frac{(t\log b)^l}{l!}}\right)^{\alpha}c^{xt}(\lambda c^t+1)=\left[2^ut^v\displaystyle\left(\frac{(2^ut^v)^{m}}{\lambda b^t+\sum\limits_{l=0}^{m-1}\frac{(t\log b)^l}{l!}}\right)^{\alpha}c^{xt}\right]\left(\displaystyle \frac{2^ut^v}{\lambda c^t+1}\right)^{(-1)}\\ 
& = \displaystyle 2^u\frac{\partial^v}{\partial x^v}\left[\left(\frac{(2^ut^v)^{m}}{\lambda b^t+\sum\limits_{l=0}^{m-1}\frac{(t\log b)^l}{l!}}\right)^{\alpha}c^{xt}\right] \left(\frac{2^ut^v}{\lambda c^t+1}\right)^{(-1)}c^{0t}\\
&= \displaystyle \frac{\partial^v}{\partial x^v}\left[ \sum\limits_{n=0}^{\infty}\mathcal{Q}_n^{[m-1,\alpha]}(x,b,c;\lambda;u,v)\frac{t^n}{n!}\right]\left(\sum\limits_{n=0}^{\infty}\mathcal{Q}_n^{(-1)}(0;\lambda;1,c,a;u,v)\frac{t^n}{n!}\right)\\
&=\displaystyle \left[ \sum\limits_{n=0}^{\infty}\frac{\partial^v}{\partial x^v}\left(\mathcal{Q}_n^{[m-1,\alpha]}(x,b,c;\lambda;u,v)\right)\frac{t^n}{n!}\right]\left(\sum\limits_{n=0}^{\infty}\mathcal{Q}_n^{(-1)}(0;\lambda;1,c,a;u,v)\frac{t^n}{n!}\right)\\ 
& = \displaystyle \left[ \sum\limits_{n=0}^{\infty}\frac{n!}{(n-v)!}(\log c)^v\mathcal{Q}_{n-v}^{[m-1,\alpha]}(x,b,c;\lambda;u,v)\frac{t^n}{n!}\right]\left(\sum\limits_{n=0}^{\infty}\mathcal{Q}_n^{(-1)}(0;\lambda;1,c,a;u,v)\frac{t^n}{n!}\right)\\
& =\displaystyle \sum\limits_{n=0}^{\infty} \sum\limits_{k=0}^{n}\binom{n}{k}\binom{k}{v}v!(\log c)^v\mathcal{Q}_{k-v}^{[m-1,\alpha]}(x,b,c;\lambda;u,v)\mathcal{Q}_{n-k}^{(-1)}(0;\lambda;1,c,a;u,v)\frac{t^n}{n!}.
\end{align*}
Comparando los coeficientes de $\frac{t^n}{n!}$ en ambos miembros de la ecuaci�n anterior, podemos obtener la identidad (\ref{e3}).
\end{proof}

\vspace{0.25cm}

\begin{teo}
La relaci�n
{\small
\begin{align}
& \nonumber\mathcal{Q}_n^{[m-1,\alpha]}(x+y,b,c;\lambda;u,v)\\
& \label{eqls}=\displaystyle \frac{1}{2}\sum\limits_{j=0}^{n}\binom{n}{j}\left[\lambda \mathcal{Q}_{n}^{[m-1,\alpha]}(y+1,b,c;\lambda;u,v)+(\log b)^{j}\mathcal{Q}_{n}^{[m-1,\alpha]}(y,b,c;\lambda;u,v)\right]\mathfrak{E}_{n-j}(x;\lambda).
\end{align}
}
se mantiene entre los polinomios $\mathcal{Q}_n^{[m-1,\alpha]}(x,b,c;\lambda;u,v)$ y los polinomios de Apostol-Euler $\mathfrak{E}_{n}(x;\lambda)$ definidos por (\ref{eqae})(con $\alpha = 1$).
\end{teo}

\begin{proof}
En primer lugar, sustituimos (\ref{lemeu}) en el lado derecho de la (\ref{eq9}) y obtenemos
\begin{align}
& \nonumber\mathcal{Q}_n^{[m-1,\alpha]}(x+y,b,c;\lambda;u,v)\\
&\nonumber =\displaystyle \sum\limits_{k=0}^{n}\binom{n}{k}\mathcal{Q}_k^{[m-1,\alpha]}(y,b,c;\lambda;u,v)(\log b)^{n-k}\frac{1}{2}\left[\lambda \sum\limits_{j=0}^{n-k}\binom{n-k}{j}\mathfrak{E}_{j}(x;\lambda)+\mathfrak{E}_{n-k}(x;\lambda)\right]\\
& \nonumber=\displaystyle \sum\limits_{k=0}^{n}\binom{n}{k}\mathcal{Q}_k^{[m-1,\alpha]}(y,b,c;\lambda;u,v)(\log b)^{n-k}\frac{1}{2}\lambda \sum\limits_{j=0}^{n-k}\binom{n-k}{j}\mathfrak{E}_{j}(x;\lambda)\\
& \label{eq89} + \displaystyle\sum\limits_{k=0}^{n}\binom{n}{k}\mathcal{Q}_k^{[m-1,\alpha]}(y,b,c;\lambda;u,v)(\log b)^{n-k}\frac{1}{2}\mathfrak{E}_{n-k}(x;\lambda).
\end{align}
La primera suma en (\ref{eq89}) es igual a

\begin{align}
&\nonumber\displaystyle \sum\limits_{k=0}^{n}\binom{n}{k}\mathcal{Q}_k^{[m-1,\alpha]}(y,b,c;\lambda;u,v)(\log b)^{n-k}\frac{1}{2}\lambda \sum\limits_{j=0}^{n-k}\binom{n-k}{j}\mathfrak{E}_{j}(x;\lambda) \\
&=\nonumber\displaystyle \sum\limits_{k=0}^{n}\sum\limits_{j=0}^{n-k}\binom{n}{k}(\log b)^{n-k}\frac{1}{2}\lambda \binom{n-k}{j}\mathcal{Q}_k^{[m-1,\alpha]}(y,b,c;\lambda;u,v)\mathfrak{E}_{j}(x;\lambda)\\
&=\nonumber\displaystyle \sum\limits_{j=0}^{n}\sum\limits_{k=0}^{n-j}\binom{n}{n-k}\binom{n-k}{j}(\log b)^{n-k}\frac{\lambda}{2}{F}_k^{[m-1,\alpha]}(y,b,c;\lambda;u,v)\mathfrak{E}_{j}(x;\lambda)\\
&=\nonumber\displaystyle \sum\limits_{j=0}^{n}\sum\limits_{k=0}^{n-j}\binom{n}{n-k}\binom{n-j}{n-j-k}(\log b)^{n-k}\frac{\lambda}{2}\mathcal{Q}_k^{[m-1,\alpha]}(y,b,c;\lambda;u,v)\mathfrak{E}_{j}(x;\lambda)
\end{align}

\begin{align}
&=\nonumber\displaystyle \sum\limits_{j=0}^{n}\frac{\lambda}{2}\binom{n}{j}\mathfrak{E}_{j}(x;\lambda)\sum\limits_{k=0}^{n-j}\binom{n-j}{k}\mathcal{Q}_k^{[m-1,\alpha]}(y,b,c;\lambda;u,v)(\log b)^{n-k}\\
&=\label{eq90}\displaystyle \sum\limits_{j=0}^{n}\frac{\lambda}{2}\binom{n}{j}\mathfrak{E}_{j}(x;\lambda)\mathcal{Q}_{n-j}^{[m-1,\alpha]}(y+1,b,c;\lambda;u,v).
\end{align}
Esto se logra invirtiendo el orden de la sumatoria, usando la ecuaci�n (\ref{eq9}) (cuando $x\longrightarrow 1; n\longrightarrow n-j$) y la aplicaci�n de las siguientes identidades elementales de combinatorios:
\begin{equation}
\binom{m}{l}\binom{l}{m}=\binom{m}{n}\binom{m-n}{l-n}\,\,\,\,\,(m\geq l\geq n: l,m,n \in \NN_{0}),
\end{equation}
\begin{equation}\label{comb}
\binom{n}{n-k}=\binom{n}{k}\,\,\,\,\,(n\geq k: k,n \in \NN_{0}),
\end{equation}
la segunda suma en (\ref{eq89}) es igual a
\begin{align}
& \nonumber\displaystyle \sum\limits_{k=0}^{n}\binom{n}{k}\mathcal{Q}_k^{[m-1,\alpha]}(y,b,c;\lambda;u,v)(\log b)^{n-k}\frac{1}{2}\mathfrak{E}_{n-k}(x;\lambda)\\
& \label{eq91}\displaystyle=\sum\limits_{k=0}^{n}\binom{n}{k}\mathcal{Q}_{n-k}^{[m-1,\alpha]}(y,b,c;\lambda;u,v)(\log b)^{k}\frac{1}{2}\mathfrak{E}_{k}(x;\lambda).
\end{align}
Combinando (\ref{eq90}) y (\ref{eq91})
\begin{align*}
& \mathcal{Q}_n^{[m-1,\alpha]}(x+y,b,c;\lambda;u,v)\\ 
& =\displaystyle \sum\limits_{j=0}^{n}\frac{\lambda}{2}\binom{n}{j}\mathfrak{E}_{j}(x;\lambda)\mathcal{Q}_{n-j}^{[m-1,\alpha]}(y+1,b,c;\lambda;u,v)\\
& \displaystyle +\sum\limits_{j=0}^{n}\binom{n}{j}\mathcal{Q}_{n-j}^{[m-1,\alpha]}(y,b,c;\lambda;u,v)(\log b)^{j}\frac{1}{2}\mathfrak{E}_{j}(x;\lambda)\\
& =\displaystyle \frac{1}{2}\sum\limits_{j=0}^{n}\binom{n}{j}\left[\lambda \mathcal{Q}_{n-j}^{[m-1,\alpha]}(y+1,b,c;\lambda;u,v)+(\log b)^{j}\mathcal{Q}_{n-j}^{[m-1,\alpha]}(y,b,c;\lambda;u,v)\right]\mathfrak{E}_{j}(x;\lambda)\\
& =\displaystyle \frac{1}{2}\sum\limits_{j=0}^{n}\binom{n}{j}\left[\lambda \mathcal{Q}_{n}^{[m-1,\alpha]}(y+1,b,c;\lambda;u,v)+(\log b)^{j}\mathcal{Q}_{n}^{[m-1,\alpha]}(y,b,c;\lambda;u,v)\right]\mathfrak{E}_{n-j}(x;\lambda).
\end{align*}
\end{proof}



\vspace{0.5cm}
\section{F�rmula de recurrencia y ecuaci�n diferencial}\label{3d2}
En esta secci�n ilustraremos la f�rmula de recurrencia y ecuaci�n diferencial que satisface $\mathcal{Q}_n^{[m-1,\alpha]}(x,b,c;\lambda;u,v)$ usando las identidades fundamentales de las series de potencias y las definiciones de polinomios de Appell.

\subsection{Ecuaci�n diferencial para $\mathcal{Q}_n^{[m-1,\alpha]}(x,b,c;\lambda;u,v)$}
\begin{teo}\label{teoeqdif}
Los polinomios $\mathcal{Q}_n^{[m-1,\alpha]}(x,b;\lambda;u,0)$ satisfacen la ecuaci�n diferencial:
\begin{equation}\label{eqappell}
\displaystyle \frac{\alpha_{n-1}}{(n-1)!}y^{(n)}+\frac{\alpha_{n-2}}{(n-2)!}y^{(n-1)}+\cdots+\frac{\alpha_{1}}{1!}y^{''}+(x+\alpha_0)y^{'}-ny=0,
\end{equation}
donde los coeficientes num�ricos $\alpha_{k}$, $k = 1,2,\cdots,n-1$ est�n definidos por
{\footnotesize
\begin{align}
\displaystyle \alpha_n& \nonumber =\frac{-\alpha(\log b)\lambda}{2^m}\frac{ n!}{(n+m)!}\mathcal{G}_{n+m}^{[m-1]}(1,b,b;\lambda)-\frac{\alpha(\log b)}{2^m}\left(\sum\limits_{k=0}^{m-1}\frac{(\log b)^k}{k!}\frac{n!}{(n+m-k)!}\mathcal{G}_{n+m-k}^{[m-1]}(b;\lambda)\right)\\
& +\frac{\alpha(\log b)^mn!}{2^m(m-1)!(n+1)!}\mathcal{G}_{n+1}^{[m-1]}(b;\lambda)
\end{align}
}
\end{teo}

\vspace{0.25cm}

\begin{proof}
Consideremos $\mathcal{Q}_n^{[m-1,\alpha]}(x,b,c;\lambda;u,v)$ y definamos
\begin{equation*}
\mathcal{Q}_n^{[m-1,\alpha]}(x,b;\lambda;u,v):=\mathcal{Q}_n^{[m-1,\alpha]}(x,b,e;\lambda;u,v)
\end{equation*}
\begin{equation*}
\mathcal{G}_n^{[m-1,\alpha]}(x,b;\lambda):=\mathcal{Q}_n^{[m-1,\alpha]}(x,b;\lambda;1,1).
\end{equation*}
Por otro lado tenemos que $\mathcal{Q}_n^{[m-1,\alpha]}(x,b;\lambda;u,v)$ es un polinomio de Appell ya que verifica (\ref{ruler}) como se mostr� en (\ref{e7}), por lo tanto satisface la ecuaci�n diferencial (\ref{eqappell}). A continuaci�n determinaremos las contantes $\alpha_k$, identificando $A(t)$ tenemos:
{\footnotesize
\begin{align*}
& A(t)=\displaystyle\left(\frac{(2^ut^v)^{m}}{\lambda b^t+\sum\limits_{l=0}^{m-1}\frac{(t\log b)^l}{l!}}\right)^{\alpha}\\
\intertext{{\normalsize derivando en ambos miembros, obtenemos}}
& A'(t)=\displaystyle\left[\frac{mv2^{mu}t^{mv-1}\left(\lambda b^t+\sum\limits_{l=0}^{m-1}\frac{(t\log b)^l}{l!}\right)-2^{mu}t^{mv}(\log b)\left(\lambda b^t+\sum\limits_{l=0}^{m-2}\frac{(t\log b)^{l}}{l!}\right)}{\left(\lambda b^t+\sum\limits_{l=0}^{m-1}\frac{(t\log b)^l}{l!}\right)^2}\right]\alpha\left(\frac{(2^ut^v)^{m}}{\lambda b^t+\sum\limits_{l=0}^{m-1}\frac{(t\log b)^l}{l!}}\right)^{\alpha-1}\\
& A'(t)=\displaystyle\left[\frac{mvt^{-1}\left(\lambda b^t+\sum\limits_{l=0}^{m-1}\frac{(t\log b)^l}{l!}\right)-(\log b)\left(\lambda b^t+\sum\limits_{l=0}^{m-2}\frac{(t\log b)^{l}}{l!}\right)}{\left(\lambda b^t+\sum\limits_{l=0}^{m-1}\frac{(t\log b)^l}{l!}\right)}\right]\alpha\left(\frac{(2^ut^v)^{m}}{\lambda b^t+\sum\limits_{l=0}^{m-1}\frac{(t\log b)^l}{l!}}\right)^{\alpha}\\
\intertext{{\normalsize dividiendo por $A(t)$, tenemos}}
& \displaystyle \frac{A'(t)}{A(t)} =\alpha\left[\frac{mvt^{-1}\left(\lambda b^t+\sum\limits_{l=0}^{m-1}\frac{(t\log b)^l}{l!}\right)-(\log b)\left(\lambda b^t+\sum\limits_{l=0}^{m-2}\frac{(t\log b)^{l}}{l!}\right)}{\left(\lambda b^t+\sum\limits_{l=0}^{m-1}\frac{(t\log b)^l}{l!}\right)}\right]\\
\intertext{{\normalsize haciendo $v=0$, tenemos}}
& \displaystyle \frac{A'(t)}{A(t)} =-\alpha\left[\frac{(\log b)\left(\lambda b^t+\sum\limits_{l=0}^{m-2}\frac{(t\log b)^{l}}{l!}\right)}{\left(\lambda b^t+\sum\limits_{l=0}^{m-1}\frac{(t\log b)^l}{l!}\right)}\right]\\
\intertext{{\normalsize separando la adici�n en el numerador del miembro de la derecha tenemos}}
& \displaystyle \frac{A'(t)}{A(t)} =-\alpha\left[\frac{(\log b)\lambda b^t}{\left(\lambda b^t+\sum\limits_{l=0}^{m-1}\frac{(t\log b)^l}{l!}\right)}\right]-\alpha\left[\frac{(\log b)\left(\sum\limits_{l=0}^{m-2}\frac{(t\log b)^{l}}{l!}\right)}{\left(\lambda b^t+\sum\limits_{l=0}^{m-1}\frac{(t\log b)^l}{l!}\right)}\right]\\
& \displaystyle \frac{A'(t)}{A(t)} =\frac{-\alpha(\log b)\lambda}{2^m}t^{-m}\left[\frac{ 2^mt^mb^t}{\left(\lambda b^t+\sum\limits_{l=0}^{m-1}\frac{(t\log b)^l}{l!}\right)}\right]-\alpha(\log b)\left[\sum\limits_{k=0}^{m-2}\frac{(t\log b)^k}{k!\left(\lambda b^t+\sum\limits_{l=0}^{m-1}\frac{(t\log b)^l}{l!}\right)}\right]\\
& \displaystyle \frac{A'(t)}{A(t)} =\frac{-\alpha(\log b)\lambda}{2^m}t^{-m}\left[\frac{ 2^mt^mb^t}{\left(\lambda b^t+\sum\limits_{l=0}^{m-1}\frac{(t\log b)^l}{l!}\right)}\right]-\frac{\alpha(\log b)}{2^m}\left[\sum\limits_{k=0}^{m-2}\frac{(\log b)^k}{k!}t^{k-m}\frac{2^mt^m}{\left(\lambda b^t+\sum\limits_{l=0}^{m-1}\frac{(t\log b)^l}{l!}\right)}\right]\\
\intertext{{\normalsize usando el \textit{Teorema \ref{par2}}}, tenemos}
& \displaystyle \frac{A'(t)}{A(t)}=\frac{-\alpha(\log b)\lambda}{2^m}\sum\limits_{n=0}^{\infty}\frac{ n!}{(n+m)!}\mathcal{G}_{n+m}^{[m-1]}(1,b,b;\lambda)\frac{t^n}{n!}-\frac{\alpha(\log b)}{2^m}\left(\sum\limits_{k=0}^{m-2}\frac{(\log b)^k}{k!}\sum\limits_{n=0}^{\infty}\frac{ n!}{(n+m-k)!}\mathcal{G}_{n+m-k}^{[m-1]}(b;\lambda)\frac{t^n}{n!}\right)\\
& \displaystyle \frac{A'(t)}{A(t)}=\displaystyle \sum\limits_{n=0}^{\infty}\left[\frac{-\alpha(\log b)\lambda}{2^m}\frac{ n!}{(n+m)!}\mathcal{G}_{n+m}^{[m-1]}(1,b,b;\lambda)-\frac{\alpha(\log b)}{2^m}\left(\sum\limits_{k=0}^{m-2}\frac{(\log b)^k}{k!}\frac{n!}{(n+m-k)!}\mathcal{G}_{n+m-k}^{[m-1]}(b;\lambda)\right)\right]\frac{t^n}{n!}\\
\intertext{{\normalsize usando (\ref{eqalpha}), tenemos que}}
\displaystyle \alpha_n& =\frac{-\alpha(\log b)\lambda}{2^m}\frac{ n!}{(n+m)!}\mathcal{G}_{n+m}^{[m-1]}(1,b,b;\lambda)-\frac{\alpha(\log b)}{2^m}\left(\sum\limits_{k=0}^{m-2}\frac{(\log b)^k}{k!}\frac{n!}{(n+m-k)!}\mathcal{G}_{n+m-k}^{[m-1]}(b;\lambda)\right)\\
\intertext{{\normalsize finalmente}}
\displaystyle \alpha_n& =\frac{-\alpha(\log b)\lambda}{2^m}\frac{ n!}{(n+m)!}\mathcal{G}_{n+m}^{[m-1]}(1,b,b;\lambda)-\frac{\alpha(\log b)}{2^m}\left(\sum\limits_{k=0}^{m-1}\frac{(\log b)^k}{k!}\frac{n!}{(n+m-k)!}\mathcal{G}_{n+m-k}^{[m-1]}(b;\lambda)\right)\\
& +\frac{\alpha(\log b)}{2^m}\frac{(\log b)^{m-1}}{(m-1)!}\frac{n!}{(n+1)!}\mathcal{G}_{n+1}^{[m-1]}(b;\lambda)
\end{align*}
}
\end{proof}


\vspace{0.25cm}

Especialmente haciendo $u=1$ en el \textit{Teorema \ref{teoeqdif}} tenemos el siguiente corolario.

\vspace{0.25cm}

\begin{coro}\label{eqdif2}
Los polinomios $\mathcal{E}_n^{[m-1,\alpha]}(x,b;\lambda)$ satisfacen la ecuaci�n diferencial:
\begin{equation}\label{eqappell2}
\displaystyle \frac{\alpha_{n-1}}{(n-1)!}y^{(n)}+\frac{\alpha_{n-2}}{(n-2)!}y^{(n-1)}+\cdots+\frac{\alpha_{1}}{1!}y^{''}+(x+\alpha_0)y^{'}-ny=0,
\end{equation}
donde los coeficientes num�ricos $\alpha_{k}$, $k = 1,2,\cdots,n-1$ est�n definidos por
{\footnotesize
\begin{align}
\displaystyle \alpha_n& \nonumber =\frac{-\alpha(\log b)\lambda}{2^m}\frac{ n!}{(n+m)!}\mathcal{G}_{n+m}^{[m-1]}(1,b,b;\lambda)-\frac{\alpha(\log b)}{2^m}\left(\sum\limits_{k=0}^{m-1}\frac{(\log b)^k}{k!}\frac{n!}{(n+m-k)!}\mathcal{G}_{n+m-k}^{[m-1]}(b;\lambda)\right)\\
& +\frac{\alpha(\log b)^mn!}{2^m(m-1)!(n+1)!}\mathcal{G}_{n+1}^{[m-1]}(b;\lambda)
\end{align}
}
\end{coro}

\vspace{0.25cm}

\begin{obs}
Si hacemos $m=1$ y $b=e$ en el \textit{Corolario \ref{eqdif2}} tenemos que los polinomios $\mathcal{E}_n^{(\alpha)}(x;\lambda)$ satisfacen la ecuaci�n diferencial:
\begin{equation}\label{eqappell3}
\displaystyle \frac{\alpha_{n-1}}{(n-1)!}y^{(n)}+\frac{\alpha_{n-2}}{(n-2)!}y^{(n-1)}+\cdots+\frac{\alpha_{1}}{1!}y^{''}+(x+\alpha_0)y^{'}-ny=0,
\end{equation}
donde los coeficientes num�ricos $\alpha_{k}$, $k = 1,2,\cdots,n-1$ est�n definidos por
\begin{align}
\displaystyle \alpha_n&  =-\frac{\alpha\lambda\mathcal{G}_{n+1}(1;\lambda)}{2(n+1)} \nonumber
\end{align}
la cual es justamente el \textit{Corolario 2.8} de \cite{atp1}
\end{obs}


\vspace{0.25cm}

Haciendo $m=1$ y $b=e$ en el \textit{Teorema \ref{teoeqdif}} tenemos el siguiente corolario.

\vspace{0.25cm}

\begin{coro}\label{teoeqdif1}
Los polinomios $\mathcal{F}_n^{(\alpha)}(x;\lambda;u,0)$ satisfacen la ecuaci�n diferencial:
\begin{equation}\label{eqappell1}
\displaystyle \frac{\alpha_{n-1}}{(n-1)!}y^{(n)}+\frac{\alpha_{n-2}}{(n-2)!}y^{(n-1)}+\cdots+\frac{\alpha_{1}}{1!}y^{''}+(x+\alpha_0)y^{'}-ny=0,
\end{equation}
donde los coeficientes num�ricos $\alpha_{k}$, $k = 1,2,\cdots,n-1$ est�n definidos por
\begin{align}
\displaystyle \alpha_n&  =-\frac{\alpha\lambda\mathcal{G}_{n+1}(1;\lambda)}{2(n+1)}
\intertext{que es justamente el \textit{Teorema 2.7} de \cite{atp1}} \nonumber
\end{align}
\end{coro}

\vspace{0.25cm}

\subsection{F�rmula de recurrencia para $\mathcal{Q}_n^{[m-1,\alpha]}(x,b,c;\lambda;u,v)$}
\begin{teo}
La nueva clase de polinomios tipo Apostol generalizados verifica la siguiente relaci�n de recurrencia
\begin{align}
& \nonumber \left(\log \frac{b^{\alpha}}{c^x} \right)\mathcal{Q}_{n}^{[m-1,\alpha]}(x,b,c;\lambda;u,v)=\left(\frac{\alpha mv}{n+1}-1\right)\mathcal{Q}_{n+1}^{[m-1,\alpha]}(x,b,c;\lambda;u,v)\\
& \label{relrec} +\frac{\alpha (\log b)^m}{2^{um}(m-1)!}\frac{n!}{(n+vm-m+1)!}\mathcal{Q}_{n+vm-m+1}^{[m-1,\alpha+1]}(x,b,c;\lambda;u,v)
\end{align}
\end{teo}


\vspace{0.25cm}

\begin{proof}
Consideremos 
{\footnotesize
\begin{align*}
& \displaystyle\left(\frac{(2^ut^v)^{m}}{\lambda b^t+\sum\limits_{l=0}^{m-1}\frac{(t\log b)^l}{l!}}\right)^{\alpha}c^{xt} =\displaystyle\sum\limits_{n=0}^{\infty}
\mathcal{Q}_n^{[m-1,\alpha]}(x,b,c;\lambda;u,v)\frac{t^n}{n!}\\
\intertext{{\normalsize derivando respecto a $t$ en ambos miembros, tenemos}}
& \displaystyle\frac{\partial}{\partial t}\left[\left(\frac{(2^ut^v)^{m}}{\lambda b^t+\sum\limits_{l=0}^{m-1}\frac{(t\log b)^l}{l!}}\right)^{\alpha}\right]c^{xt} +\left(\frac{(2^ut^v)^{m}}{\lambda b^t+\sum\limits_{l=0}^{m-1}\frac{(t\log b)^l}{l!}}\right)^{\alpha}\frac{\partial}{\partial t}\left[c^{xt}\right] =\displaystyle\frac{\partial}{\partial t}\left[\sum\limits_{n=0}^{\infty}
\mathcal{Q}_n^{[m-1,\alpha]}(x,b,c;\lambda;u,v)\frac{t^n}{n!}\right]\\
\intertext{{\normalsize resolviendo las derivadas ambos miembros, se sigue que}}
& \displaystyle\left[\frac{mv2^{mu}t^{mv-1}\left(\lambda b^t+\sum\limits_{l=0}^{m-1}\frac{(t\log b)^l}{l!}\right)-2^{mu}t^{mv}(\log b)\left(\lambda b^t+\sum\limits_{l=0}^{m-2}\frac{(t\log b)^{l}}{l!}\right)}{\left(\lambda b^t+\sum\limits_{l=0}^{m-1}\frac{(t\log b)^l}{l!}\right)^2}\right]c^{xt}\alpha\left(\frac{(2^ut^v)^{m}}{\lambda b^t+\sum\limits_{l=0}^{m-1}\frac{(t\log b)^l}{l!}}\right)^{\alpha-1}\\
& +\displaystyle \left(\frac{(2^ut^v)^{m}}{\lambda b^t+\sum\limits_{l=0}^{m-1}\frac{(t\log b)^l}{l!}}\right)^{\alpha}xc^{xt}(\log c)=\sum\limits_{n=0}^{\infty}
\mathcal{Q}_{n+1}^{[m-1,\alpha]}(x,b,c;\lambda;u,v)\frac{t^n}{n!}\\
\intertext{{\normalsize resolviendo la primera suma y expresando como serie de potencia la segunda suma, tenemos}}
& \displaystyle\left[\frac{mvt^{-1}\left(\lambda b^t+\sum\limits_{l=0}^{m-1}\frac{(t\log b)^l}{l!}\right)-(\log b)\left(\lambda b^t+\sum\limits_{l=0}^{m-2}\frac{(t\log b)^{l}}{l!}\right)}{(2^ut^v)^{m}}\right]c^{xt}\alpha\left(\frac{(2^ut^v)^{m}}{\lambda b^t+\sum\limits_{l=0}^{m-1}\frac{(t\log b)^l}{l!}}\right)^{\alpha+1}\\
& +\displaystyle \sum\limits_{n=0}^{\infty}
x(\log c)\mathcal{Q}_{n}^{[m-1,\alpha]}(x,b,c;\lambda;u,v)\frac{t^n}{n!}=\sum\limits_{n=0}^{\infty}
\mathcal{Q}_{n+1}^{[m-1,\alpha]}(x,b,c;\lambda;u,v)\frac{t^n}{n!}\\
\intertext{{\normalsize expresando la primera suma en t�rminos de $m-1$, se sigue que}}
& \displaystyle\left[\frac{mvt^{-1}\left(\lambda b^t+\sum\limits_{l=0}^{m-1}\frac{(t\log b)^l}{l!}\right)-(\log b)\left(\lambda b^t+\sum\limits_{l=0}^{m-1}\frac{(t\log b)^{l}}{l!}\right)+(\log b)\left(\frac{(t\log b)^{m-1}}{(m-1)!}\right)}{(2^ut^v)^{m}}\right]c^{xt}\alpha\left(\frac{(2^ut^v)^{m}}{\lambda b^t+\sum\limits_{l=0}^{m-1}\frac{(t\log b)^l}{l!}}\right)^{\alpha+1}\\
& +\displaystyle \sum\limits_{n=0}^{\infty}
x(\log c)\mathcal{Q}_{n}^{[m-1,\alpha]}(x,b,c;\lambda;u,v)\frac{t^n}{n!}=\sum\limits_{n=0}^{\infty}
\mathcal{Q}_{n+1}^{[m-1,\alpha]}(x,b,c;\lambda;u,v)\frac{t^n}{n!}\\
\intertext{{\normalsize ahora, por aplicaci�n del \textit{Teorema \ref{par2}} y expresando la primera suma como serie de potencia, tenemos}}
& \displaystyle \sum\limits_{n=0}^{\infty}\frac{\alpha mv}{n+1}\mathcal{Q}_{n+1}^{[m-1,\alpha]}(x,b,c;\lambda;u,v)\frac{t^n}{n!}-\sum\limits_{n=0}^{\infty}\alpha (\log b)\mathcal{Q}_{n}^{[m-1,\alpha]}(x,b,c;\lambda;u,v)\frac{t^n}{n!}\\
& +\sum\limits_{n=0}^{\infty}\frac{\alpha (\log b)^m}{2^{um}(m-1)!}\frac{n!}{(n+vm-m+1)!}\mathcal{Q}_{n+vm-m+1}^{[m-1,\alpha+1]}(x,b,c;\lambda;u,v)\frac{t^n}{n!}+\displaystyle \sum\limits_{n=0}^{\infty}
x(\log c)\mathcal{Q}_{n}^{[m-1,\alpha]}(x,b,c;\lambda;u,v)\frac{t^n}{n!}\\
& =\sum\limits_{n=0}^{\infty}
\mathcal{Q}_{n+1}^{[m-1,\alpha]}(x,b,c;\lambda;u,v)\frac{t^n}{n!}\\
\intertext{{\normalsize finalmente por identidad de series de potencias concluimos la prueba}}
& \left(\alpha \log b-x\log c \right)\mathcal{Q}_{n}^{[m-1,\alpha]}(x,b,c;\lambda;u,v)=\left(\frac{\alpha mv}{n+1}-1\right)\mathcal{Q}_{n+1}^{[m-1,\alpha]}(x,b,c;\lambda;u,v)\\
& +\frac{\alpha (\log b)^m}{2^{um}(m-1)!}\frac{n!}{(n+vm-m+1)!}\mathcal{Q}_{n+vm-m+1}^{[m-1,\alpha+1]}(x,b,c;\lambda;u,v)
\end{align*}
}
\end{proof}





\vspace{0.5cm}
\section{F�rmulas de conexi�n  entre $\mathcal{Q}_n^{[m-1,\alpha]}(x,b,c;\lambda;u,v)$ y otros polinomios y n�meros}\label{3e}
En esta secci�n se estudian algunas conexiones entre los polinomios $\mathcal{Q}_n^{[m-1,\alpha]}(x,b,c;\lambda;u,v)$ y otras familias de polinomios como los polinomios de Genocchi, los polinomios de Jacobi, los polinomios de Laguerre, los polinomios de Hermite, los polinomios de Bernoulli generalizados $B_{n}^{[m-1]}(x)$, los polinomios de Charlier y los polinomios de Bessel. Por �ltimo, se establece la relaci�n de la nueva clase de polinomios tipo Apostol generalizados con los n�meros de Stirling de segunda clase, todo ello siguiendo las mismas ideas usadas en la prueba de los teoremas de la secci�n anterior, y tambi�n se utilizan algunas de las siguientes propiedades b�sicas de las sumatorias:
\begin{equation}\label{s4}
\displaystyle \sum\limits_{j=0}^{n}A_{n-j}B_{n}=\sum\limits_{j=0}^{n}A_{j}B_{n-j},
\end{equation}
\begin{equation}\label{s3}
\displaystyle \sum\limits_{j=0}^{n-k}A_{j}=\sum\limits_{j=k}^{n}A_{n-j},
\end{equation}
\begin{equation}\label{s2}
\displaystyle \sum\limits_{j=0}^{n}\sum\limits_{k=0}^{n}a_{j}a_{k}=\left(\sum\limits_{j=0}^{n}a_{j}\right)\left(\sum\limits_{k=0}^{n}a_{k}\right) 
\end{equation}
y
\begin{equation}\label{s1}
\displaystyle \sum\limits_{j=0}^{n}\sum\limits_{k=0}^{n-j}a_{jk}=\sum\limits_{k=0}^{n}\sum\limits_{j=0}^{n-k}a_{jk}.\\
\end{equation}

\vspace{0.25cm}

\begin{teo}\label{teogen}
La relaci�n
{\footnotesize
\begin{align}\label{eq11}
& \nonumber\mathcal{Q}_n^{[m-1,\alpha]}(x+y,b,c;\lambda;u,v)\\
&=\displaystyle\frac{1}{2}\sum\limits_{k=0}^{n}
\frac{(\log c)^{k}}{k+1}\left[\binom{n}{k}\mathcal{Q}_{n-k}^{[m-1,\alpha]}(y,b,c;\lambda;u,v)+\sum\limits_{j=k}^{n}\binom{n}{j}\binom{j}{k}\mathcal{Q}_{n-j}^{[m-1,\alpha]}(y,b,c;\lambda;u,v)(\log c)^{j-k}\right]G_{k+1}(x).
\end{align}
}
se mantiene entre los polinomios $\mathcal{Q}_n^{[m-1,\alpha]}(x,b,c;\lambda;u,v)$ y los polinomios de Genocchi $G_{n}(x)$ definidos por (\ref{genpol23}). 
\end{teo}

\vspace{0.25cm}

\begin{proof} 
{\footnotesize
Por sustituci�n de (\ref{eq12}) en el lado derecho de (\ref{eq9}), tenemos
\begin{align*}
&\mathcal{Q}_n^{[m-1,\alpha]}(x+y,b,c;\lambda;u,v)\\ &=\displaystyle\sum
\limits_{j=0}^{n}
\binom{n}{j}\mathcal{Q}_j^{[m-1,\alpha]}(y,b,c;\lambda;u,v)\frac{(\log c)^{n-j}}{2(n-j+1)}\left[\sum\limits_{k=0}^{n-j}\binom{n-j+1}{k+1}G_{k+1}(x)+G_{n-j+1}(x)\right]\\
& =\displaystyle\sum
\limits_{j=0}^{n}
\binom{n}{j}\mathcal{Q}_j^{[m-1,\alpha]}(y,b,c;\lambda;u,v))\frac{(\log c)^{n-j}}{2(n-j+1)}\sum\limits_{k=0}^{n-j}\binom{n-j+1}{k+1}G_{k+1}(x)\\
&+ \displaystyle\sum
\limits_{j=0}^{n}
\binom{n}{j}\mathcal{Q}_j^{[m-1,\alpha]}(y,b,c;\lambda;u,v)\frac{(\log c)^{n-j}}{2(n-j+1)}G_{n-j+1}(x)\\
\intertext{usando (\ref{s2}) tenemos}
& =\displaystyle\sum
\limits_{j=0}^{n}\sum\limits_{k=0}^{n-j}\binom{n}{j}\mathcal{Q}_j^{[m-1,\alpha]}(y,b,c;\lambda;u,v)\frac{(\log c)^{n-j}}{2(n-j+1)}\binom{n-j+1}{k+1}G_{k+1}(x)\\
&+ \displaystyle\sum
\limits_{j=0}^{n}
\binom{n}{j}\mathcal{Q}_j^{[m-1,\alpha]}(y,b,c;\lambda;u,v)\frac{(\log c)^{n-j}}{2(n-j+1)}G_{n-j+1}(x)\\
\intertext{usando (\ref{s1}) en la primera suma y cambiando $j$ por $k$ se sigue que}
& =\displaystyle\sum
\limits_{k=0}^{n}\sum\limits_{j=0}^{n-k}\binom{n}{j}\mathcal{Q}_j^{[m-1,\alpha]}(y,b,c;\lambda;u,v)\frac{(\log c)^{n-j}}{2(n-j+1)}\binom{n-j+1}{k+1}G_{k+1}(x)\\
&+ \displaystyle\sum
\limits_{k=0}^{n}
\binom{n}{k}\mathcal{Q}_j^{[m-1,\alpha]}(y,b,c;\lambda;u,v))\frac{(\log c)^{n-k}}{2(n-k+1)}G_{n-k+1}(x)\\
\intertext{ahora, usando (\ref{s3}) en la primera suma y (\ref{s4}) en la segunda suma se sigue que}
& =\displaystyle\sum
\limits_{k=0}^{n}\sum\limits_{j=k}^{n}\binom{n}{n-j}\mathcal{Q}_j^{[m-1,\alpha]}(y,b,c;\lambda;u,v)\frac{(\log c)^{j}}{2(j+1)}\binom{j+1}{k+1}G_{k+1}(x)\\
&+ \displaystyle\sum
\limits_{k=0}^{n}
\binom{n}{n-k}\mathcal{Q}_{n-k}^{[m-1,\alpha]}(y,b,c;\lambda;u,v)\frac{(\log c)^{k}}{2(k+1)}G_{k+1}(x)\\
\intertext{amplificando por $\displaystyle \binom{j}{k}$ en la primera suma, tenemos}
& =\displaystyle\sum
\limits_{k=0}^{n}\sum\limits_{j=k}^{n}\binom{n}{j}\binom{j}{k}\mathcal{Q}_{n-j}^{[m-1,\alpha]}(y,b,c;\lambda;u,v)\frac{(\log c)^{j}}{2(j+1)}\binom{j+1}{k+1}\binom{j}{k}^{-1}G_{k+1}(x)\\
&+ \displaystyle\sum
\limits_{k=0}^{n}
\binom{n}{n-k}\mathcal{Q}_{n-k}^{[m-1,\alpha]}(y,b,c;\lambda;u,v)\frac{(\log c)^{k}}{2(k+1)}G_{k+1}(x)\\
\intertext{desarrolando el combinatorio en la primera suma}
& =\displaystyle\sum
\limits_{k=0}^{n}\sum\limits_{j=k}^{n}\binom{n}{j}\binom{j}{k}\mathcal{Q}_{n-j}^{[m-1,\alpha]}(y,b,c;\lambda;u,v)\frac{(\log c)^{j}}{2(k+1)}G_{k+1}(x)\\
&+ \displaystyle\sum
\limits_{k=0}^{n}
\binom{n}{k}\mathcal{Q}_{n-k}^{[m-1,\alpha]}(y,b,c;\lambda;u,v)\frac{(\log c)^{k}}{2(k+1)}G_{k+1}(x),\\
\intertext{finalmente, 
factorizando tenemos}
& =\displaystyle\frac{1}{2}\sum\limits_{k=0}^{n}
\frac{(\log c)^{k}}{k+1}\left[\sum\limits_{j=k}^{n}\binom{n}{j}\binom{j}{k}\mathcal{Q}_{n-j}^{[m-1,\alpha]}(y,b,c;\lambda;u,v)(\log c)^{j-k}+\binom{n}{k}\mathcal{Q}_{n-k}^{[m-1,\alpha]}(y,b,c;\lambda;u,v)\right]G_{k+1}(x).
\end{align*}
}
\end{proof}

\vspace{0.25cm}

Tomando $\lambda=:-\lambda, u=0$ y $v=1$ en el \textit{Teorema \ref{teogen}}, y luego multiplicando por $(-1)^\alpha$ en ambos lados del resultado, tenemos el siguiente corolario.

\vspace{0.25cm}

\begin{coro}\label{corber}
{\footnotesize
\begin{align}\label{eq121}
& \nonumber\mathcal{B}_n^{[m-1,\alpha]}(x+y,b,c;\lambda)\\
&=\displaystyle\frac{1}{2}\sum\limits_{k=0}^{n}
\frac{(\log c)^{k}}{k+1}\left[\binom{n}{k}\mathcal{B}_{n-k}^{[m-1,\alpha]}(y,b,c;\lambda)+\sum\limits_{j=k}^{n}\binom{n}{j}\binom{j}{k}\mathcal{B}_{n-j}^{[m-1,\alpha]}(y,b,c;\lambda)(\log c)^{j-k}\right]G_{k+1}(x).
\end{align}
}
\end{coro}

\vspace{0.25cm}

\begin{obs}
Si hacemos $b=c=e$ en el \textit{Corolario \ref{corber}} tenemos
{\footnotesize
\begin{align}\label{eq113}
& \nonumber\mathcal{B}_n^{[m-1,\alpha]}(x+y;\lambda)\\
&=\displaystyle\frac{1}{2}\sum\limits_{k=0}^{n}
\frac{1}{k+1}\left[\binom{n}{k}\mathcal{B}_{n-k}^{[m-1,\alpha]}(y;\lambda)+\sum\limits_{j=k}^{n}\binom{n}{j}\binom{j}{k}\mathcal{B}_{n-j}^{[m-1,\alpha]}(y;\lambda)\right]G_{k+1}(x).
\end{align}
}
\end{obs}

\vspace{0.25cm}

Tomando $u=1$ y $v=0$ en el \textit{Teorema \ref{teogen}}, tenemos el siguiente corolario.

\vspace{0.25cm}

\begin{coro}\label{coreul}
{\footnotesize
\begin{align}\label{eq1187}
& \nonumber\mathcal{E}_n^{[m-1,\alpha]}(x+y,b,c;\lambda)\\
&=\displaystyle\frac{1}{2}\sum\limits_{k=0}^{n}
\frac{(\log c)^{k}}{k+1}\left[\binom{n}{k}\mathcal{E}_{n-k}^{[m-1,\alpha]}(y,b,c;\lambda)+\sum\limits_{j=k}^{n}\binom{n}{j}\binom{j}{k}\mathcal{E}_{n-j}^{[m-1,\alpha]}(y,b,c;\lambda)(\log c)^{j-k}\right]G_{k+1}(x).
\end{align}
}
\end{coro}

\vspace{0.25cm}

\begin{obs}
Si hacemos $b=c=e$ en el \textit{Corolario \ref{coreul}} tenemos
{\footnotesize
\begin{align}\label{eq1123}
& \nonumber\mathfrak{E}_n^{[m-1,\alpha]}(x+y;\lambda)\\
&=\displaystyle\frac{1}{2}\sum\limits_{k=0}^{n}
\frac{1}{k+1}\left[\binom{n}{k}\mathfrak{E}_{n-k}^{[m-1,\alpha]}(y;\lambda)+\sum\limits_{j=k}^{n}\binom{n}{j}\binom{j}{k}\mathfrak{E}_{n-j}^{[m-1,\alpha]}(y;\lambda)\right]G_{k+1}(x).
\end{align}
} la cual es justamente la Ec.(4.1) de \cite{Ming}
\end{obs}

\vspace{0.25cm}

Al establecer $u=1$ y $v=1$ en el \textit{Teorema \ref{teogen}}, tenemos el siguiente corolario.

\vspace{0.25cm}

\begin{coro}\label{corgen}
{\footnotesize
\begin{align}\label{eq115}
& \nonumber\mathcal{G}_n^{[m-1,\alpha]}(x+y,b,c;\lambda)\\
&=\displaystyle\frac{1}{2}\sum\limits_{k=0}^{n}
\frac{(\log c)^{k}}{k+1}\left[\binom{n}{k}\mathcal{G}_{n-k}^{[m-1,\alpha]}(y,b,c;\lambda)+\sum\limits_{j=k}^{n}\binom{n}{j}\binom{j}{k}\mathcal{G}_{n-j}^{[m-1,\alpha]}(y,b,c;\lambda)(\log c)^{j-k}\right]G_{k+1}(x).
\end{align}
}
\end{coro}

\vspace{0.25cm}

\begin{obs}
Si hacemos $b=c=e$ en el \textit{Corolario \ref{corgen}} tenemos
{\footnotesize
\begin{align}\label{eq117}
& \nonumber\mathfrak{G}_n^{[m-1,\alpha]}(x+y;\lambda)\\
&=\displaystyle\frac{1}{2}\sum\limits_{k=0}^{n}
\frac{1}{k+1}\left[\binom{n}{k}\mathfrak{G}_{n-k}^{[m-1,\alpha]}(y;\lambda)+\sum\limits_{j=k}^{n}\binom{n}{j}\binom{j}{k}\mathfrak{G}_{n-j}^{[m-1,\alpha]}(y;\lambda)\right]G_{k+1}(x).
\end{align}
}
\end{obs}

\vspace{0.25cm}

\begin{teo}\label{teojac}
La relaci�n
{\footnotesize
\begin{align}
& \nonumber \mathcal{Q}_{n}^{[m-1,\mu]}(x+y,b,c;\lambda;u,v)\\
&\label{eq17} =\displaystyle\sum\limits_{k=0}^{n}(-1)^k
\sum\limits_{j=k}^{n}j!(\log c)^{j}\binom{j+\alpha}{j-k}\binom{n}{j}\frac{\alpha+\beta+2k+1}{(\alpha+\beta+k+1)_{j+1}}    \mathcal{Q}_{n-j}^{[m-1,\mu]}(y,b,c;\lambda;u,v)P_k^{(\alpha,\beta)}(1-2x)
\end{align}
}
se mantiene entre los polinomios $\mathcal{Q}_{n}^{[m-1,\mu]}(x,b,c;\lambda;u,v)$ y los polinomios de Jacobi $P_n^{(\alpha,\beta)}(x)$ definidos por (\ref{eq30}).
\end{teo}

\begin{proof}
{\footnotesize Por sustituci�n de (\ref{eq31}) en el lado derecho de (\ref{eq9}), tenemos
\begin{align*}
& \mathcal{Q}_{n}^{[m-1,\mu]}(x+y,b,c;\lambda;u,v)\\
& =\displaystyle\sum
\limits_{j=0}^{n}
\binom{n}{j}\mathcal{Q}_{j}^{[m-1,\mu]}(y,b,c;\lambda;u,v)(n-j)!(\log c)^{n-j}\displaystyle\sum\limits_{k=0}^{n-j}(-1)^k\binom{n-j+\alpha}{n-j-k}
\frac{\alpha+\beta+2k+1}{(\alpha+\beta+k+1)_{n-j+1}}P_k^{(\alpha,\beta)}(1-2x)\\
\intertext{aplicando (\ref{s2}), tenemos}
& =\displaystyle\sum
\limits_{j=0}^{n}\sum\limits_{k=0}^{n-j}
\binom{n}{j}\mathcal{Q}_{j}^{[m-1,\mu]}(y,b,c;\lambda;u,v)(n-j)!(\log c)^{n-j}(-1)^k\binom{n-j+\alpha}{n-j-k}
\frac{\alpha+\beta+2k+1}{(\alpha+\beta+k+1)_{n-j+1}}P_k^{(\alpha,\beta)}(1-2x)\\
\intertext{usando (\ref{s1}) se sigue que}
& =\displaystyle\sum
\limits_{k=0}^{n}\sum\limits_{j=0}^{n-k}
\binom{n}{j}\mathcal{Q}_{j}^{[m-1,\mu]}(y,b,c;\lambda;u,v)(n-j)!(\log c)^{n-j}(-1)^k\binom{n-j+\alpha}{n-j-k}
\frac{\alpha+\beta+2k+1}{(\alpha+\beta+k+1)_{n-j+1}}P_k^{(\alpha,\beta)}(1-2x)\\
\intertext{por uso de (\ref{s2}) nuevamente, obtenemos}
& =\displaystyle\sum
\limits_{k=0}^{n}(-1)^k\sum\limits_{j=0}^{n-k}
\binom{n}{j}\binom{n-j+\alpha}{n-j-k}\mathcal{Q}_{j}^{[m-1,\mu]}(y,b,c;\lambda;u,v)(n-j)!
(\log c)^{n-j}\frac{\alpha+\beta+2k+1}{(\alpha+\beta+k+1)_{n-j+1}}P_k^{(\alpha,\beta)}(1-2x)\\
\intertext{aplicando (\ref{s3}) se sigue que}
& =\displaystyle\sum
\limits_{k=0}^{n}(-1)^k\sum\limits_{j=0}^{n-k}
\binom{n}{n-j}\binom{j+\alpha}{j-k}\mathcal{Q}_{j}^{[m-1,\mu]}(y,b,c;\lambda;u,v)j!
(\log c)^{j}\frac{\alpha+\beta+2k+1}{(\alpha+\beta+k+1)_{n-j+1}}P_k^{(\alpha,\beta)}(1-2x)\\
\intertext{finalmenete, por (\ref{comb}) y ordenando los combinatorios, esto completa la prueba}
& =\displaystyle\sum\limits_{k=0}^{n}(-1)^k
\sum\limits_{j=k}^{n}j!(\log c)^{j}\binom{j+\alpha}{j-k}\binom{n}{j}\frac{\alpha+\beta+2k+1}{(\alpha+\beta+k+1)_{j+1}}    \mathcal{Q}_{n-j}^{[m-1,\mu]}(y,b,c;\lambda;u,v)P_k^{(\alpha,\beta)}(1-2x).
\end{align*}
}
\end{proof}

\vspace{0.25cm}

Al establecer $\lambda=:-\lambda, u=0$ y $v=1$ en el \textit{Teorema \ref{teojac}}, y luego multiplicar por $(-1)^\mu$ tenemos el siguiente corolario.

\vspace{0.25cm}

\begin{coro}\label{corber1}
{\footnotesize
\begin{align}
& \nonumber \mathcal{B}_{n}^{[m-1,\mu]}(x+y,b,c;\lambda)\\
&\nonumber =\displaystyle\sum\limits_{k=0}^{n}(-1)^k
\sum\limits_{j=k}^{n}j!(\log c)^{j}\binom{j+\alpha}{j-k}\binom{n}{j}\frac{\alpha+\beta+2k+1}{(\alpha+\beta+k+1)_{j+1}}    \mathcal{B}_{n-j}^{[m-1,\mu]}(y,b,c;\lambda)P_k^{(\alpha,\beta)}(1-2x)
\end{align}
}
\end{coro}

\vspace{0.25cm}

\begin{obs}
Si hacemos $b=c=e$ en el \textit{Corolario \ref{corber1}} tenemos
{\footnotesize
\begin{align}
& \nonumber \mathfrak{B}_{n}^{[m-1,\mu]}(x+y;\lambda)\\
&\nonumber =\displaystyle\sum\limits_{k=0}^{n}(-1)^k
\sum\limits_{j=k}^{n}j!\binom{j+\alpha}{j-k}\binom{n}{j}\frac{\alpha+\beta+2k+1}{(\alpha+\beta+k+1)_{j+1}}    \mathfrak{B}_{n-j}^{[m-1,\mu]}(y;\lambda)P_k^{(\alpha,\beta)}(1-2x)
\end{align}
}
\end{obs}

\vspace{0.25cm}

Al establecer $u=1$ y $v=0$ en el \textit{Teorema \ref{teojac}}, tenemos el siguiente corolario.

\begin{coro}\label{coreul1}
{\footnotesize
\begin{align}
& \nonumber \mathcal{E}_{n}^{[m-1,\mu]}(x+y,b,c;\lambda)\\
&\nonumber =\displaystyle\sum\limits_{k=0}^{n}(-1)^k
\sum\limits_{j=k}^{n}j!(\log c)^{j}\binom{j+\alpha}{j-k}\binom{n}{j}\frac{\alpha+\beta+2k+1}{(\alpha+\beta+k+1)_{j+1}}    \mathcal{E}_{n-j}^{[m-1,\mu]}(y,b,c;\lambda)P_k^{(\alpha,\beta)}(1-2x)
\end{align}
}
\end{coro}

\vspace{0.25cm}

\begin{obs}
Si hacemos $b=c=e$ en el \textit{Corolario \ref{coreul1}} tenemos
{\footnotesize
\begin{align}
& \nonumber \mathfrak{E}_{n}^{[m-1,\mu]}(x+y;\lambda)\\
&\nonumber =\displaystyle\sum\limits_{k=0}^{n}(-1)^k
\sum\limits_{j=k}^{n}j!\binom{j+\alpha}{j-k}\binom{n}{j}\frac{\alpha+\beta+2k+1}{(\alpha+\beta+k+1)_{j+1}}    \mathfrak{E}_{n-j}^{[m-1,\mu]}(y;\lambda)P_k^{(\alpha,\beta)}(1-2x)
\end{align}
} la cual es justamente la Ec.(4.7) de \cite{Ming}.
\end{obs}

\vspace{0.25cm}

Al tomar $u=1$ y $v=1$ en el \textit{Teorema \ref{teojac}}, tenemos el siguiente corolario.

\vspace{0.25cm}

\begin{coro}\label{corgen1}
{\footnotesize
\begin{align}
& \nonumber \mathcal{G}_{n}^{[m-1,\mu]}(x+y,b,c;\lambda)\\
&\nonumber =\displaystyle\sum\limits_{k=0}^{n}(-1)^k
\sum\limits_{j=k}^{n}j!(\log c)^{j}\binom{j+\alpha}{j-k}\binom{n}{j}\frac{\alpha+\beta+2k+1}{(\alpha+\beta+k+1)_{j+1}}    \mathcal{G}_{n-j}^{[m-1,\mu]}(y,b,c;\lambda)P_k^{(\alpha,\beta)}(1-2x)
\end{align}
}
\end{coro}

\vspace{0.25cm}

\begin{obs}
Si hacemos $b=c=e$ en el \textit{Corolario \ref{corgen1}} tenemos
{\footnotesize
\begin{align}
& \nonumber \mathfrak{G}_{n}^{[m-1,\mu]}(x+y;\lambda)\\
&\nonumber =\displaystyle\sum\limits_{k=0}^{n}(-1)^k
\sum\limits_{j=k}^{n}j!\binom{j+\alpha}{j-k}\binom{n}{j}\frac{\alpha+\beta+2k+1}{(\alpha+\beta+k+1)_{j+1}}    \mathfrak{G}_{n-j}^{[m-1,\mu]}(y;\lambda)P_k^{(\alpha,\beta)}(1-2x)
\end{align}
}
\end{obs}

\vspace{0.25cm}

\begin{teo}\label{bergen}
La relaci�n
\begin{equation}\label{eq94}
\mathcal{Q}_{n}^{[m-1,\mu]}(x+y,b,c;\lambda;u,v)=\displaystyle\sum\limits_{k=0}^{n}\sum\limits_{j=k}^{n}\frac{k!(\log c)^{j}}{(k+m)!}\binom{n}{j}\binom{j}{k}\mathcal{Q}_{n-j}^{[m-1,\mu]}(y,b,c;\lambda;u,v)B_{j-k}^{[m-1]}(x)
\end{equation}
se mantiene entre los polinomios $\mathcal{Q}_{n}^{[m-1,\mu]}(x,b,c;\lambda;u,v)$ y los polinomios de Bernoulli generalizados definidos por (\ref{eq5}) (con $\alpha=1$).
\end{teo}

\begin{proof}
Por sustituci�n de (\ref{eq60}) en el lado derecho de (\ref{eq9}), tenemos
\begin{align*}
& \mathcal{Q}_{n}^{[m-1,\mu]}(x+y,b,c;\lambda;u,v)\\
&=\displaystyle\sum \limits_{j=0}^{n} \binom{n}{j}\mathcal{Q}_{j}^{[m-1,\mu]}(y,b,c;\lambda;u,v)(\log c)^{n-j}\sum\limits_{k=0}^{n-j}\binom{n-j}{k}\frac{k!}{(k+m)!}B_{n-j-k}^{[m-1]}(x).\\
\intertext{por (\ref{s2}), tenemos}
& =\displaystyle\sum \limits_{j=0}^{n}\sum\limits_{k=0}^{n-j} \binom{n}{j}\mathcal{Q}_{j}^{[m-1,\mu]}(y,b,c;\lambda;u,v)(\log c)^{n-j}\binom{n-j}{k}\frac{k!}{(k+m)!}B_{n-j-k}^{[m-1]}(x).\\
\intertext{aplicando (\ref{s1}) se sigue que}
& =\displaystyle\sum \limits_{k=0}^{n}\sum\limits_{j=0}^{n-k} \binom{n}{j}\mathcal{Q}_{j}^{[m-1,\mu]}(y,b,c;\lambda;u,v)(\log c)^{n-j}\binom{n-j}{k}\frac{k!}{(k+m)!}B_{n-j-k}^{[m-1]}(x).\\
\intertext{aplicando (\ref{s3}), obtenemos}
& =\displaystyle\sum \limits_{k=0}^{n}\sum\limits_{j=k}^{n} \binom{n}{n-j}\mathcal{Q}_{n-j}^{[m-1,\mu]}(y,b,c;\lambda;u,v)(\log c)^{j}\binom{j}{k}\frac{k!}{(k+m)!}B_{j-k}^{[m-1]}(x).\\
\intertext{Finalmente, usando (\ref{comb}) la prueba se completa}
& =\displaystyle\sum\limits_{k=0}^{n}\sum\limits_{j=k}^{n}\frac{k!(\log c)^{j}}{(k+m)!}\binom{n}{j}\binom{j}{k}\mathcal{Q}_{n-j}^{[m-1,\mu]}(y,b,c;\lambda;u,v)B_{j-k}^{[m-1]}(x)
\end{align*}
\end{proof}

\vspace{0.25cm}

Estableciendo $\lambda=:-\lambda, u=0$ y $v=1$ en el \textit{Teorema \ref{bergen}}, y luego multiplicando por $(-1)^\mu$ tenemos el siguiente corolario.

\vspace{0.25cm}

\begin{coro}\label{corber2}
\begin{equation}\label{eq934}
\mathcal{B}_{n}^{[m-1,\mu]}(x+y,b,c;\lambda)=\displaystyle\sum\limits_{k=0}^{n}\sum\limits_{j=k}^{n}\frac{k!(\log c)^{j}}{(k+m)!}\binom{n}{j}\binom{j}{k}\mathcal{B}_{n-j}^{[m-1,\mu]}(y,b,c;\lambda)B_{j-k}^{[m-1]}(x)
\end{equation}
\end{coro}

\vspace{0.25cm}

\begin{obs}
Si hacemos $b=c=e$ en el \textit{Corolario \ref{corber2}} tenemos
\begin{equation}
\mathfrak{B}_{n}^{[m-1,\mu]}(x+y;\lambda)=\displaystyle\sum\limits_{k=0}^{n}\sum\limits_{j=k}^{n}\frac{k!}{(k+m)!}\binom{n}{j}\binom{j}{k}\mathfrak{B}_{n-j}^{[m-1,\mu]}(y;\lambda)B_{j-k}^{[m-1]}(x)
\end{equation}
\end{obs}

\vspace{0.25cm}

Estableciendo $u=1$ y $v=0$ en el \textit{Teorema \ref{bergen}}, tenemos el siguiente corolario.

\vspace{0.25cm}

\begin{coro}\label{coreul2}
\begin{equation}
\mathcal{E}_{n}^{[m-1,\mu]}(x+y,b,c;\lambda)=\displaystyle\sum\limits_{k=0}^{n}\sum\limits_{j=k}^{n}\frac{k!(\log c)^{j}}{(k+m)!}\binom{n}{j}\binom{j}{k}\mathcal{E}_{n-j}^{[m-1,\mu]}(y,b,c;\lambda)B_{j-k}^{[m-1]}(x)
\end{equation}
\end{coro}

\vspace{0.25cm}

\begin{obs}
Si hacemos $b=c=e$ en el \textit{Corolario \ref{coreul2}} tenemos
\begin{equation}
\mathfrak{E}_{n}^{[m-1,\mu]}(x+y;\lambda)=\displaystyle\sum\limits_{k=0}^{n}\sum\limits_{j=k}^{n}\frac{k!}{(k+m)!}\binom{n}{j}\binom{j}{k}\mathfrak{E}_{n-j}^{[m-1,\mu]}(y;\lambda)B_{j-k}^{[m-1]}(x)
\end{equation}
la cual es la Ec.(4.11) de \cite{Ming}.
\end{obs}

\vspace{0.25cm}

Establceciendo $u=1$ y $v=1$ en el \textit{Teorema \ref{bergen}}, tenemos el siguiente corolario.

\vspace{0.25cm}

\begin{coro}\label{corgen2}
\begin{equation}
\mathcal{G}_{n}^{[m-1,\mu]}(x+y,b,c;\lambda)=\displaystyle\sum\limits_{k=0}^{n}\sum\limits_{j=k}^{n}\frac{k!(\log c)^{j}}{(k+m)!}\binom{n}{j}\binom{j}{k}\mathcal{G}_{n-j}^{[m-1,\mu]}(y,b,c;\lambda)B_{j-k}^{[m-1]}(x)
\end{equation}
\end{coro}

\vspace{0.25cm}

\begin{obs}
Si hacemos $b=c=e$ en el \textit{Corolario \ref{corgen2}} tenemos
\begin{equation}
\mathfrak{G}_{n}^{[m-1,\mu]}(x+y;\lambda)=\displaystyle\sum\limits_{k=0}^{n}\sum\limits_{j=k}^{n}\frac{k!}{(k+m)!}\binom{n}{j}\binom{j}{k}\mathfrak{G}_{n-j}^{[m-1,\mu]}(y;\lambda)B_{j-k}^{[m-1]}(x)
\end{equation}
\end{obs}

\vspace{0.25cm}

\begin{teo}\label{teostir}
La relaci�n
\begin{equation}\label{eq14}
\mathcal{Q}_{n}^{[m-1,\mu]}(x+y,b,c;\lambda;u,v)=\displaystyle\sum\limits_{k=0}^{n}k!
\binom{x}{k}\sum\limits_{j=k}^{n}\binom{n}{j}\mathcal{Q}_{n-j}^{[m-1,\mu]}(y,b,c;\lambda;u,v)(\log c)^{j}\mathfrak{S}(j,k)
\end{equation}
se mantiene entre los polinomios $\mathcal{Q}_{n}^{[m-1,\mu]}(x,b,c;\lambda;u,v)$ y los n�meros de Stirling de segunda clase $\mathfrak{S}(n,k)$ definidos en (\ref{eq15}).
\end{teo}

\begin{proof}
Por sustituci�n de (\ref{eq15}) en el lado derecho de (\ref{eq9}), tenemos
\begin{align*}
& \mathcal{Q}_{n}^{[m-1,\mu]}(x+y,b,c;\lambda;u,v)=\displaystyle\sum \limits_{j=0}^{n} \binom{n}{j}\mathcal{Q}_{j}^{[m-1,\mu]}(y,b,c;\lambda;u,v)(\log c)^{n-j}\sum\limits_{k=0}^{n-j}\binom{x}{k}k!\mathfrak{S}(n-j,k)\\
\intertext{por (\ref{s2}) se sigue que}
& =\displaystyle\sum \limits_{j=0}^{n}\sum\limits_{k=0}^{n-j} \binom{n}{j}\mathcal{Q}_{j}^{[m-1,\mu]}(y,b,c;\lambda;u,v)(\log c)^{n-j}\binom{x}{k}k!\mathfrak{S}(n-j,k)\\
\intertext{ahora, por (\ref{s1}) entonces}
& =\displaystyle\sum \limits_{k=0}^{n}\sum\limits_{j=0}^{n-k} \binom{n}{j}\mathcal{Q}_{j}^{[m-1,\mu]}(y,b,c;\lambda;u,v)(\log c)^{n-j}\binom{x}{k}k!\mathfrak{S}(n-j,k)\\
\intertext{usando (\ref{s2}) nuevamente, tenemos}
& =\displaystyle\sum \limits_{k=0}^{n}\binom{x}{k}k!\sum\limits_{j=0}^{n-k} \binom{n}{j}\mathcal{Q}_{j}^{[m-1,\mu]}(y,b,c;\lambda;u,v)(\log c)^{n-j}\mathfrak{S}(n-j,k)\\
\intertext{usando (\ref{s3}) , obtenemos}
& =\displaystyle\sum \limits_{k=0}^{n}\binom{x}{k}k!\sum\limits_{j=k}^{n} \binom{n}{n-j}\mathcal{Q}_{n-j}^{[m-1,\mu]}(y,b,c;\lambda;u,v)(\log c)^{j}\mathfrak{S}(j,k).\\
\intertext{Finalmente, por (\ref{comb}) se completa la prueba}
& =\displaystyle\sum\limits_{k=0}^{n}k!
\binom{x}{k}\sum\limits_{j=k}^{n}\binom{n}{j}\mathcal{Q}_{n-j}^{[m-1,\mu]}(y,b,c;\lambda;u,v)(\log c)^{j}\mathfrak{S}(j,k).
\end{align*}
\end{proof}

\vspace{0.25cm}

Estableciendo $\lambda=:-\lambda, u=0$ y $v=1$ en el \textit{Teorema \ref{teostir}}, y luego multiplicando por $(-1)^\mu$ tenemos el siguiente corolario.

\vspace{0.25cm}

\begin{coro}\label{corst2}
\begin{equation}\label{st934}
\mathcal{B}_{n}^{[m-1,\mu]}(x+y,b,c;\lambda)=\displaystyle\sum\limits_{k=0}^{n}k!
\binom{x}{k}\sum\limits_{j=k}^{n}\binom{n}{j}\mathcal{B}_{n-j}^{[m-1,\mu]}(y,b,c;\lambda)(\log c)^{j}\mathfrak{S}(j,k)
\end{equation}
\end{coro}

\vspace{0.25cm}

\begin{obs}
Si hacemos $b=c=e$ en el \textit{Corolario \ref{corst2}} tenemos
\begin{equation}
\mathfrak{B}_{n}^{[m-1,\mu]}(x+y;\lambda)=\displaystyle\sum\limits_{k=0}^{n}k!
\binom{x}{k}\sum\limits_{j=k}^{n}\binom{n}{j}\mathfrak{B}_{n-j}^{[m-1,\mu]}(y;\lambda)\mathfrak{S}(j,k)
\end{equation}
\end{obs}

\vspace{0.25cm}

Al establecer $u=1$ y $v=0$ en el \textit{Teorema \ref{teostir}}, tenemos el siguiente corolario.

\vspace{0.25cm}

\begin{coro}\label{corstl2}
\begin{equation}
\mathcal{E}_{n}^{[m-1,\mu]}(x+y,b,c;\lambda)=\displaystyle\sum\limits_{k=0}^{n}k!
\binom{x}{k}\sum\limits_{j=k}^{n}\binom{n}{j}\mathcal{E}_{n-j}^{[m-1,\mu]}(y,b,c;\lambda)(\log c)^{j}\mathfrak{S}(j,k)
\end{equation}
\end{coro}

\vspace{0.25cm}

\begin{obs}
Si hacemos $b=c=e$ en el \textit{Corolario \ref{corstl2}} tenemos
\begin{equation}
\mathfrak{E}_{n}^{[m-1,\mu]}(x+y;\lambda)=\displaystyle\sum\limits_{k=0}^{n}k!
\binom{x}{k}\sum\limits_{j=k}^{n}\binom{n}{j}\mathfrak{E}_{n-j}^{[m-1,\mu]}(y;\lambda)\mathfrak{S}(j,k)
\end{equation}
la cual es justamente la Ec.(4.3) de \cite{Ming}.
\end{obs}

\vspace{0.25cm}

Haciendo $u=1$ y $v=1$ en el \textit{Teorema \ref{teostir}}, tenemos el siguiente corolario.

\vspace{0.25cm}

\begin{coro}\label{corgst2}
\begin{equation}
\mathcal{G}_{n}^{[m-1,\mu]}(x+y,b,c;\lambda)=\displaystyle\sum\limits_{k=0}^{n}k!
\binom{x}{k}\sum\limits_{j=k}^{n}\binom{n}{j}\mathcal{G}_{n-j}^{[m-1,\mu]}(y,b,c;\lambda)(\log c)^{j}\mathfrak{S}(j,k)
\end{equation}
\end{coro}

\vspace{0.25cm}

\begin{obs}
Si hacemos $b=c=e$ en el \textit{Corolario \ref{corgst2}} tenemos
\begin{equation}
\mathfrak{G}_{n}^{[m-1,\mu]}(x+y;\lambda)=\displaystyle\sum\limits_{k=0}^{n}k!
\binom{x}{k}\sum\limits_{j=k}^{n}\binom{n}{j}\mathfrak{G}_{n-j}^{[m-1,\mu]}(y;\lambda)\mathfrak{S}(j,k)
\end{equation}
\end{obs}

\vspace{0.25cm}

\begin{teo}\label{teolag}
La relaci�n
\begin{align}
& \nonumber \mathcal{Q}_{n}^{[m-1,\mu]}(x+y,b,c;\lambda;u,v)\\
& \label{eq20} =\displaystyle
\sum\limits_{k=0}^{n}(-1)^k
\sum\limits_{j=k}^{n}j!\binom{n}{j}\binom{j+\alpha}{j-k}(\log c)^j\mathcal{Q}_{n-j}^{[m-1,\mu]}(y,b,c;\lambda;u,v)L_k^{(\alpha)}(x).
\end{align}
se mantiene entre los polinomios $\mathcal{Q}_{n}^{[m-1,\mu]}(x,b,c;\lambda;u,v)$ y los polinomios de Laguerre $L_n^{(\alpha)}(x)$ definidos por
(\ref{lag}).
\end{teo}

\vspace{0.25cm}

\begin{proof}
Si hacemos $m=n-j$ en (\ref{lg1}), tenemos:
\begin{equation}\label{eq22}
x^{n-j}=(n-j)!\displaystyle\sum\limits_{k=0}^{n-j}(-1)^k
\binom{n-j+\alpha}{n-j-k}L_k^{(\alpha)}(x).
\end{equation}
Ahora, al sustituir (\ref{eq22}) en el miembro de la derecha de (\ref{eq9}) se sigue que:
\begin{align*}
& \mathcal{Q}_{n}^{[m-1,\mu]}(x+y,b,c;\lambda;u,v)\\
& =\displaystyle\sum
\limits_{j=0}^{n}
\binom{n}{j}\mathcal{Q}_{j}^{[m-1,\mu]}(y,b,c;\lambda;u,v)(\log c)^{n-j}(n-j)!\displaystyle\sum\limits_{k=0}^{n-j}(-1)^k
\binom{n-j+\alpha}{n-j-k}L_k^{\alpha}(x)\\
\intertext{por aplicaci\'on de (\ref{s2}), obtenemos}
& =\displaystyle\sum
\limits_{j=0}^{n}\sum\limits_{k=0}^{n-j}
\binom{n}{j}\mathcal{Q}_{j}^{[m-1,\mu]}(y,b,c;\lambda;u,v)(\log c)^{n-j}(n-j)!(-1)^k
\binom{n-j+\alpha}{n-j-k}L_k^{\alpha}(x)\\
\intertext{usando (\ref{s1}), tenemos}
& =\displaystyle\sum
\limits_{k=0}^{n}\sum\limits_{j=0}^{n-k}
\binom{n}{j}\mathcal{Q}_{j}^{[m-1,\mu]}(y,b,c;\lambda;u,v)(\log c)^{n-j}(n-j)!(-1)^k
\binom{n-j+\alpha}{n-j-k}L_k^{\alpha}(x)
\end{align*}
Finalmente al aplicar nuevamente (\ref{s2}) y luego (\ref{s3}) concluimos la prueba.
\end{proof}

\vspace{0.25cm}
Haciendo $x:=k\alpha$, $a=:x$ en (\ref{ch3}) y sustituyendo $L_k^{(\alpha)}(x)$ en el \textit{Teorema \ref{teolag}} tenemos el siguiente corolario.

\vspace{0.25cm}
\begin{coro}
La relaci�n
\begin{align}
& \nonumber \mathcal{Q}_{n}^{[m-1,\mu]}(x+y,b,c;\lambda;u,v)\\
& \label{eq201} =\displaystyle
\sum\limits_{k=0}^{n}\frac{(-1)^k}{k!}
\sum\limits_{j=k}^{n}j!\binom{n}{j}\binom{j+\alpha}{j-k}(\log c)^j\mathcal{Q}_{n-j}^{[m-1,\mu]}(y,b,c;\lambda;u,v)C_k^{(x)}(k+\alpha).
\end{align}
se mantiene entre los polinomios $\mathcal{Q}_{n}^{[m-1,\mu]}(x,b,c;\lambda;u,v)$ y los polinomios de Charlier $C_n^{(a)}(x)$ definidos por
(\ref{chr1}).
\end{coro}


\vspace{0.25cm}

Por otro lado estableciendo $\lambda=:-\lambda, u=0$ y $v=1$ en el \textit{Teorema \ref{teolag}}, y luego multiplicando por $(-1)^\mu$ tenemos el siguiente corolario.

\vspace{0.25cm}

\begin{coro}\label{corst2l}
\begin{align}
& \nonumber \mathcal{B}_{n}^{[m-1,\mu]}(x+y,b,c;\lambda)\\
& \label{st934l} =\displaystyle
\sum\limits_{k=0}^{n}(-1)^k
\sum\limits_{j=k}^{n}j!\binom{n}{j}\binom{j+\alpha}{j-k}(\log c)^j\mathcal{B}_{n-j}^{[m-1,\mu]}(y,b,c;\lambda)L_k^{(\alpha)}(x).
\end{align}
\end{coro}

\vspace{0.25cm}

\begin{obs}
Si hacemos $b=c=e$ en el \textit{Corolario \ref{corst2l}} tenemos
\begin{align}
& \nonumber \mathfrak{B}_{n}^{[m-1,\mu]}(x+y;\lambda)\\
& \label{uil90} =\displaystyle
\sum\limits_{k=0}^{n}(-1)^k
\sum\limits_{j=k}^{n}j!\binom{n}{j}\binom{j+\alpha}{j-k}\mathfrak{B}_{n-j}^{[m-1,\mu]}(y;\lambda)L_k^{(\alpha)}(x).
\end{align}
\end{obs}

\vspace{0.25cm}

Al establecer $u=1$ y $v=0$ en el \textit{Teorema \ref{teolag}}, tenemos el siguiente corolario.

\vspace{0.25cm}

\begin{coro}
\begin{align}
& \nonumber \mathcal{E}_{n}^{[m-1,\mu]}(x+y,b,c;\lambda)\\
& \label{corstl2l} =\displaystyle
\sum\limits_{k=0}^{n}(-1)^k
\sum\limits_{j=k}^{n}j!\binom{n}{j}\binom{j+\alpha}{j-k}(\log c)^j\mathcal{E}_{n-j}^{[m-1,\mu]}(y,b,c;\lambda)L_k^{(\alpha)}(x).
\end{align}
\end{coro}

\vspace{0.25cm}

\begin{obs}
Si hacemos $b=c=e$ en el \textit{Corolario \ref{corstl2l}} tenemos
\begin{align}
& \nonumber \mathfrak{E}_{n}^{[m-1,\mu]}(x+y;\lambda)\\
& \label{uil901} =\displaystyle
\sum\limits_{k=0}^{n}(-1)^k
\sum\limits_{j=k}^{n}j!\binom{n}{j}\binom{j+\alpha}{j-k}\mathfrak{E}_{n-j}^{[m-1,\mu]}(y;\lambda)L_k^{(\alpha)}(x).
\end{align}
la cual es justamente la Ec.(4.5) de \cite{Ming}.
\end{obs}

\vspace{0.25cm}

Haciendo $u=1$ y $v=1$ en el \textit{Teorema \ref{teolag}}, tenemos el siguiente corolario.

\vspace{0.25cm}

\begin{coro}
\begin{align}
& \nonumber \mathcal{G}_{n}^{[m-1,\mu]}(x+y,b,c;\lambda)\\
& \label{corgst2l} =\displaystyle
\sum\limits_{k=0}^{n}(-1)^k
\sum\limits_{j=k}^{n}j!\binom{n}{j}\binom{j+\alpha}{j-k}(\log c)^j\mathcal{G}_{n-j}^{[m-1,\mu]}(y,b,c;\lambda)L_k^{(\alpha)}(x).
\end{align}
\end{coro}

\vspace{0.25cm}

\begin{obs}
Si hacemos $b=c=e$ en el \textit{Corolario \ref{corgst2l}} tenemos
\begin{align}
& \nonumber \mathfrak{G}_{n}^{[m-1,\mu]}(x+y;\lambda)\\
& \label{uil902} =\displaystyle
\sum\limits_{k=0}^{n}(-1)^k
\sum\limits_{j=k}^{n}j!\binom{n}{j}\binom{j+\alpha}{j-k}\mathfrak{G}_{n-j}^{[m-1,\mu]}(y;\lambda)L_k^{(\alpha)}(x).
\end{align}
\end{obs}

\vspace{0.25cm}

\begin{teo}\label{teoherm}
La relaci�n
\begin{align}
& \nonumber \mathcal{Q}_{n}^{[m-1,\mu]}(x+y,b,c;\lambda;u,v)
\\
& \label{eq23} =\displaystyle\sum\limits_{j=0}^{n}\sum\limits_{k=0}^{[(n-j)/2]}2^{-(n-j)}\binom{n}{j}\binom{n-j}{2k}\displaystyle\frac{(2k)!}{k!}(\log c)^{n-j}\mathcal{Q}_{j}^{[m-1,\mu]}(y,b,c;\lambda;u,v)H_{n-j-2k}(x).
\end{align}
se mantiene entre los polinomios $\mathcal{Q}_{n}^{[m-1,\mu]}(x,b,c;\lambda;u,v)$ y los polinomios de Hermite $H_n(x)$ definidos por
(\ref{her}).
\end{teo}

\vspace{0.25cm}

\begin{proof}
Consideramos la expresi\'on (\ref{her1}), despejamos $x^m$  y hacemos $m=n-j$ tenemos:
\begin{equation}\label{eq25}
x^{n-j}=\displaystyle\sum\limits_{k=0}^{[n-j/2]}2^{-(n-j)}\binom{n-j}{2k}\frac{(2k)!}{k!}H_{n-j-2k}(x),
\end{equation}
finalmente al sustituir (\ref{eq25}) en el miembro de la derecha de (\ref{eq9}) y haciendo uso de (\ref{s2}) finalizamos la prueba.
\end{proof}

\vspace{0.25cm}

Estableciendo $\lambda=:-\lambda, u=0$ y $v=1$ en el \textit{Teorema \ref{teoherm}}, y luego multiplicando por $(-1)^\mu$ tenemos el siguiente corolario.

\vspace{0.25cm}

\begin{coro}\label{huy3}
\begin{align}
& \nonumber \mathcal{B}_{n}^{[m-1,\mu]}(x+y,b,c;\lambda)
\\
& \label{st934h} =\displaystyle\sum\limits_{j=0}^{n}\sum\limits_{k=0}^{[(n-j)/2]}2^{-(n-j)}\binom{n}{j}\binom{n-j}{2k}\displaystyle\frac{(2k)!}{k!}(\log c)^{n-j}\mathcal{B}_{j}^{[m-1,\mu]}(y,b,c;\lambda)H_{n-j-2k}(x).
\end{align}
\end{coro}

\vspace{0.25cm}

\begin{obs}
Si hacemos $b=c=e$ en el \textit{Corolario \ref{huy3}} tenemos
\begin{align}
& \nonumber \mathfrak{B}_{n}^{[m-1,\mu]}(x+y;\lambda)
\\
& \label{st934h1} =\displaystyle\sum\limits_{j=0}^{n}\sum\limits_{k=0}^{[(n-j)/2]}2^{-(n-j)}\binom{n}{j}\binom{n-j}{2k}\displaystyle\frac{(2k)!}{k!}\mathfrak{B}_{j}^{[m-1,\mu]}(y;\lambda)H_{n-j-2k}(x).
\end{align}
\end{obs}

\vspace{0.25cm}

Al establecer $u=1$ y $v=0$ en el \textit{Teorema \ref{teoherm}}, tenemos el siguiente corolario.

\vspace{0.25cm}

\begin{coro}\label{huy2}
\begin{align}
& \nonumber \mathcal{E}_{n}^{[m-1,\mu]}(x+y,b,c;\lambda)
\\
& \label{st934h2} =\displaystyle\sum\limits_{j=0}^{n}\sum\limits_{k=0}^{[(n-j)/2]}2^{-(n-j)}\binom{n}{j}\binom{n-j}{2k}\displaystyle\frac{(2k)!}{k!}(\log c)^{n-j}\mathcal{E}_{j}^{[m-1,\mu]}(y,b,c;\lambda)H_{n-j-2k}(x).
\end{align}
el cual es justamente el \textit{Teorema 4.3} en \cite{Tremblay2}.
\end{coro}

\vspace{0.25cm}

\begin{obs}
Si hacemos $b=c=e$ en el \textit{Corolario \ref{huy2}} tenemos
\begin{align}
& \nonumber \mathfrak{E}_{n}^{[m-1,\mu]}(x+y;\lambda)
\\
& \label{st934h3} =\displaystyle\sum\limits_{j=0}^{n}\sum\limits_{k=0}^{[(n-j)/2]}2^{-(n-j)}\binom{n}{j}\binom{n-j}{2k}\displaystyle\frac{(2k)!}{k!}\mathfrak{E}_{j}^{[m-1,\mu]}(y;\lambda)H_{n-j-2k}(x).
\end{align}
\end{obs}

\vspace{0.25cm}

Haciendo $u=1$ y $v=1$ en el \textit{Teorema \ref{teoherm}}, tenemos el siguiente corolario.

\vspace{0.25cm}

\begin{coro}\label{huy}
\begin{align}
& \nonumber \mathcal{G}_{n}^{[m-1,\mu]}(x+y,b,c;\lambda)
\\
& \label{st934h4} =\displaystyle\sum\limits_{j=0}^{n}\sum\limits_{k=0}^{[(n-j)/2]}2^{-(n-j)}\binom{n}{j}\binom{n-j}{2k}\displaystyle\frac{(2k)!}{k!}(\log c)^{n-j}\mathcal{G}_{j}^{[m-1,\mu]}(y,b,c;\lambda)H_{n-j-2k}(x).
\end{align}
\end{coro}

\vspace{0.25cm}

\begin{obs}
Si hacemos $b=c=e$ en el \textit{Corolario \ref{huy}} tenemos
\begin{align}
& \nonumber \mathfrak{G}_{n}^{[m-1,\mu]}(x+y;\lambda)
\\
& \label{st934h5} =\displaystyle\sum\limits_{j=0}^{n}\sum\limits_{k=0}^{[(n-j)/2]}2^{-(n-j)}\binom{n}{j}\binom{n-j}{2k}\displaystyle\frac{(2k)!}{k!}\mathfrak{G}_{j}^{[m-1,\mu]}(y;\lambda)H_{n-j-2k}(x).
\end{align}
\end{obs}

\vspace{0.25cm}

\begin{teo}\label{teobess}
La relaci�n
\begin{align}
& \nonumber \mathcal{Q}_{n}^{[m-1,\mu]}(x+y,b,d;\lambda;u,v)
\\
& \label{eq23b} =\displaystyle\sum\limits_{k=0}^{n}(-1)^kk!(c+2k)\varphi_k (x,c)\sum\limits_{j=k}^{n}\binom{n}{j}\binom{j}{k}\displaystyle\frac{(\log d)^{j}\mathcal{Q}_{n-j}^{[m-1,\mu]}(y,b,d;\lambda;u,v)}{(c)_{j+k+1}}.
\end{align}
se mantiene entre los polinomios $\mathcal{Q}_{n}^{[m-1,\mu]}(x,b,d;\lambda;u,v)$ y los polinomios de Bessel generalizados $\varphi_n (x,c)$ definidos por
(\ref{sta}).
\end{teo}

\vspace{0.25cm}

\begin{proof}
De acuerdo a la ecuaci�n (\ref{lbess}), haciendo $m=n-j$, tenemos:
\begin{align}
& \label{bess1} x^{n-j}=\displaystyle \sum\limits_{k=0}^{n-j}\binom{n-j}{k}k!\frac{(-1)^k(c+2k)\varphi_k (x,c)}{(c)_{n-j+k+1}}\\
\intertext{sustituyendo (\ref{bess1}) en el miembro de la derecha de (\ref{eq9}), tenemos}
& \nonumber \mathcal{Q}_{n}^{[m-1,\mu]}(x+y,b,d;\lambda;u,v)
\\
& \nonumber =\displaystyle\sum\limits_{j=0}^{n}\binom{n}{j}\mathcal{Q}_{j}^{[m-1,\mu]}(y,b,d;\lambda;u,v)(\log d)^{n-j}\sum\limits_{k=0}^{n-j}\binom{n-j}{k}k!\frac{(-1)^k(c+2k)\varphi_k (x,c)}{(c)_{n-j+k+1}}\\
\intertext{aplicando (\ref{s2}), tenemos}
& \nonumber =\displaystyle\sum\limits_{j=0}^{n}\sum\limits_{k=0}^{n-j}\binom{n}{j}\mathcal{Q}_{j}^{[m-1,\mu]}(y,b,d;\lambda;u,v)(\log d)^{n-j}\binom{n-j}{k}k!\frac{(-1)^k(c+2k)\varphi_k (x,c)}{(c)_{n-j+k+1}}\\
\intertext{usando (\ref{s1}), tenemos}
& \nonumber =\displaystyle\sum\limits_{k=0}^{n}\sum\limits_{j=0}^{n-k}\binom{n}{j}\mathcal{Q}_{j}^{[m-1,\mu]}(y,b,d;\lambda;u,v)(\log d)^{n-j}\binom{n-j}{k}k!\frac{(-1)^k(c+2k)\varphi_k (x,c)}{(c)_{n-j+k+1}}\\
\intertext{por (\ref{s3}), obtenemos}
& \nonumber =\displaystyle\sum\limits_{k=0}^{n}\sum\limits_{j=k}^{n}\binom{n}{n-j}\mathcal{Q}_{n-j}^{[m-1,\mu]}(y,b,d;\lambda;u,v)(\log d)^{j}\binom{j}{k}k!\frac{(-1)^k(c+2k)\varphi_k (x,c)}{(c)_{j+k+1}}.
\end{align}
Finalmente usando (\ref{s2}) nuevamente y (\ref{comb}) se concluye la prueba.
\end{proof}

\vspace{0.25cm}

Haciendo $c=a-1$, $x:=(-x/b_0)$ y amplificando por $\displaystyle \frac{k!}{(a-1)_k}$ en el \textit{Teorema \ref{teobess}}, tenemos el siguiente corolario.

\vspace{0.25cm}

\begin{coro}
La relaci�n
\begin{align}
& \nonumber \mathcal{Q}_{n}^{[m-1,\mu]}(-x/b_0+y,b,d;\lambda;u,v)=\displaystyle\sum\limits_{k=0}^{n}(-1)^k\Gamma(a-1+k)(a-1+2k)y_k(a,b_0,x)\\
& \label{eq23b1}\times\sum\limits_{j=k}^{n}\binom{n}{j}\binom{j}{k}(\log d)^{j}\frac{\mathcal{Q}_{n-j}^{[m-1,\mu]}(y,b,d;\lambda;u,v)}{\Gamma(a+j+k)}.
\end{align}

se mantiene entre los polinomios $\mathcal{Q}_{n}^{[m-1,\mu]}(x,b,d;\lambda;u,v)$ y los polinomios de Bessel generalizados $y_n(a,b_0,x)$ que son soluci�n de (\ref{difeq1}).
\end{coro}

\vspace{0.25cm}

Haciendo $c=1$, $x:=(-x/2)$ y como $(1)_n=n!$ en el \textit{Teorema \ref{teobess}}, tenemos el siguiente corolario.

\vspace{0.25cm}

\begin{coro}
La relaci�n
\begin{align}
& \nonumber \mathcal{Q}_{n}^{[m-1,\mu]}(-x/2+y,b,d;\lambda;u,v)=\displaystyle\sum\limits_{k=0}^{n}(-1)^kk!(2k+1)y_k(x)\\
& \label{eq23b2}\times\sum\limits_{j=k}^{n}\binom{n}{j}\binom{j}{k}(\log d)^{j}\frac{\mathcal{Q}_{n-j}^{[m-1,\mu]}(y,b,d;\lambda;u,v)}{(j+k+1)!}.
\end{align}

se mantiene entre los polinomios $\mathcal{Q}_{n}^{[m-1,\mu]}(x,b,d;\lambda;u,v)$ y los polinomios de Bessel $y_n(x)$ que son soluci�n de (\ref{difeq}).
\end{coro}

\vspace{0.25cm}


\begin{obs}
Si establecemos $m=1$ y $b=c=e$ en (\ref{eq11}), (\ref{eq17}), (\ref{eq94}), (\ref{eq14}),(\ref{eq20}), (\ref{eq201}), (\ref{eq23}) y  $m=1$, $b=d=e$ en (\ref{eq23b}) entonces se obtienen los resultados correspondientes de los polinomios tipo Apostol generalizados $\mathcal{F}_{n}^{(\mu)}(x;\lambda;u,v)$.
\end{obs}

\vspace{0.25cm}

\section{La nueva clase de polinomios tipo Apostol generalizados basados en Hermite}\label{3g}
Finalmente, en esta secci�n damos una generalizaci�n de los polinomios tipo Apostol generalizados.\\

Consideremos la identidad (\ref{hermite5}), si hacemos $a=0$ y $b=t\log c$, tenemos:
\begin{equation}\label{h6}
\exp\left(y\frac{\partial^2}{\partial x^2}\right)\{c^{xt}\}=c^{xt+(\log c)yt^2}.
\end{equation}
Aplicando el operador $\exp\left(y\frac{\partial^2}{\partial x^2}\right)$ en (\ref{newpol}) y  por la identidad (\ref{h6}) podemos definir la nueva clase de polinomios tipo Apostol generalizados basados en Hermite ${_H}\mathcal{Q}_{n}^{[m-1,\alpha]}(x,y,b,c;\lambda;u,v)$ por la siguiente funci�n generatriz
\begin{equation}\label{new787}
\displaystyle\left(\frac{(2^ut^v)^{m}}{\lambda b^t+\sum\limits_{l=0}^{m-1}\frac{(t\log b)^l}{l!}}\right)^{\alpha}c^{xt+(\log c)yt^2} =\displaystyle\sum\limits_{n=0}^{\infty}
{_H}\mathcal{Q}_{n}^{[m-1,\alpha]}(x,y,b,c;\lambda;u,v)\frac{t^n}{n!}
\end{equation}
con $|t\log b|<|log(-\lambda)|$.

\vspace{0.25cm}

\begin{obs}
Si hacemos $m=1$ y $b=c=e$ en (\ref{new787}) tenemos ${_H}\mathcal{Q}_{n}^{(\alpha)}(x,y,e,e;\lambda;u,v)={_H}\mathcal{F}_{n}^{(\alpha)}(x,y;\lambda;u,v)$ la cual es justamente la Ec.(4.5) de \cite{atp1}.
\end{obs}

\vspace{0.25cm}

\begin{teo}
La nueva clase de polinomios tipo Apostol generalizados basados en Hermite ${_{H}}\mathcal{Q}_n^{[m-1,\alpha]}(x,b,c;\lambda;u,v)$ satisfacen las siguientes relaciones:
\begin{align}
\label{eq823i}
{_{H}}\mathcal{Q}_n^{[m-1,\alpha]}(x,b,c;\lambda;u,v)=& \displaystyle\sum\limits_{j=0}^{n}
\binom{n}{j}\mathcal{Q}_{n-j}^{[m-1,\alpha-1]}(b,c;\lambda;u,v){_{H}}\mathcal{Q}_j^{[m-1]}(x,b,c;\lambda;u,v),\\
\label{eq349i}
{_{H}}\mathcal{Q}_n^{[m-1,\alpha]}(x,b,c;\lambda;u,v)=& \displaystyle\sum\limits_{j=0}^{n}
\binom{n}{j}\mathcal{Q}_{n-j}^{[m-1,\alpha]}(b,c;\lambda;u,v)(\log c)^{j}H_n(x,y),
\end{align}
\end{teo}

\begin{proof}
Aplicando el operador $\exp\left(y\frac{\partial^2}{\partial x^2}\right)$  en (\ref{eq823}) y (\ref{eq349}) y luego usando (\ref{hermite4}) en el miembro de la derecha de (\ref{eq349}) la prueba se concluye.
\end{proof}

\vspace{0.25cm}

\begin{obs}
Si hacemos $m=1$ y $b=c=e$ en (\ref{eq823i}) y (\ref{eq349i}) tenemos  
\begin{align}
\label{eq349ij}
{_{H}}\mathcal{F}_n^{(\alpha)}(x;\lambda;u,v)=& \displaystyle\sum\limits_{j=0}^{n}
\binom{n}{j}\mathcal{F}_{n-j}^{(\alpha)}(\lambda;u,v)H_n(x,y),\\
\label{eq823ij}
{_{H}}\mathcal{F}_n^{(\alpha)}(x;\lambda;u,v)=& \displaystyle\sum\limits_{j=0}^{n}
\binom{n}{j}\mathcal{F}_{n-j}^{(\alpha-1)}(b,c;\lambda;u,v){_{H}}\mathcal{F}_j^{[m-1]}(x;\lambda;u,v),
\end{align}
las cuales son justamente las Ecuaciones (4.9) y (4.10) de \cite{atp1} respectivamente.
\end{obs}

\vspace{0.25cm}

Podemos definir las nuevas clases de los polinomios de Bernoulli basados en Hermite, Euler basados en Hermite y Genocchi basados en Hermite as�:
\begin{align}
\label{newji}
\displaystyle\left(\frac{t^{m}}{\lambda b^t-\sum\limits_{l=0}^{m-1}\frac{(t\log b)^l}{l!}}\right)^{\alpha}c^{xt+(\log c)yt^2} = & \displaystyle\sum\limits_{n=0}^{\infty}
{_H}\mathcal{B}_{n}^{[m-1,\alpha]}(x,b,c;\lambda)\frac{t^n}{n!}\\
\intertext{con $|t\log b|<|log(\lambda)|$,}
\label{newji1} \displaystyle\left(\frac{2^{m}}{\lambda b^t+\sum\limits_{l=0}^{m-1}\frac{(t\log b)^l}{l!}}\right)^{\alpha}c^{xt+(\log c)yt^2} = & \displaystyle\sum\limits_{n=0}^{\infty}
{_H}\mathcal{E}_{n}^{[m-1,\alpha]}(x,b,c;\lambda)\frac{t^n}{n!}\\
\intertext{con $|t\log b|<|log(-\lambda)|$,}
\label{newji2}\displaystyle\left(\frac{2^{m}t^{m}}{\lambda b^t+\sum\limits_{l=0}^{m-1}\frac{(t\log b)^l}{l!}}\right)^{\alpha}c^{xt+(\log c)yt^2} = & \displaystyle\sum\limits_{n=0}^{\infty}
{_H}\mathcal{G}_{n}^{[m-1,\alpha]}(x,b,c;\lambda)\frac{t^n}{n!}\\
\intertext{con $|t\log b|<|log(-\lambda)|$.}\nonumber
\end{align}




 
\chapter{CONCLUSIONES Y RECOMENDACIONES}
\markboth{CAP\'ITULO 4\quad CONCLUSIONES Y RECOMENDACIONES}{CAP\'ITULO 4\quad CONCLUSIONES Y RECOMENDACIONES}
\label{AB4}

%recuperar numeracion arabica%
\def\thechapter{\arabic{chapter}}

En este \'ultimo cap\'itulo haremos un resumen de las t\'ecnicas y resultados fundamentales de nuestro trabajo y expondremos problemas abiertos y futuras l\'ineas de investigaci\'on relativos a *****.

\section{Conclusiones}
Podemos decir que el resultado m\'as importante de este trabajo es el relacionado con *****, (ver teorema ****).



\section{Problemas Abiertos y Futuras L\'{\i}neas de Investigaci\'on}

De lo expuesto en la secci\'on precedente se desprende que algunos  problemas interesantes que permitir\'{\i}an  continuar con investigaciones relacionadas, ser\'{\i}an:


\textbf{Problema 1.}
Series de Fourier y representaci\'on integral de los polinomios $\mathcal{Q}_n^{[m-1,\alpha]}(x,b,c;\lambda;u,v)$.

\textbf{Problema 2.}
Estudio de las propiedades de los polinomios $\mathcal{Q}_n^{(\alpha)}(x;\lambda;a,b,c;u,v)$ definidos en (\ref{eqnewpol2}) y relaci\'on con otros polinomios y n\'umeros.

\textbf{Problema 3.}
Series de Fourier y representaci\'on integral de los polinomios $\mathcal{Q}_n^{(\alpha)}(x;\lambda;a,b,c;u,v)$ definidos en (\ref{eqnewpol2}).

\textbf{Problema 4.}
Estudio de la nueva clase de polinomios tipo Apostol generalizados basados en Hermite ${_{H}}\mathcal{Q}_n^{[m-1,\alpha]}(x,b,c;\lambda;u,v)$ que est\'an definidos en (\ref{new787}).

\textbf{Problema 5.}
Estudio de otra nueva clase de polinomios tipo Apostol generalizados basados en Hermite ${_{H}}\mathcal{Q}_n^{(\alpha)}(x;\lambda;a,b,c;u,v)$ que est\'an definidos por la siguiente funci\'on generatriz:
\begin{equation}\label{eqnewpol23}
\displaystyle\left(\frac{2^uz^v}{\lambda b^z+a^z}\right)^{\alpha}c^{xz+(\log c)yz^2} =\sum\limits_{n=0}^{\infty}
{_{H}}\mathcal{Q}_n^{(\alpha)}(x;\lambda;a,b,c;u,v)\frac{z^n}{n!} 
\end{equation}




  \include{bibliografia}
\end{document}









