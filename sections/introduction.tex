%%%%%%%%%%%%%%%%%%%%%%%%%%%%%%%%

% Introduction 

%%%%%%%%%%%%%%%%%%%%%%%%%%%%%%%%

\chapter[Introduction]{\textbf{I}ntroduction}\label{sec:intro}
\sectionblue*{Summary}
\begin{center}
\begin{tabular}{c}
\fcolorbox{blue}{verylightgrey}{
\begin{minipage}[][4cm][c]{0.8\linewidth}
\sffamily
%abstract 
empezarenos con algunos  fundamentos teoricos de los
sistemas de ecuaciones diferencial no lineales, presentamos algunos conceptos preliminares y deniciones. En primer lugar, solo consideraremos sistemas
autonomos de ecuaciones diferenciales ordinarias\\
\begin{center}
\begin{equation}
\left\{
\begin{array}{lcc}
\dot{x}=f(x)\\
x(0)=y
\end{array}\right.
\end{equation}
\end{center}
\textbf{Teorema (Dependencia de la Condicion Inicial)}  
Sea E un subconjunto abierto de $\mathbb{R}^{n}$ que contiene a $x_{0}$ y suponga que $f\in  C^{1}(E)$. Entonces existe $a>0$\ y \ $\delta >0$ tal que para todo $y\in N_{\delta}(x_{0})$ el problema de valor inicial
\begin{center}
\begin{equation}
\left\{
\begin{array}{lcc}
\dot{x}=f(x)\\
x(0)=y
\end{array}\right.
\end{equation}
\end{center}
tiene una unica solucion $u(t,y)$ con $u\in c^{1}(G)$ con $G=[-a,a]N_{\delta}(x_{0})\subset \mathbb{R}^{n+1} $ ademas, $\forall y\in N_{\delta}(x_{0}), \ u(t,y)$ es una funcion dos veces continuamente diferenciable para $t\in [-a,a]$\\
\textbf{ Teorema  (Dependencia de Parametros)} Sea E un subconjunto abierto de $\mathbb{R}^{n+m}$ que contiene $(x_{0},\mu _{0})$ donde $ x_{0}\in \mathbb{R}^{n}$ \ y \ $\mu _{0} \in \mathbb{R}^{m}$ y supongamos que $f\in C^{1}(E)$, Entonces existe $a>0$\ y \ $\delta >0$ tal que para todo $y\in N_{\delta}(x_{0})$ y $\mu \in N_{\delta}(\mu _{0})$  el problema de valor inicial
\begin{center}
\begin{equation}
\left\{
\begin{array}{lcc}
\dot{x}=f(x,\mu )\\
x(0)=y
\end{array}\right.
\end{equation}
\end{center} 
 tiene una unica solucion $u(t,y,\mu )$ con $u\in C^{1}(G)$ donde $G=[-a,a]* N_{\delta}(x_{0})* \in N_{\delta}(\mu _{0})$\\
 \textbf{DEFINICION(flujo definido por una ecuacion diferencial):} Sea E un subconjunto abierto de $\mathbb{R}^{n}$ y sea $f\in C^{1}(E)$ para $x_{0}\in E$, sea $\phi (t,x_{0})$ la solucion del problema de valor inicial. Definido sobre su intervalo maximo de existencia $I(x_{0})$ entoces para $t\in I(x_{0})$, el conjunto de funciones $\phi _{t}$ definido por:\\
 \begin{center}
 $\phi _{t}(x_{0})= \phi (t,x_{0})$
\end{center}       
es llamado flujo definido por la ecuacion diferencial es tambien referido como el flujo del campo vectorial $f(x)$\\
el flujo $\phi _{t}$ satisfacelas siguientes propiedades:\\
1) $\phi _{0}(x)=x$\\
2) $\phi _{s}(\phi _{t}(x))=\phi _{s+t}(x)$, para todo $s,t\in \mathbb{R}$\\
3)$ \phi _{-t}(\phi _{t}(x))$, para todo $t\in \mathbb{R} $\\
\textbf{DEFINICION:} sea E un subconjunto abierto de $\mathbb{R}^{n}$, sea $f\in C^{1}(E)$ y sea $\phi _{t}:E\longrightarrow E $, el flujo de la ecuacion diferencial, para todo $t\in \mathbb{R}$ entonces el conjunto $ S\subset E$ es llamado invariante con respecto al flujo $\phi _{t}$ si $\phi _{t}(S)\subset S$ para todo $t\in \mathbb{R}$\\
\textbf{DEFINICION:} un punto $x_{0}\in \mathbb{R}^{n}$ es un punto de equilibrio o un punto critico si $f(x_{0})=0$. Un punto de equilibrio $x_{0}$ es hiperbolico si ninguno de los valores propios de la matriz $Df(x_{0})$ tiene parte real cero.\\
\textbf{NOTA:} note que si $x_{0}$ es un punto de equilibrio de $\dot{x} f(x) \ y\ \phi _{t}:E\longrightarrow \mathbb{R}^{n}$  es el flujo definido por la ecuacion diferencial entonces $\phi _{t}(x_{0})=x_{0}$ para todo $t\in \mathbb{R}$ asi $ x_{0}$ es un punto fijo del flujo $\phi _{t}$; este es tambien llamado cero, punto critico, o punto singular del campo vectorial $f:E\longrightarrow \mathbb{R}^{n}$\\
\textbf{DEFINICION:} un punto de equilibrio $x_{0}$ es atractor si todos los valores propios de la matriz $Df(x_{0})$ tienen parte real negativa; es una fuente o un repulsor si todos los valores propios de $Df(x_{0})$ tienen parte real positiva; y es una silla si el punto de equilibrio es hiperbolico y $Df(x_{0})$ tiene al menos un valor propio con parte real positiva y al menos un valor propio con parte real negativa.\\
\textbf{DEFINICION (equivalencia de ecuaciones diferenciales ordinarias):} decimos que dos o mas ecuaciones diferenciales son equivalentes topologicamente si existe un homeomorfismo que envia orbitas en orbitas manteniendo la orientacion.\\
\textbf{DEFINICION:} $x_{0}$ es un punto de equilibrio estable si ningun valor propio de la matriz $Df(x_{0})$ tiene parte real positiva.     empezarenos con algunos  fundamentos teoricos de los
sistemas de ecuaciones diferencial no lineales, presentamos algunos conceptos preliminares y deniciones. En primer lugar, solo consideraremos sistemas
autonomos de ecuaciones diferenciales ordinarias\\
\begin{center}
\begin{equation}
\left\{
\begin{array}{lcc}
\dot{x}=f(x)\\
x(0)=y
\end{array}\right.
\end{equation}
\end{center}
\textbf{Teorema (Dependencia de la Condicion Inicial)}  
Sea E un subconjunto abierto de $\mathbb{R}^{n}$ que contiene a $x_{0}$ y suponga que $f\in  C^{1}(E)$. Entonces existe $a>0$\ y \ $\delta >0$ tal que para todo $y\in N_{\delta}(x_{0})$ el problema de valor inicial
\begin{center}
\begin{equation}
\left\{
\begin{array}{lcc}
\dot{x}=f(x)\\
x(0)=y
\end{array}\right.
\end{equation}
\end{center}
tiene una unica solucion $u(t,y)$ con $u\in c^{1}(G)$ con $G=[-a,a]N_{\delta}(x_{0})\subset \mathbb{R}^{n+1} $ ademas, $\forall y\in N_{\delta}(x_{0}), \ u(t,y)$ es una funcion dos veces continuamente diferenciable para $t\in [-a,a]$\\
\textbf{ Teorema  (Dependencia de Parametros)} Sea E un subconjunto abierto de $\mathbb{R}^{n+m}$ que contiene $(x_{0},\mu _{0})$ donde $ x_{0}\in \mathbb{R}^{n}$ \ y \ $\mu _{0} \in \mathbb{R}^{m}$ y supongamos que $f\in C^{1}(E)$, Entonces existe $a>0$\ y \ $\delta >0$ tal que para todo $y\in N_{\delta}(x_{0})$ y $\mu \in N_{\delta}(\mu _{0})$  el problema de valor inicial
\begin{center}
\begin{equation}
\left\{
\begin{array}{lcc}
\dot{x}=f(x,\mu )\\
x(0)=y
\end{array}\right.
\end{equation}
\end{center} 
 tiene una unica solucion $u(t,y,\mu )$ con $u\in C^{1}(G)$ donde $G=[-a,a]* N_{\delta}(x_{0})* \in N_{\delta}(\mu _{0})$\\
 \textbf{DEFINICION(flujo definido por una ecuacion diferencial):} Sea E un subconjunto abierto de $\mathbb{R}^{n}$ y sea $f\in C^{1}(E)$ para $x_{0}\in E$, sea $\phi (t,x_{0})$ la solucion del problema de valor inicial. Definido sobre su intervalo maximo de existencia $I(x_{0})$ entoces para $t\in I(x_{0})$, el conjunto de funciones $\phi _{t}$ definido por:\\
 \begin{center}
 $\phi _{t}(x_{0})= \phi (t,x_{0})$
\end{center}       
es llamado flujo definido por la ecuacion diferencial es tambien referido como el flujo del campo vectorial $f(x)$\\
el flujo $\phi _{t}$ satisfacelas siguientes propiedades:\\
1) $\phi _{0}(x)=x$\\
2) $\phi _{s}(\phi _{t}(x))=\phi _{s+t}(x)$, para todo $s,t\in \mathbb{R}$\\
3)$ \phi _{-t}(\phi _{t}(x))$, para todo $t\in \mathbb{R} $\\
\textbf{DEFINICION:} sea E un subconjunto abierto de $\mathbb{R}^{n}$, sea $f\in C^{1}(E)$ y sea $\phi _{t}:E\longrightarrow E $, el flujo de la ecuacion diferencial, para todo $t\in \mathbb{R}$ entonces el conjunto $ S\subset E$ es llamado invariante con respecto al flujo $\phi _{t}$ si $\phi _{t}(S)\subset S$ para todo $t\in \mathbb{R}$\\
\textbf{DEFINICION:} un punto $x_{0}\in \mathbb{R}^{n}$ es un punto de equilibrio o un punto critico si $f(x_{0})=0$. Un punto de equilibrio $x_{0}$ es hiperbolico si ninguno de los valores propios de la matriz $Df(x_{0})$ tiene parte real cero.\\
\textbf{NOTA:} note que si $x_{0}$ es un punto de equilibrio de $\dot{x} f(x) \ y\ \phi _{t}:E\longrightarrow \mathbb{R}^{n}$  es el flujo definido por la ecuacion diferencial entonces $\phi _{t}(x_{0})=x_{0}$ para todo $t\in \mathbb{R}$ asi $ x_{0}$ es un punto fijo del flujo $\phi _{t}$; este es tambien llamado cero, punto critico, o punto singular del campo vectorial $f:E\longrightarrow \mathbb{R}^{n}$\\
\textbf{DEFINICION:} un punto de equilibrio $x_{0}$ es atractor si todos los valores propios de la matriz $Df(x_{0})$ tienen parte real negativa; es una fuente o un repulsor si todos los valores propios de $Df(x_{0})$ tienen parte real positiva; y es una silla si el punto de equilibrio es hiperbolico y $Df(x_{0})$ tiene al menos un valor propio con parte real positiva y al menos un valor propio con parte real negativa.\\
\textbf{DEFINICION (equivalencia de ecuaciones diferenciales ordinarias):} decimos que dos o mas ecuaciones diferenciales son equivalentes topologicamente si existe un homeomorfismo que envia orbitas en orbitas manteniendo la orientacion.\\
\textbf{DEFINICION:} $x_{0}$ es un punto de equilibrio estable si ningun valor propio de la matriz $Df(x_{0})$ tiene parte real positiva.     