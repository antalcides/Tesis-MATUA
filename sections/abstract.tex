\begin{small}
%BACKGROUND
The sequences are very versatile data structures. In a straightforward manner,
a sequence of symbols can store any type of information. Systematic analysis
of sequences is a very rich area of algorithmics, with lots of successful
applications. The comparison by sequence alignment is a very powerful analysis 
tool. Dynamic programming is one of the most popular and efficient approaches 
to align two sequences. However, despite their utility, alignments
are not always the best option for characterizing the function of two
sequences. Sequences often encode information in different levels of
organization (meta-information). In these cases, direct sequence comparison
is not able to unveil those higher-order structures that can actually explain
the relationship between the sequences.\\

%METHODS
We have contributed with the work presented here to improve the way in which
two sequences can be compared, developing a new family of algorithms that
align high level information encoded in biological sequences (meta-alignment).
Initially, we have redesigned an existent algorithm, based in dynamic programming,
to align two sequences of meta-information, introducing later several improvements
for a better performance. Next, we have developed a multiple meta-alignment algorithm,
by combining the general algorithm with the progressive schema. In addition,
we have studied the properties of the resulting meta-alignments, modifying the
algorithm to identify non-collinear or permuted configurations.\\

%RESULTS
Molecular life is a great example of the sequence versatility. Comparative genomics
provide the identification of numerous biologically functional elements. The
nucleotide sequence of many genes, for example, is relatively well conserved
between different species. In contrast, the sequences that regulate the gene
expression are shorter and weaker. Thus, the simultaneous activation of a set
of genes only can be explained in terms of conservation between configurations
of higher-order regulatory elements, that can not be detected at the sequence level.
We, therefore, have trained our meta-alignment programs in several datasets
of regulatory regions collected from the literature. Then, we have tested the
accuracy of our approximation to successfully characterize the promoter regions
of human genes and their orthologs in other species.\\
\end{small}
