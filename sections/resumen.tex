
\addcontentsline{toc}{chapter}{RESUMEN}

\chapter*{RESUMEN}

\vspace{1.0cm}


\hspace{0.5cm}
En este trabajo estudiamos una nueva clase de polinomios tipo Apostol generalizados $\mathcal{Q}_n^{[m-1,\alpha]}(x,b,c;\lambda;u,v)$ con $(\alpha, u, v\in \CC$ y $b, c \in \RR^{+})$ los cuales est�n definidos en un entorno adecuado de $t=0$ por la siguiente funci�n generatriz:
\begin{equation}
\displaystyle\left(\frac{(2^ut^v)^{m}}{\lambda b^t+\sum\limits_{l=0}^{m-1}\frac{(t\log b)^l}{l!}}\right)^{\alpha}c^{xt} =\displaystyle\sum\limits_{n=0}^{\infty}
\mathcal{Q}_n^{[m-1,\alpha]}(x,b,c;\lambda;u,v)\frac{t^n}{n!}; \quad |t\log b|<|log(-\lambda)|,
\end{equation}
estos polinomios generan a las nuevas clases de todos los polinomios conocidos de Apostol, as�:\\
\begin{align}
& \mathcal{B}_n^{[m-1,\alpha]}(x;b,c;\lambda)=(-1)^{\alpha}\mathcal{Q}_n^{[m-1,\alpha]}(x;b,c;-\lambda;0,1),\\
& \mathcal{E}_n^{[m-1,\alpha]}(x;b,c;\lambda)=\mathcal{Q}_n^{[m-1,\alpha]}(x;b,c;\lambda;1,0),\\
& \mathcal{G}_n^{[m-1,\alpha]}(x;b,c;\lambda)=\mathcal{Q}_n^{[m-1,\alpha]}(x;b,c;\lambda;1,1).
\end{align}

Establecemos algunas propiedades b�sicas para estos polinomios, incluyendo relaci�n de recurrencia y la ecuaci�n diferencial que estos polinomios satisfacen. Finalmente se determinan f�rmulas de conexi�n entre los polinomios y los polinomios de Genocchi, los polinomios de Jacobi, los polinomios de Hermite, los polinomios de Laguerre, los polinomios de Charlier, los polinomios de Bessel, los polinomios de Bernoulli generalizados $B_n^{[m-1]}(x)$ y los n�meros de Stirling de segunda clase, lo cual extiende algunos resultados conocidos. Finalmente introducimos una nueva clase de polinomios tipo Apostol generalizados basados en los polinomios de Hermite de dos variables ${_{H}}\mathcal{Q}_n^{[m-1,\alpha]}(x,b,c;\lambda;u,v)$ y mencionamos propiedades b�sicas.


\vspace{2cm}

\noindent \textbf{Palabras Claves:} polinomios tipo Apostol generalizados; relaci�n de recurrencia; ecuaci�n diferencial; polinomios de Apostol-Bernoulli; polinomios de Genocchi; polinomios de Jacobi; polinomios de Hermite; polinomios de Laguerre; polinomios de Charlier; polinomios de Bessel; polinomios de Bernoulli generalizados; n�meros de Stirling de segunda clase.


