\chapter{SOBRE LAS DISTINTAS GENERALIZACIONES DE POLINOMIOS  DE APOSTOL-BERNOULLI, DE APOSTOL-EULER Y DE APOSTOL-GENOCCHI}\markboth{CAP\'ITULO 2\, GENERALIZACIONES DE LOS POL. DE A-B, A-E Y A-G}{CAP\'ITULO 2\, GENERALIZACIONES DE LOS POL. DE A-B, DE A-E Y A-G}




%recuperar numeracion arabica%
\def\thechapter{\arabic{chapter}}

Consideremos los polinomios de Apostol-Bernoulli $\mathfrak{B}_n(x;\lambda)$, los polinomios de Apostol-Euler $\mathfrak{E}_n(x;\lambda)$ y los polinomios de Apostol-Genocchi $\mathfrak{G}_n(x;\lambda)$, definidos en (\ref{eq3}), (\ref{eq300}) y (\ref{eq301}) respectivamente. En este cap\'{\i}tulo mostraremos, las diferentes generalizaciones que han tenido estos polinomios hasta llegar a una de las m�s actuales conocidas. Lo anterior implica que estudiaremos fundamentalmente las definiciones y algunas propiedades de los polinomios de Apostol-Euler generalizados, los polinomios de Apostol-Genocchi generalizados, las nuevas familias y clases de polinomios de Apostol-Bernoulli, de Apostol-Euler y de Apostol-Genocchi generalizados y finalmente los polinomios tipo Apostol generalizados. 

\section{Polinomios de Apostol-Bernoulli, de Apostol-Euler y de Apostol-Genocchi generalizados}
Muchas interesantes generalizaciones de los polinomios de Apostol-Bernoulli, Apostol-Euler y Apostol-Genocchi han sido dadas. En part�cular Luo y Srivastava \cite{gen, REF9} introdujeron los polinomios de Apostol-Bernoulli generalizados
$\mathfrak{B}_{n}^{(\alpha)}(x;\lambda)$ de orden $\alpha \in \CC$; Luo \cite{REF6} invent� los polinomios de Apostol-Euler $\mathfrak{E}_{n}^{(\alpha)}(x;\lambda)$ de orden $\alpha \in \CC$ y los polinomios de Genocchi $\mathfrak{G}_{n}^{(\alpha)}(x;\lambda)$ de orden $\alpha \in \CC$ en \cite{Genpolt}

\begin{definition}\label{abp}
Los polinomios de Apostol-Bernoulli generalizados $\mathfrak{B}_{n}^{(\alpha)}(x;\lambda)$, los polinomios de Apostol Euler generalizados $\mathfrak{E}_{n}^{(\alpha)}(x;\lambda)$ y los polinomios de Apostol-Genocchi generalizados $\mathfrak{G}_{n}^{(\alpha)}(x;\lambda)$ de orden $\alpha \in \CC$ est�n definidos por las siguientes funciones generatrices:
\begin{equation}\label{eqae}
\displaystyle\left(\frac{z}{\lambda e^z-1}\right)^{\alpha}e^{xz} =\displaystyle \sum\limits_{n=0}^{\infty}
\mathfrak{B}_{n}^{(\alpha)}(x;\lambda)\frac{z^n}{n!}; \quad (|z|<|\log(\lambda)|),
\end{equation}

\begin{equation}\label{eqae1}
\displaystyle\left(\frac{2}{\lambda e^z+1}\right)^{\alpha}e^{xz} =\displaystyle \sum\limits_{n=0}^{\infty}
\mathfrak{E}_{n}^{(\alpha)}(x;\lambda)\frac{z^n}{n!}; \quad (|t|<|\log(-\lambda)|),
\end{equation}

\begin{equation}\label{eqae2}
\displaystyle\left(\frac{2z}{\lambda e^z+1}\right)^{\alpha}e^{xz} =\displaystyle \sum\limits_{n=0}^{\infty}
\mathfrak{G}_{n}^{(\alpha)}(x;\lambda)\frac{z^n}{n!}; \quad (|z|<|\log(-\lambda)|).
\end{equation}
\end{definition}

\vspace{0.25cm}

\begin{obs}
Los polinomios de Apostol-Bernoulli $\mathfrak{B}_{n}(x;\lambda) = \mathfrak{B}_{n}^{(1)}(x;\lambda)$ y los polinomios de Apostol-Bernoulli generalizados $B_{n}^{(\alpha)}(x) = \mathfrak{B}_{n}^{(\alpha)}(x;1)$.\\
Los polinomios de Apostol-Euler $\mathfrak{E}_{n}(x;\lambda) = \mathfrak{E}_{n}^{(1)}(x;\lambda)$ y los polinomios de Apostol-Euler generalizados $E_{n}^{(\alpha)}(x) = \mathfrak{E}_{n}^{(\alpha)}(x;1)$.\\
Los polinomios de Apostol-Genocchi $\mathfrak{G}_{n}(x;\lambda) = \mathfrak{G}_{n}^{(1)}(x;\lambda)$ y los polinomios de Apostol-Genocchi generalizados $G_{n}^{(\alpha)}(x) = \mathfrak{G}_{n}^{(\alpha)}(x;1)$.
\end{obs}


\section{Nuevas familias de polinomios de Apostol-Euler y de Apostol-Genocchi generalizados}
En este apartado presentaremos dos familias de polinomios de Apostol-Euler y Apostol-Genocchi generalizados  que fueron introducidas por Srivastava et. al.\cite{Srivasnew}. Ellos investigaron las siguientes formas.

\vspace{0.25cm}
\begin{definition}\label{abs}
Sean $a,b,c \in \RR^{+} \quad (a\neq b)$ y $n \in \NN_0:=\NN \cup \{0\}$. Entonces los polinomios de Apostol-Euler generalizados $\mathfrak{E}_n^{(\alpha)}(x;\lambda;a,b,c)$ de orden $\alpha \in \CC$ est�n definidos por la siguiente funci�n generatriz:
\begin{equation}\label{eqaps}
\displaystyle\left(\frac{2}{\lambda b^z+a^z}\right)^{\alpha}c^{xz} =\sum\limits_{n=0}^{\infty}
\mathfrak{E}_n^{(\alpha)}(x;\lambda;a,b,c)\frac{z^n}{n!} 
\end{equation}

\begin{center}
$\displaystyle\quad \left(\left|z\log \left(\frac{b}{a}\right)\right|<|\log(-\lambda)|; 1^\alpha:=1; x\in \RR\right)$
\end{center}
\end{definition}

\vspace{0.25cm}

\begin{definition}\label{abs1}
Sean $a,b,c \in \RR^{+} \quad (a\neq b)$ y $n \in \NN_0:=\NN \cup \{0\}$. Entonces los polinomios Apostol-Genocchi generalizados $\mathfrak{G}_n^{(\alpha)}(x;\lambda;a,b,c)$ de orden $\alpha \in \CC$ est�n definidos por la siguiente funci�n generatriz:
\begin{equation}\label{eqaps1}
\displaystyle\left(\frac{2z}{\lambda b^z+a^z}\right)^{\alpha}c^{xz} =\sum\limits_{n=0}^{\infty}
\mathfrak{G}_n^{(\alpha)}(x;\lambda;a,b,c)\frac{z^n}{n!} 
\end{equation}

\begin{center}
$\displaystyle\quad \left(\left|z\log \left(\frac{b}{a}\right)\right|<|\log(-\lambda)|; 1^\alpha:=1; x\in \RR\right)$
\end{center}
\end{definition}

\vspace{0.25cm}

\section{Nuevas clases de polinomios de Apostol-Bernoulli, de Apostol-Euler y de Apostol-Genocchi generalizados}
En esta secci�n ilustraremos algunas nuevas clases de polinomios de Apostol-Bernoulli generalizados $\mathcal{B}_n^{[m-1,\alpha]}(x,b,c;\lambda)$ que fueron introducidos por Tremblay, Gaboury y Fug\`{e}re \cite{Tremblay1} y nuevas clases de polinomios de Apostol-Euler $\mathcal{E}_n^{[m-1,\alpha]}(x,b,c;\lambda)$ y Apostol-Genocchi generalizados $\mathcal{G}_n^{[m-1,\alpha]}(x,b,c;\lambda)$ que fueron introducidos por los mismos autores en \cite{Tremblay2}.\\

\begin{definition}\label{Gabory}
Para par�metros complejos o reales arbitrarios $\alpha$ y para $b,c \in \RR^{+}$, la nueva clase de polinomios Apostol-Bernoulli generalizados $\mathcal{B}_n^{[m-1,\alpha]}(x,b,c;\lambda)$, la nueva clase de polinomios Apostol-Euler generalizados $\mathcal{E}_n^{[m-1,\alpha]}(x,b,c;\lambda)$ y la nueva clase de polinomios Apostol-Genocchi generalizados $\mathcal{G}_n^{[m-1,\alpha]}(x,b,c;\lambda)$, $m \in \NN$, $\lambda \in \CC$ est�n definidos en un entorno adecuado de $z=0$, a trav�s de la siguientes funciones generatrices:

\begin{equation}\label{eq5}
\displaystyle\left(\frac{z^{m}}{\lambda b^z-\sum\limits_{l=0}^{m-1}\frac{(z\log b)^l}{l!}}\right)^{\alpha}c^{xz} =\displaystyle\sum\limits_{n=0}^{\infty}
\mathcal{B}_n^{[m-1,\alpha]}(x,b,c;\lambda)\frac{z^n}{n!},
\end{equation}

\begin{equation}\label{eq51}
\displaystyle\left(\frac{2^{m}}{\lambda b^z+\sum\limits_{l=0}^{m-1}\frac{(z\log b)^l}{l!}}\right)^{\alpha}c^{xz} =\displaystyle\sum\limits_{n=0}^{\infty}
\mathcal{E}_n^{[m-1,\alpha]}(x,b,c;\lambda)\frac{z^n}{n!},
\end{equation}

\begin{equation}\label{eq52}
\displaystyle\left(\frac{(2z)^{m}}{\lambda b^z+\sum\limits_{l=0}^{m-1}\frac{(z\log b)^l}{l!}}\right)^{\alpha}c^{xz} =\displaystyle\sum\limits_{n=0}^{\infty}
\mathcal{G}_n^{[m-1,\alpha]}(x,b,c;\lambda)\frac{zs^n}{n!}.
\end{equation}

\end{definition}

\vspace{0.25cm}

\begin{obs}
Si hacemos $b=c=e$ en (\ref{eq5}), (\ref{eq51}) y (\ref{eq52}) obtenemos

\begin{equation}\label{eq50}
\displaystyle\left(\frac{z^{m}}{\lambda e^z-\sum\limits_{l=0}^{m-1}\frac{z^l}{l!}}\right)^{\alpha}e^{xz} =\displaystyle\sum\limits_{n=0}^{\infty}
\mathcal{B}_n^{[m-1,\alpha]}(x;\lambda)\frac{z^n}{n!},
\end{equation}

\begin{equation}\label{eq511}
\displaystyle\left(\frac{2^{m}}{\lambda e^z+\sum\limits_{l=0}^{m-1}\frac{z^l}{l!}}\right)^{\alpha}e^{xz} =\displaystyle\sum\limits_{n=0}^{\infty}
\mathcal{E}_n^{[m-1,\alpha]}(x;\lambda)\frac{z^n}{n!},
\end{equation}

\begin{equation}\label{eq522}
\displaystyle\left(\frac{(2z)^{m}}{\lambda e^z+\sum\limits_{l=0}^{m-1}\frac{z^l}{l!}}\right)^{\alpha}e^{xz} =\displaystyle\sum\limits_{n=0}^{\infty}
\mathcal{G}_n^{[m-1,\alpha]}(x;\lambda)\frac{z^n}{n!}.
\end{equation}

\end{obs}

\vspace{0.25cm}

\begin{obs}
Si hacemos $m=1$ en (\ref{eq50}), (\ref{eq511}) y (\ref{eq522}) obtenemos (\ref{eqae}), (\ref{eqae1}) y (\ref{eqae2}) respectivamente.
\end{obs}

\vspace{0.25cm}

\begin{lemma}
 Los polinomios de Bernoulli generalizados $B_{n}^{[m-1]}(x):=\mathcal{B}_{n}^{[m-1,1]}(x;1)$, $m \in \NN$ verifican la siguiente relaci�n:
\begin{equation}\label{eq60}
x^n=\displaystyle\sum\limits_{k=0}^{n}\binom{n}{k}\frac{k!}{(k+m)!}B_{n-k}^{[m-1]}(x).
\end{equation}
\end{lemma}

\vspace{0.25cm}

Para detalles de la prueba del anterior resultados, ver \cite{REF_14}, p.158, Ec.(2.6).


\vspace{0.25cm}

\section{Polinomios tipo Apostol generalizados}
Consideramos en esta secci�n la generalizaci�n m�s importante y m�s actual para los polinomios de Apostol-Bernoulli, Apostol-Euler y Apostol-Gencocchi, la cual fue introducida en 2013 por Lu y Luo en \cite{atp1}.\\

\begin{definition}\label{abp1}
Los polinomios tipo Apostol generalizados $\mathcal{F}_n^{(\alpha)}(x;\lambda;u,v) \quad \alpha \in \NN_0,\lambda, u, v \in \CC)$ de orden $\alpha$ est�n definidos a trav�s de la siguiente funci�n generatriz:
\begin{equation}\label{eqapt1}
\displaystyle\left(\frac{2^{u}z^{v}}{\lambda e^z+1}\right)^{\alpha}e^{xz} =\sum\limits_{n=0}^{\infty}
\mathcal{F}_n^{(\alpha)}(x;\lambda;u,v)\frac{z^n}{n!}; \quad (|z|<|log(-\lambda)|),
\end{equation}
donde
\begin{equation}
\mathcal{F}_n^{(\alpha)}(\lambda;u,v)=:\mathcal{F}_n^{(\alpha)}(0;\lambda;u,v) 
\end{equation}
denotan los tambi�n llamados n�meros de tipo Apostol de orden $\alpha$.
\end{definition}

\vspace{0.25cm}

So that, by comparing \textit{Definition \ref{abp1}} and \textit{Definition \ref{abp}} 
\begin{align}
& \mathfrak{B}_n^{(\alpha)}(x;\lambda)=(-1)^{\alpha}\mathcal{F}_n^{(\alpha)}(x;-\lambda;0,1),\\
& \mathfrak{E}_n^{(\alpha)}(x;\lambda)=\mathcal{F}_n^{(\alpha)}(x;\lambda;1,0),\\
& \mathfrak{G}_n^{(\alpha)}(x;\lambda)=\mathcal{F}_n^{(\alpha)}(x;\lambda;1,1).
\end{align}

\vspace{0.25cm}

