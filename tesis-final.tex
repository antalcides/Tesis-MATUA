\batchmode
\makeatletter
\def\input@path{{style/}{sections/}}
\makeatother
\documentclass[es,en,utf8,utf8x,latin1,letter,dvipsnames]{msc-matua}
\usepackage[latin9]{inputenc}
\usepackage{color}
\usepackage{float}
\usepackage{calc}
\usepackage{amsthm}
\usepackage{amsbsy}
\usepackage{amstext}
\usepackage{graphicx}

\makeatletter

%%%%%%%%%%%%%%%%%%%%%%%%%%%%%% LyX specific LaTeX commands.
%% Because html converters don't know tabularnewline
\providecommand{\tabularnewline}{\\}

%%%%%%%%%%%%%%%%%%%%%%%%%%%%%% Textclass specific LaTeX commands.

\renewcommand*{\LettrineTextFont}{\normalfont}
\renewcommand{\LettrineFontHook}{\color{ptctitle}}
\newcommand{\Hrule}{\rule{\linewidth}{1mm}}
\renewcommand{\thefootnote}{\fnsymbol{footnote}}
\newcommand{\CC}{\mathbb{C}}
%\newcommand{\NN}{\mathbb{N}}
\newcommand{\PP}{\mathbb{P}}
\newcommand{\HH}{\mathbb{H}}
\newcommand{\pp}{\mathbb{\overline{P}}}
\newcommand{\DD}{\mathbb{D}}
\newcommand{\QQ}{\mathbb{Q}}
\newcommand{\RR}{\mathbb{R}}
\newcommand{\ZZ}{\mathbb{Z}}
\newcommand{\EE}{\mathbb{E}}
\newcommand{\TT}{\mathbb{T}}
\newcommand{\XX}{\mathbb{X}}

\newcommand{\II}{\mathbb{I}}

\newcommand{\der}{\mathcal{D}}
\newcommand{\kk}{\mathcal{K}}
\newcommand{\mm}[1]{{\mathcal{M}\/}(#1)}
\newcommand{\nequiv}{{\equiv \hspace*{-3.7mm}/}}
\newcommand{\capa}[1]{\mbox{{\em Cap\/}}(#1)}
\newcommand{\gra}[1]{\mbox{{\em grad\/}}(#1)}
\newcommand{\sop}[1]{\mbox{{\em supp\/}}(#1)}
\newcommand{\esup}[1]{{\mbox{\rm ess\,sup\/}}#1}
\newcommand{\lqqd}{\hfill $\blacksquare$}

\renewcommand{\thefootnote}{\fnsymbol{footnote}}
\newcommand{\ca}[1]{{\em card\/}(#1)}
\newcommand{\dis}[1]{{\em dist\/}(#1)}
\newcommand{\imm}[1]{{\em Im\/}#1}
\newcommand{\ree}[1]{{\em Re\/}#1}
\newcommand{\conv}[1]{{\em conv\/}(#1)}
\newcommand{\uni}{\; \; \overrightarrow{\longrightarrow}\; \;}
\newcommand{\unin}{\overrightarrow{\longrightarrow}}
\newenvironment{ob}{\begin{obs}{\rm} \end{obs}}

\def\bbuildrel#1_#2^#3{\mathrel{
 \mathop{\kern 0pt#1}\limits_{#2}^{#3}}}
\def\bbbuildrel#1_#2{\mathrel{
 \mathop{\kern 0pt#1}\limits_{#2}}}
\def\limsup{\mathop{\overline{\rm lim}}}
\def\liminf{\mathop{\underline{\rm lim}}}

\newenvironment{prueba}[1][Demostraci\'on]{\noindent\textbf{#1.} }{\ $\hfill\blacksquare$\linebreak}

\def\bbuildrel#1_#2^#3{\mathrel{
 \mathop{\kern 0pt#1}\limits_{#2}^{#3}}}
\def\bbbuildrel#1_#2{\mathrel{
 \mathop{\kern 0pt#1}\limits_{#2}}}
\def\limsup{\mathop{\overline{\rm lim}}}
\def\liminf{\mathop{\underline{\rm lim}}}
\numberwithin{section}{chapter}
\numberwithin{equation}{section}
\numberwithin{figure}{section}

\theoremstyle{plain}
\newtheorem{thm}{\protect\theoremname}

%%%%%%%%%%%%%%%%%%%%%%%%%%%%%% User specified LaTeX commands.
%&  --enable-write18
\batchmode
\def\input@path{{style/}{sections/}{pdf/}}
%\usepackage[frame,center,letter,pdflatex,cam]{crop}
\graphicspath{{ps/}{logo/}{figures/}{sections/Figures/}}
\usepackage{yfonts}
%\usepackage{auto-pst-pdf}
%\renewcommand{\LettrineTextFont}{\scshape}
%%%%% nueva definicion  de entornos theorem %%%%%%%%%%%%%%%%
\definecolor{ptcbackground}{RGB}{212,237,252}
%\definecolor{ptctitle}{RGB}{0,177,235}
\definecolor{ptctitle}{cmyk}{1,0.25,0,0.08}
%%%%%%%%%%%%%%%%%%%%

\makeatother

\providecommand{\theoremname}{Teorema}

\begin{document}
\frontmatter \pagenumbering{alph}

\pagenumbering{roman}
\protect\thispagestyle{empty}%
\pagecolor{ptctitle}
\newgeometry{left=1.5cm,bottom=2cm,top=1.5cm,right=1.5cm}
\protect\enlargethispage{10cm}%
 \vskip -2.70cm%   
\hspace{-2.8cm}%
\begin{center}
\begin{tabular}[t]{@{}c@{}}%
\colorbox{ptctitle}{% 
\begin{minipage}[b][0.8\thesisHeight][c]{\thesisWidth}% 
\color{white}%


\chapter*{\textcolor{blue}{Abstract}}

\addstarredchapter{Abstract}

\thispagestyle{empty} \begin{small} 


\lettrine[lines=4,loversize=-0.1,lraise=0.1,lhang=.2]{T}{}{\vskip -2exhe sequences are very versatile data structures. In
a straightforward manner, a sequence of symbols can store any type
of information. Systematic analysis of sequences is a very rich area
of algorithmics, with lots of successful applications. The comparison
by sequence alignment is a very powerful analysis tool. Dynamic programming
is one of the most popular and efficient approaches to align two sequences.
However, despite their utility, alignments are not always the best
option for characterizing the function of two sequences. Sequences
often encode information in different levels of organization (meta-information).
In these cases, direct sequence comparison is not able to unveil those
higher-order structures that can actually explain the relationship
between the sequences.}

We have contributed with the work presented here to improve the way
in which two sequences can be compared, developing a new family of
algorithms that align high level information encoded in biological
sequences (meta-alignment). Initially, we have redesigned an existent
algorithm, based in dynamic programming, to align two sequences of
meta-information, introducing later several improvements for a better
performance. Next, we have developed a multiple meta-alignment algorithm,
by combining the general algorithm with the progressive schema. In
addition, we have studied the properties of the resulting meta-alignments,
modifying the algorithm to identify non-collinear or permuted configurations.

Molecular life is a great example of the sequence versatility. Comparative
genomics provide the identification of numerous biologically functional
elements. The nucleotide sequence of many genes, for example, is relatively
well conserved between different species. In contrast, the sequences
that regulate the gene expression are shorter and weaker. Thus, the
simultaneous activation of a set of genes only can be explained in
terms of conservation between configurations of higher-order regulatory
elements, that can not be detected at the sequence level. We, therefore,
have trained our meta-alignment programs in several datasets of regulatory
regions collected from the literature. Then, we have tested the accuracy
of our approximation to successfully characterize the promoter regions
of human genes and their orthologs in other species. \textbf{Palabras
Claves:} polinomios tipo Apostol generalizados; relaci�n de recurrencia;
ecuaci�n diferencial; polinomios de Apostol-Bernoulli; polinomios
de Genocchi; polinomios de Jacobi; polinomios de Hermite; polinomios
de Laguerre; polinomios de Charlier; polinomios de Bessel; polinomios
de Bernoulli generalizados; n�meros de Stirling de segunda clase.
\end{small}

\clearemptydoublepage


\chapter*{\textcolor{blue}{Resumen}}

\addstarredchapter{Resumen}\thispagestyle{empty}\begin{small} 


\lettrine[lines=4,loversize=-0.1,lraise=0.1,lhang=.2]{E}{}{n este trabajo estudiamos una nueva clase de polinomios tipo Apostol
generalizados $\mathcal{Q}_{n}^{[m-1,\alpha]}(x,b,c;\lambda;u,v)$
con $(\alpha,u,v\in\CC$ y $b,c\in\RR^{+})$ los cuales est�n definidos
en un entorno adecuado de $t=0$ por la siguiente funci�n generatriz:
\begin{equation}
{\displaystyle \left(\frac{(2^{u}t^{v})^{m}}{\lambda b^{t}+\sum\limits _{l=0}^{m-1}\frac{(t\log b)^{l}}{l!}}\right)^{\alpha}c^{xt}={\displaystyle \sum\limits _{n=0}^{\infty}\mathcal{Q}_{n}^{[m-1,\alpha]}(x,b,c;\lambda;u,v)\frac{t^{n}}{n!};\quad|t\log b|<|log(-\lambda)|,}}
\end{equation}
estos polinomios generan a las nuevas clases de todos los polinomios
conocidos de Apostol, as\'{i}:\\
 
\begin{align}
 & \mathcal{B}_{n}^{[m-1,\alpha]}(x;b,c;\lambda)=(-1)^{\alpha}\mathcal{Q}_{n}^{[m-1,\alpha]}(x;b,c;-\lambda;0,1),\\
 & \mathcal{E}_{n}^{[m-1,\alpha]}(x;b,c;\lambda)=\mathcal{Q}_{n}^{[m-1,\alpha]}(x;b,c;\lambda;1,0),\\
 & \mathcal{G}_{n}^{[m-1,\alpha]}(x;b,c;\lambda)=\mathcal{Q}_{n}^{[m-1,\alpha]}(x;b,c;\lambda;1,1).
\end{align}
}

Establecemos algunas propiedades b�sicas para estos polinomios, incluyendo
relaci�n de recurrencia y la ecuaci�n diferencial que estos polinomios
satisfacen. Finalmente se determinan f�rmulas de conexi�n entre los
polinomios y los polinomios de Genocchi, los polinomios de Jacobi,
los polinomios de Hermite, los polinomios de Laguerre, los polinomios
de Charlier, los polinomios de Bessel, los polinomios de Bernoulli
generalizados $B_{n}^{[m-1]}(x)$ y los n�meros de Stirling de segunda
clase, lo cual extiende algunos resultados conocidos. Finalmente introducimos
una nueva clase de polinomios tipo Apostol generalizados basados en
los polinomios de Hermite de dos variables ${_{H}}\mathcal{Q}_{n}^{[m-1,\alpha]}(x,b,c;\lambda;u,v)$
y mencionamos propiedades b�sicas. \\


\noindent \textbf{Palabras Claves:} polinomios tipo Apostol generalizados;
relaci�n de recurrencia; ecuaci�n diferencial; polinomios de Apostol-Bernoulli;
polinomios de Genocchi; polinomios de Jacobi; polinomios de Hermite;
polinomios de Laguerre; polinomios de Charlier; polinomios de Bessel;
polinomios de Bernoulli generalizados; n�meros de Stirling de segunda
clase. \end{small}

\clearemptydoublepage
\global\long\def\thyauthor{\mbox{Eddie Edinson\:Rodr�guez\:Bossio}}
 %
\fbox{%
\fbox{\begin{minipage}[t][1.02\textheight][c]{0.8\textwidth}%
\begin{center}
\vspace*{2cm}\setlength{\fboxsep}{0pt}\setlength{\fboxrule}{2.5pt}\includegraphics[scale=0.15]{logo_uab.png}
\par\end{center}

\vspace{-6.5cm}

\begin{center}
\scalebox{1.1}{ %
\begin{minipage}[t][0.6\textheight][c]{0.8\textwidth}%
\centering\textsf{\Huge{}\mytitle}\textsf{\large{}{}\textsf{\Huge{}Dise�os
�ptimos en la presencia de efectos de bloques aleatorios}}{\Huge \par}

\vspace*{-4.5cm}%
\end{minipage}} 
\par\end{center}

\begin{center}
\textsf{\scalebox{0.9}{}\textsf{\Huge{}\thyauthor}}
\par\end{center}{\Huge \par}

\vspace*{1.5cm}

\begin{center}
\textsf{\huge{}Tesis De Maestr�a \\[1.5ex] }\textsf{\large{}Barranquilla,
\thydate }
\par\end{center}{\large \par}%
\end{minipage}}} 


\end{minipage}%
    }% 
\end{tabular}%
\end{center}
\restoregeometry
\newpage
\protect\thispagestyle{empty}%
\pagecolor{White}

\pagenumbering{roman}\include{sections/title}\maketitle

\global\long\def\thyauthor{\mbox{Dise�os �ptimos En La Presencia De Efectos De Bloques Aleatorios}}
 \thispagestyle{empty}

\begin{figure}[H]
\centering{}\includegraphics[scale=0.15,]{logo_ua2.png} 
\end{figure}


\vspace{-2cm}


\begin{center}
{\avantgarboldHuge\mytitle\vspace{0.5cm}
} 
\par\end{center}

\vspace{0.5cm}


\begin{center}
{\avantgarboldLarge\thyauthor}\\[3cm] \textsf{\small{} Trabo
Especial de Grado Presentado a la Universidad del Atl�ntico por }
\par\end{center}{\small \par}

\begin{center}
\textbf{Eddie Edinson Rodriguez Bossio}
\par\end{center}

\begin{center}
{\small{}Trabajo de grado presentado para optar al grado de}
\par\end{center}{\small \par}

\begin{center}
\textbf{\small{}Magister en Ciencias Matem�tica}{\small{} }\\

\par\end{center}{\small \par}

\begin{center}
{\small{}Este trabajo de investigaci�n ha sido realizada bajo la direcci�n
de}\\
{\small{} }\textbf{\small{}Dr. rer. nat. Jes�s Alonso Cabrera \thyadvisor}{\small{}$^{\dagger}$
\\[2ex]}
\par\end{center}{\small \par}

\begin{center}
{\small{}$\dagger$ Departamento de Matem�tica,}\\
{\small{} Universidad del Norte (UN) }
\par\end{center}{\small \par}

\begin{center}
\vspace{0.25cm}
 \hrule \vspace{0.25cm}

\par\end{center}

\vspace{0.5cm}


\begin{center}
{\avantgar Barranquilla, \thydate} 
\par\end{center}


\textcolor{black}{\footnotesize{}\include{sections/jurado}}{\footnotesize \par}

\normalsize


\chapter*{\textcolor{blue}{Agradecimientos} }

\addstarredchapter{Agradecimientos}

\protect\thispagestyle{empty}


\lettrine[lines=4,loversize=-0.1,lraise=0.1,lhang=.2]{E}{}{n primer lugar doy gracias a Dios dador de la vida y oportunidades;
quien puso en mi camino a personas e instituciones que fueron piezas
claves para este logro en mi vida. Ellos son mi familia, mi tutor
de tesis Dr. rer. nat. Jes�s Alonso Cabrera por compartir sus conocimientos
y consejos, Dr. Jorge Rodriguez, Dr. Alejandro Urieles, profesores
de la maestr\'{i}a, compa�eros de estudio y las instituciones de la
Universidad del Atl�ntico donde curs� mi pregrado y maestr\'{i}a,
y la Universidad del Norte donde realic� mi especializaci�n en matem�ticas
y se llevaron a cabo las asesor\'{i}as de este trabajo de grado, ya
que mi asesor es docente de planta de esta alma mater.}

\vspace{0.2cm}
 \hspace{11cm} Eddie Rodriguez


\protect\thispagestyle{empty} 
\dominitoc[c]
 \nomtcrule

{    
 \setlength{\parskip}{0.4ex plus0.2ex minus0.2ex}
    \tableofcontents              
    \addstarredchapter{Contenido} 
   %\clearemptydoublepage
    %\listoftables                 
   %\addstarredchapter{Lista de Tablas} 
   %\clearemptydoublepage
    %\listoffigures 
     %\addstarredchapter{Lista de Figuras} 
   \clearemptydoublepage } 

\textcolor{black}{\small{}
\chapter*{\textcolor{blue}{Introducci�n}}

\addstarredchapter{Introducci�n}

%\iflettrine


\lettrine[lines=4,loversize=-0.1,lraise=0.1,lhang=.2]{A}{}{l interior de los experimentos estad\'{i}sticos la teor\'{i}a de
los dise�os �ptimos ha sido desarrollada. En general el tema de esta
teor\'{i}a es que para un apropiado modelo, si queremos poner �nfasis
sobre una cualidad particular de los par�metros a estimar, entonces
la configuraci�n experimental deber\'{i}a ser elegida de acuerdo a
ciertos criterios con sentido estad\'{i}stico. En la literatura relacionada
con los dise�os �ptimos, un prominente autor fue Kiefer (1959), el
cu�l present� los principales conceptos, tales como dise�os aproximados
y una variedad de criterios de �ptimalidad para esta rama de los dise�os
de experimentos; Kiefer, en particular dio el nombre $D-$optimalidad
al criterio introducido por Wald (1943), este criterio es el m�s comunmente
aplicado y est� definido en funci�n del determinante de la matriz
de covarianza.\\
}

M�s recientemente son reconocidos los libros de Atkinson y Donev (1992)
y Pukelsheim (1993), donde los autores hacen una presentaci�n estad\'{i}stica
formal de los dise�os �ptimos. El presente trabajo se ha organizado
en tres cap\'{i}tulos: el cap\'{i}tulo uno (Preliminares) contiene
conceptos generales que sirven de apoyo y base a la teor\'{i}a que
se desarrolla en los siguientes dos cap\'{i}tulos. El cap\'{i}tulo
dos trata sobre dise�os �ptimos en la presencia de efectos de bloques
aleatorios, y en el cap\'{i}tulo tres se daran a conocer las conclusiones
y una serie de problemas abiertos para futuras investigaciones relacionadas
con el tema central de este trabajo de investigaci�n. 
}{\small \par}

\protect\thispagestyle{empty} 
\clearemptydoublepage

\mainmatter % NOTE (3 of 3): Explicit numbering style definition to ensure% mkindex/hyperref to link pages ok (and avoiding duplicate labels...)\pagenumbering{arabic}

%BIBLIO in blue\global\long\def\bibname{\textcolor{blue}{Bibliography}}


%%% BLOCK 1\partblue{[}Preliminares{]}{\textbf{P}reliminares}

\include{sections/cap1}

\clearemptydoublepage

%%%% BLOCK 3\partgreen{[}{]}{\textbf{M}eta} 
\chapter{\textcolor{blue}{Dise�os �ptimos en la presencia de efectos de bloques
aleatorios}}

\label{AB2}

%recuperar numeracion arabica%\global\long\def\thechapter{\arabic{chapter}}



\lettrine[lines=4,loversize=-0.1,lraise=0.1,lhang=.2]{P}{}{ara un modelo lineal en la presencia de efectos de bloques aleatorios
se describe la situaci�n donde se tienen $b$ bloques, cada uno con
$m_{i}$ observaciones. Por lo tanto,la $j$-�sima observaci�n $Y_{ij}$
al bloque $i$ se puede escribir como\\
 
\begin{equation}
Y_{ij}=\gamma_{i}+\beta_{0}+\mathbf{f}(\mathrm{x}_{ij})^{\top}\boldsymbol{\beta}+\epsilon_{ij}
\end{equation}
\\
 donde $\mathrm{x}_{ij}$ son los puntos experimentales, $j=1,...,m_{i}$,
$\boldsymbol{f}=(f_{1},...,f_{p})^{\top}$ es un conjunto de funciones
(conocidas) de regresi�n y $\beta_{0}\in\mathbb{R}$, $\boldsymbol{\beta}=(\beta_{1},...,\beta_{p})^{\top}$
son los par�metros desconocidos. $f$ puede ser la funci�n identidad
para regresi�n lineal simple.\\
}

El t�rmino $\gamma_{i}$ es el efecto del $i$-�simo bloque aleatorio
con $E(\gamma_{i})=0$ y $Var(\gamma_{i})=\sigma_{\gamma}^{2}$. Los
errores de observaci�n aleatorio $\epsilon_{ij}$ se supone que son
homoscedasticos, $E(\epsilon_{ij})=0$, $Var(\epsilon_{ij})=\sigma^{2}$
y $Cov(\gamma_{i},\epsilon_{ij})=0$. El an�lisis adicional depender�
del cociente de varianza $d=\sigma_{\gamma}^{2}/\sigma^{2}$ . Nos
centraremos en los par�metros de la poblaci�n $\theta=(\beta_{0},\boldsymbol{\beta}^{\top})^{\top}$.
Asumiremos que el n�mero de observaciones por bloque es constante,
es decir, $m_{i}=m$.\\


Denotemos por $\boldsymbol{Y}_{i}=(Y_{i1},...,Y_{im})^{\top}$ el
vector de observaciones para el bloque $i$. La matriz de covarianza
correspondiente $Cov(\boldsymbol{Y}_{i})=\sigma^{2}\boldsymbol{V}$
es completamente sim�trica, $\boldsymbol{V}=\boldsymbol{I}_{m}+d\boldsymbol{1}_{m}\boldsymbol{1}_{m}^{\top}$,
donde $\boldsymbol{I}_{m}$ indica la matriz identidad $m\times m$
y $\boldsymbol{1}_{m}$ es un vector de longitud $m$ con todas las
entradas iguales a uno. El efecto fijo individual de la matriz de
dise�o $\boldsymbol{X}_{i}=(\boldsymbol{1}_{m}\mid\boldsymbol{F}_{i})$
se puede descomponer en la primera columna de unos correspondiente
a la intersecci�n $\beta_{0}$ y la matriz de dise�o para el vector
de par�metros $\boldsymbol{\beta}$.\\


La inversa de $\boldsymbol{V}$ la podemos hallar mediante �lgebra
matricial\
\begin{equation}
\begin{aligned}\boldsymbol{V}^{-1} & =(\boldsymbol{I}_{m}+d\boldsymbol{1}_{m}\boldsymbol{1}_{m}^{\top})^{-1}\\
 & =\boldsymbol{I}-d^{\frac{1}{2}}\boldsymbol{1}_{m}(\boldsymbol{I}+d\boldsymbol{1}_{m}^{\top}\boldsymbol{I}\boldsymbol{1}_{m})^{-1}d^{\frac{1}{2}}\boldsymbol{1}_{m}^{\top}\\
 & =\boldsymbol{I}-d\boldsymbol{1}_{m}(\boldsymbol{I}+dm\boldsymbol{I})^{-1}\boldsymbol{1}_{m}^{\top}\\
 & =\boldsymbol{I}-d\boldsymbol{1}_{m}(\boldsymbol{I}(1+dm))^{-1}\boldsymbol{1}_{m}^{\top}\\
 & =\boldsymbol{I}-\frac{d}{1+dm}\boldsymbol{1}_{m}\boldsymbol{1}_{m}^{\top}
\end{aligned}
\end{equation}
\\


Luego, la matriz de informaci�n por bloque, utilizando $(2\ldotp2)$
queda $\boldsymbol{X}_{i}^{\top}\boldsymbol{V}^{-1}\boldsymbol{X}_{i}=\boldsymbol{X}_{i}^{\top}\boldsymbol{X}_{i}-\frac{d}{1+md}\boldsymbol{X}_{i}^{\top}\boldsymbol{1}_{m}\boldsymbol{1}_{m}^{\top}\boldsymbol{X}_{i}$
que es proporcional a la inversa de la matriz de varianza-covarianza
$Cov(\boldsymbol{\hat{\theta}}_{i})$ si $\boldsymbol{X}_{i}$ es
de rango completo. As\'{i} $\boldsymbol{\theta}$ es estimado sobre
una base por bloque

\begin{equation}
\begin{aligned}\boldsymbol{\hat{\theta}}_{i} & =(\boldsymbol{X}_{i}^{\top}\boldsymbol{V}^{-1}\boldsymbol{X}_{i})^{-1}\boldsymbol{X}_{i}^{\top}\boldsymbol{V}^{-1}\boldsymbol{Y}_{i}\\
 & =(\boldsymbol{X}_{i}^{\top}\boldsymbol{X}_{i})^{-1}\boldsymbol{X}_{i}^{\top}\boldsymbol{Y}_{i}
\end{aligned}
\end{equation}
\\


Sobre la base de la poblaci�n el mejor estimador lineal insesgado
se puede calcular como $\boldsymbol{\hat{\theta}}=\left({\displaystyle \sum_{i=1}^{b}\boldsymbol{X}_{i}^{\top}\boldsymbol{V}^{-1}\boldsymbol{X}_{i}}\right)^{-1}{\displaystyle \sum_{i=1}^{b}\boldsymbol{X}_{i}^{\top}\boldsymbol{V}^{-1}\boldsymbol{X}_{i}\boldsymbol{\hat{\theta}}_{i}}$
si $d$ es conocido. Entonces $Cov(\boldsymbol{\hat{\theta}})=\sigma^{2}\boldsymbol{M}_{d}^{-1}$,
donde $\boldsymbol{M}_{d}={\displaystyle \sum_{i=1}^{b}\boldsymbol{X}_{i}^{\top}\boldsymbol{V}^{-1}\boldsymbol{X}_{i}}$
es la matriz de informaci�n sobre la base de la poblaci�n. El sub\'{i}ndice
$d$ indica la dependencia del cociente de varianzas $d$. Como $\boldsymbol{M}_{d}={\displaystyle \sum_{i=1}^{b}\boldsymbol{X}_{i}^{\top}\boldsymbol{X}_{i}-\frac{d}{1+md}{\displaystyle \sum_{i=1}^{b}\boldsymbol{X}_{i}^{\top}\boldsymbol{1}_{m}\boldsymbol{1}_{m}^{\top}\boldsymbol{X}_{i}}}$.
\\
\\


La matriz de informaci�n particionada de acuerdo a $\beta_{0}$ y
$\boldsymbol{\beta}$, es \\
 
\begin{equation}
\begin{aligned}\boldsymbol{M}_{d} & ={\displaystyle \sum_{i=1}^{b}\boldsymbol{X}_{i}^{\top}\boldsymbol{X}_{i}-\frac{d}{1+md}{\displaystyle \sum_{i=1}^{b}\boldsymbol{X}_{i}^{\top}\boldsymbol{1}_{m}\boldsymbol{1}_{m}^{\top}\boldsymbol{X}_{i}}}\\
\\
 & ={\displaystyle \sum_{i=1}^{b}(\boldsymbol{1}_{m}\mid\boldsymbol{F}_{i})^{\top}(\boldsymbol{1}_{m}\mid\boldsymbol{F}_{i})-\frac{d}{1+md}{\displaystyle \sum_{i=1}^{b}(\boldsymbol{1}_{m}\mid\boldsymbol{F}_{i})^{\top}\boldsymbol{1}_{m}\boldsymbol{1}_{m}^{\top}(\boldsymbol{1}_{m}\mid\boldsymbol{F}_{i})}}\\
\\
 & ={\displaystyle \sum_{i=1}^{b}\begin{pmatrix}\boldsymbol{1}_{m}^{\top}\\
\\
\boldsymbol{F}_{i}^{\top}
\end{pmatrix}(\boldsymbol{1}_{m}\mid\boldsymbol{F}_{i})-\frac{d}{1+md}{\displaystyle \sum_{i=1}^{b}\begin{pmatrix}\boldsymbol{1}_{m}^{\top}\\
\\
\boldsymbol{F}_{i}^{\top}
\end{pmatrix}\boldsymbol{1}_{m}\boldsymbol{1}_{m}^{\top}(\boldsymbol{1}_{m}\mid\boldsymbol{F}_{i})}}\\
\\
 & ={\displaystyle \sum_{i=1}^{b}\begin{pmatrix}m & \boldsymbol{1}_{m}^{\top}\boldsymbol{F}_{i}\\
\\
\boldsymbol{F}_{i}^{\top}\boldsymbol{1}_{m} & \boldsymbol{F}_{i}^{\top}\boldsymbol{F}_{i}
\end{pmatrix}-\frac{d}{1+md}{\displaystyle \sum_{i=1}^{b}\begin{pmatrix}m\\
\\
\boldsymbol{F}_{i}^{\top}\boldsymbol{1}_{m}
\end{pmatrix}(m\mid\boldsymbol{1}_{m}^{\top}\boldsymbol{F}_{i})}}\\
\\
 & ={\displaystyle \sum_{i=1}^{b}\begin{pmatrix}m & \boldsymbol{1}_{m}^{\top}\boldsymbol{F}_{i}\\
\\
\boldsymbol{F}_{i}^{\top}\boldsymbol{1}_{m} & \boldsymbol{F}_{i}^{\top}\boldsymbol{F}_{i}
\end{pmatrix}-\frac{d}{1+md}{\displaystyle \sum_{i=1}^{b}\begin{pmatrix}m^{2} & m\boldsymbol{1}_{m}^{\top}\boldsymbol{F}_{i}\\
\\
\boldsymbol{F}_{i}^{\top}\boldsymbol{1}_{m}m & \boldsymbol{F}_{i}^{\top}\boldsymbol{1}_{m}\boldsymbol{1}_{m}^{\top}\boldsymbol{F}_{i}
\end{pmatrix}}}\\
\\
 & =\frac{1}{1+md}{\displaystyle \sum_{i=1}^{b}\begin{pmatrix}m+m^{2}d & \boldsymbol{1}_{m}^{\top}\boldsymbol{F}_{i}+md\boldsymbol{1}_{m}^{\top}\boldsymbol{F}_{i}\\
\\
\boldsymbol{F}_{i}^{\top}\boldsymbol{1}_{m}+md\boldsymbol{F}_{i}^{\top}\boldsymbol{1}_{m} & \boldsymbol{F}_{i}^{\top}\boldsymbol{F}_{i}+md\boldsymbol{F}_{i}^{\top}\boldsymbol{F}_{i}
\end{pmatrix}}\\
\\
\end{aligned}
\end{equation}
\[
-\frac{1}{1+md}{\displaystyle \sum_{i=1}^{b}\begin{pmatrix}dm^{2} & dm\boldsymbol{1}_{m}^{\top}\boldsymbol{F}_{i}\\
\\
d\boldsymbol{F}_{i}^{\top}\boldsymbol{1}_{m}m & d\boldsymbol{F}_{i}^{\top}\boldsymbol{1}_{m}\boldsymbol{1}_{m}^{\top}\boldsymbol{F}_{i}
\end{pmatrix}}
\]


\begin{equation}
\boldsymbol{M}_{d}=\frac{1}{1+md}\left(\begin{tabular}{c|c}
 \ensuremath{bm}  &  \ensuremath{{\displaystyle \sum_{i=1}^{b}\boldsymbol{1}_{m}^{T}\boldsymbol{F}_{i}}} \\
\hline \ensuremath{{\displaystyle \sum_{i=1}^{b}\boldsymbol{F}_{i}^{\top}\boldsymbol{1}_{m}}}  &  \ensuremath{(1+md){\displaystyle \sum_{i=1}^{b}\boldsymbol{F}_{i}^{\top}\boldsymbol{F}_{i}-d{\displaystyle \sum_{i=1}^{b}\boldsymbol{F}_{i}^{\top}\boldsymbol{1}_{m}\boldsymbol{1}_{m}^{\top}\boldsymbol{F}_{i}}}} 
\end{tabular}\right)
\end{equation}
\\
 \\


Si el inter�s est� en los efectos fijos $\boldsymbol{\beta}$ solamente,
entonces las reglas para invertir las correspondientes matrices de
informaci�n parcial particionadas $\boldsymbol{M}_{\boldsymbol{\beta},d}^{-1}=Cov(\boldsymbol{\hat{\beta}})/\sigma^{2}$
es igual a\\
 
\begin{equation}
\boldsymbol{M}_{\boldsymbol{\beta},d}={\displaystyle \sum_{i=1}^{b}\boldsymbol{F}_{i}^{\top}\boldsymbol{F}_{i}-\frac{d}{1+md}{\displaystyle \sum_{i=1}^{b}\boldsymbol{F}_{i}^{\top}\boldsymbol{1}_{m}\boldsymbol{1}_{m}^{\top}\boldsymbol{F}_{i}-\frac{1}{bm}\ \frac{1}{1+md}\left({\displaystyle \sum_{i=1}^{b}\boldsymbol{F}_{i}^{\top}\boldsymbol{1}_{m}}\right)\left({\displaystyle \sum_{i=1}^{b}\boldsymbol{1}_{m}^{\top}\boldsymbol{F}_{i}}\right)}}
\end{equation}
\\
 \\
 Tambi�n consideramos los modelos l\'{i}mites para $d=0$ y $d\rightarrow\infty$,
respectivamente: Para $d=0$ obtenemos los modelos de efectos fijos
y sin interceptos de bloques \\


\begin{equation}
Y_{ij}=\beta_{0}+\boldsymbol{f}(\mathrm{x}_{ij})^{\top}\boldsymbol{\beta}+\epsilon_{ij}
\end{equation}
\\
 Obviamente, $\boldsymbol{M}_{d}$ tiende a $\boldsymbol{M}_{0}={\displaystyle \sum_{i=1}^{b}\boldsymbol{X}_{i}^{\top}\boldsymbol{X}_{i}}$
para $d\rightarrow0$. Del mismo modo, $\boldsymbol{M}_{\boldsymbol{\beta},d}$
tiende a\\
 
\begin{equation}
\boldsymbol{M}_{\boldsymbol{\beta},0}={\displaystyle \sum_{i=1}^{b}\boldsymbol{F}_{i}^{\top}\boldsymbol{F}_{i}-\frac{1}{bm}\left({\displaystyle \sum_{i=1}^{b}\boldsymbol{F}_{i}^{\top}\boldsymbol{1}_{m}}\right)\left({\displaystyle \sum_{i=1}^{b}\boldsymbol{1}_{m}^{\top}\boldsymbol{F}_{i}}\right)}
\end{equation}
\\
 \\
 Para $d\rightarrow\infty$ introducimos el modelo de efectos fijos
con bloques fijos\\
 
\begin{equation}
Y_{ij}=\mu_{i}+\boldsymbol{f}(\mathrm{x}_{ij})^{\top}\boldsymbol{\beta}+\epsilon_{ij};\ (\mu_{i}=\gamma_{i}+\beta_{0})
\end{equation}
\\
 Aqu\'{i}, el vector de par�metros $(\mu_{1},...\mu_{b},\beta_{1},...,\beta_{p})^{\top}$
tiene dimensi�n $b+p$ y la matriz de informaci�n correspondiente
tiene la forma\\


\begin{equation}
\boldsymbol{M}_{\infty}=\left(\begin{tabular}{ccc|c}
  &   &   &  \ensuremath{\boldsymbol{1}_{m}^{\top}\boldsymbol{F}_{1}} \\
 &  \ensuremath{m\boldsymbol{I}_{b}}  &   &  \ensuremath{\vdots} \\
 &   &   &  \ensuremath{\boldsymbol{1}_{m}^{\top}\boldsymbol{F}_{b}} \\
\hline \ensuremath{\boldsymbol{F}_{1}^{\top}}  &  \ensuremath{\cdots}  &  \ensuremath{\boldsymbol{F}_{b}^{\top}\boldsymbol{1}_{m}}  &  \ensuremath{{\displaystyle \sum_{i=1}^{b}\boldsymbol{F}_{i}^{\top}\boldsymbol{F}_{i}}} 
\end{tabular}\right)
\end{equation}
\\
 \\
 Para $\boldsymbol{\beta}$ la matriz de informaci�n parcial correspondiente
se puede calcular\\
 
\begin{equation}
\boldsymbol{M}_{\boldsymbol{\beta},\infty}={\displaystyle \sum_{i=1}^{b}\boldsymbol{F}_{i}^{\top}\boldsymbol{F}_{i}-\frac{1}{m}{\displaystyle \sum_{i=1}^{b}\boldsymbol{F}_{i}^{\top}\boldsymbol{1}_{m}\boldsymbol{1}_{m}^{\top}\boldsymbol{F}_{i}}}
\end{equation}
\\
 De ah� se obtiene el siguiente resultado, que establece la matriz
de informaci�n parcial $\boldsymbol{M}_{\boldsymbol{\beta},d}$.\\
 \\


$\mathbf{Lema1.}$\\
 
\begin{equation}
\boldsymbol{M}_{\boldsymbol{\beta},d}=\frac{1}{1+md}\boldsymbol{M}_{\boldsymbol{\beta},0}+\frac{md}{1+md}\boldsymbol{M}_{\boldsymbol{\beta},\infty}
\end{equation}
\\
 \textit{Tenga en cuenta que la matriz de informaci�n parcial $\boldsymbol{M}_{\boldsymbol{\beta},d}$
tiende a $\boldsymbol{M}_{\boldsymbol{\beta},\infty}$ cuando $d$
tiende a $\infty$.}\\
 \\



\section{Aspectos del dise�o}

La calidad de los estimadores $\boldsymbol{\hat{\theta}}$ y $\boldsymbol{\hat{\beta}}$
depende de la configuraci�n experimental $\mathrm{x}_{ij}$, $i=1,...,b$,
$j=1,...,m$, a trav�s de las matrices de informaci�n $\boldsymbol{M}_{d}$
y $\boldsymbol{M}_{\boldsymbol{\beta},d}$, respectivamente. El objetivo
en el dise�o experimental es elegir los puntos de una regi�n dise�o
$\mathcal{X}$ con el fin de minimizar la covarianza $Cov(\boldsymbol{\hat{\theta}})$
o $Cov(\boldsymbol{\hat{\beta}})$ o partes de ella, lo cual es equivalente
a maximizar las correspondientes matrices de informaci�n $\boldsymbol{M}_{d}$
o $\boldsymbol{M}_{\boldsymbol{\beta},d}$ respectivamente. Como esas
matrices no est�n completamente ordenadas, una optimizaci�n uniforme
no es posible, en general. Por lo tanto, algunos funcionales de valores
reales que ponen �nfasis en las propiedades particulares de los estimadores
se optimizar�n. El criterio de dise�o m�s usado es el $D-$criterio,
que tiene como objetivo maximizar el determinante de la matriz de
informaci�n $\boldsymbol{M}_{d}$. Esto es equivalente a minimizar
el volumen de un elipsoide de confianza para $\boldsymbol{\theta}$
bajo la suposici�n de normalidad.\\


Si el inter�s est� en los efectos $\boldsymbol{\beta}$ solamente,
$D_{\boldsymbol{\beta}}-$optimalidad se define en t�rminos de la
determinante de la inversa $\boldsymbol{M}_{\boldsymbol{\beta},d}^{-1}$
de la correspondiente matriz de informaci�n parcial. Como se ve a
continuaci�n,

\begin{equation}
\begin{aligned}\det(\boldsymbol{M}_{d}) & =\left|\frac{1}{1+md}\left(\begin{tabular}{c|c}
 \ensuremath{bm}  &  \ensuremath{{\displaystyle \sum_{i=1}^{b}\boldsymbol{1}_{m}^{T}\boldsymbol{F}_{i}}} \\
\hline \ensuremath{{\displaystyle \sum_{i=1}^{b}\boldsymbol{F}_{i}^{\top}\boldsymbol{1}_{m}}}  &  \ensuremath{(1+md){\displaystyle \sum_{i=1}^{b}\boldsymbol{F}_{i}^{\top}\boldsymbol{F}_{i}-d{\displaystyle \sum_{i=1}^{b}\boldsymbol{F}_{i}^{\top}\boldsymbol{1}_{m}\boldsymbol{1}_{m}^{\top}\boldsymbol{F}_{i}}}} 
\end{tabular}\right)\right|\end{aligned}
\end{equation}


$=\left(\frac{1}{1+md}\right)^{p+1}\left|bm\right||(1+md){\displaystyle \sum_{i=1}^{b}\boldsymbol{F}_{i}^{\top}\boldsymbol{F}_{i}-d{\displaystyle \sum_{i=1}^{b}\boldsymbol{F}_{i}^{\top}\boldsymbol{1}_{m}\boldsymbol{1}_{m}^{\top}\boldsymbol{F}_{i}}}$

\[
-{\displaystyle \sum_{i=1}^{b}\boldsymbol{F}_{i}^{\top}\boldsymbol{1}_{m}(bm)^{-1}{\displaystyle \sum_{i=1}^{b}\boldsymbol{1}_{m}^{T}\boldsymbol{F}_{i}|}}
\]


$=\left(\frac{1}{1+md}\right)^{p+1}\left|bm\right||{\displaystyle \sum_{i=1}^{b}\boldsymbol{F}_{i}^{\top}\boldsymbol{F}_{i}+md{\displaystyle \sum_{i=1}^{b}\boldsymbol{F}_{i}^{\top}\boldsymbol{F}_{i}-d{\displaystyle \sum_{i=1}^{b}\boldsymbol{F}_{i}^{\top}\boldsymbol{1}_{m}\boldsymbol{1}_{m}^{\top}\boldsymbol{F}_{i}}}}$

\[
-\frac{1}{bm}{\displaystyle \sum_{i=1}^{b}\boldsymbol{F}_{i}^{\top}\boldsymbol{1}_{m}{\displaystyle \sum_{i=1}^{b}\boldsymbol{1}_{m}^{T}\boldsymbol{F}_{i}|}}
\]


$=\left(\frac{1}{1+md}\right)^{p+1}\left|bm\right||{\displaystyle \sum_{i=1}^{b}\boldsymbol{F}_{i}^{\top}\boldsymbol{F}_{i}-\frac{1}{bm}{\displaystyle \sum_{i=1}^{b}\boldsymbol{F}_{i}^{\top}\boldsymbol{1}_{m}{\displaystyle \sum_{i=1}^{b}\boldsymbol{1}_{m}^{T}\boldsymbol{F}_{i}}}}$

\[
+md\left({\displaystyle \sum_{i=1}^{b}\boldsymbol{F}_{i}^{\top}\boldsymbol{F}_{i}-\frac{1}{m}{\displaystyle \sum_{i=1}^{b}\boldsymbol{F}_{i}^{\top}\boldsymbol{1}_{m}\boldsymbol{1}_{m}^{\top}\boldsymbol{F}_{i}}}\right)|
\]


$=\left(\frac{1}{1+md}\right)^{p+1}\left|bm\right|\left|\boldsymbol{M}_{\boldsymbol{\beta},0}+md\boldsymbol{M}_{\boldsymbol{\beta},\infty}\right|$\\


$=\left(\frac{1}{1+md}\right)^{p+1}\left|bm\right|\left|(1+md)\left(\frac{1}{1+md}\boldsymbol{M}_{\boldsymbol{\beta},0}+\frac{md}{1+md}\boldsymbol{M}_{\boldsymbol{\beta},\infty}\right)\right|$\\


$=\left(\frac{1}{1+md}\right)^{p+1}\left|bm\right|\left|(1+md)\boldsymbol{M}_{\boldsymbol{\beta},d}\right|$\\


$=\left(\frac{1}{1+md}\right)^{p+1}\left|bm\right|(1+md)^{p}\left|\boldsymbol{M}_{\boldsymbol{\beta},d}\right|$\\


$=\frac{bm}{1+md}\det\left(\boldsymbol{M}_{\boldsymbol{\beta},d}\right)$
\\


sujeta por la f�rmula para el determinante de matrices particionadas.
Por lo tanto, $D$ y $D_{\boldsymbol{\beta}}-$ optimalidad coinciden
tambi�n en modelos de interceptos aleatorios, un hecho bien conocido
en el ajuste de efectos fijos.\\
 \\


$\mathbf{Lema2.}$ \textit{Un dise�o $(\mathrm{x}_{ij})$ es $D-$�ptimo
si y s�lo si es $D_{\beta}$-�ptimo.}\\


Si tenemos en cuenta los dise�os que son uniformes en todos los bloques,
es decir, en que los par�metros experimentales son los mismos para
cada bloque, $\mathrm{x}_{ij}\equiv\mathrm{x}_{j}$, , entonces la
situaci�n se simplifica radicalmente. En este caso las matrices de
dise�os individuales coinciden, $\boldsymbol{\mathbf{F}}_{i}=\boldsymbol{\mathbf{F}}_{1}$
y $\boldsymbol{\mathbf{X}}_{i}=\boldsymbol{\mathbf{X}}_{1}$, respectivamente,
y $\boldsymbol{\mathbf{X}}_{1}$ tiene que ser de rango columna completa
para permitir estimabilidad de $\boldsymbol{\theta}$. Adem�s, $\boldsymbol{\hat{\theta}}=\frac{1}{b}{\displaystyle \sum_{i=1}^{b}\boldsymbol{\hat{\theta}}_{i}}$
se reduce a la media de los valores ajustados de forma individual
para los par�metros.\\


La matriz de covarianza estandarizada $\boldsymbol{M}_{d}^{-1}$ se
descompone de forma aditiva en la matriz correspondiente $\boldsymbol{M}_{0}^{-1}$
para el modelo de efectos fijos y sin intercepciones individuales
y la variabilidad de la intersecci�n aleatoria (v�ase, por ejemplo,
Entholzner y otros., (2005)). Para la matriz de informaci�n reducida
observamos\\


\begin{equation}
\boldsymbol{M}_{\boldsymbol{\beta},0}=b\left(\boldsymbol{\mathbf{F}}_{1}^{\top}\boldsymbol{\mathbf{F}}_{1}-\frac{1}{m}\boldsymbol{\mathbf{F}}_{1}^{\top}\boldsymbol{1}_{m}\boldsymbol{1}_{m}^{\top}\boldsymbol{\mathbf{F}}_{1}\right)=\boldsymbol{\mathbf{M}}_{\boldsymbol{\beta},\infty}
\end{equation}
\\


y, en consecuencia, por el Lema 1 $\boldsymbol{M}_{\boldsymbol{\beta},d}=\boldsymbol{M}_{\boldsymbol{\beta},0}$
es independiente de $d$. As\'{i}, el dise�o $D-$�ptimo para el modelo
de efectos fijos y sin intersecciones individuales es $D-$ y $D_{\boldsymbol{\beta}}-$�ptimo
para cada $d\geq0$ visto en el Lema 2.\\


{Ejemplo de aplicaci�n del modelo propuesto}

Consideremos el modelo de regresi�n cuadr�tico en dos variables sin
interacciones\\


$Y_{i}=\beta_{0}+\beta_{1}\mathrm{x}_{1i}+\beta_{2}\mathrm{x}_{2i}+\beta_{11}\mathrm{x}_{1i}^{2}+\beta_{22}\mathrm{x}_{2i}^{2}+\epsilon_{i};\ (\mathrm{x}_{1i},\mathrm{x}_{2i})\in[-1,1]\times[-1,1].$
\\


El dise�o $\boldsymbol{\xi}$ el cual asigna iguales pesos $\frac{1}{9}$
a las cuatro esquinas $(\pm1,\pm1)$, a los cuatro puntos centrales
de los lados $(0,\pm1)$; $(\pm1,0)$ y al punto central $(0,0)$
de la regi�n experimental. Este dise�o es $D$-�ptimo para este modelo.
Si el dise�o $\boldsymbol{\xi}$ es bloqueado como sigue\\


$\xi_{1}=\begin{pmatrix}(-1,0) & (0,1) & (1,-1)\\
1/3 & 1/3 & 1/3
\end{pmatrix},$ \ $\xi_{2}=\begin{pmatrix}(-1,-1) & (0,0) & (1,1)\\
1/3 & 1/3 & 1/3
\end{pmatrix},$ \\
 $\xi_{3}=\begin{pmatrix}(-1,1) & (1,0) & (0,-1)\\
1/3 & 1/3 & 1/3
\end{pmatrix}.$ \\
\\


Entonces por el lema 2, \ $\boldsymbol{\xi}$ es $D$-�ptimal para
el correspondiente modelo en la presencia de efectos de bloques con
respuesta\\


$Y_{ij}=\beta_{0}+\beta_{1}\mathrm{x}_{1ij}+\beta_{2}\mathrm{x}_{2ij}+\beta_{11}\mathrm{x}_{1ij}^{2}+\beta_{22}\mathrm{x}_{2ij}^{2}+\gamma_{i}+\epsilon_{ij},$
\\


en la $j$-�sima corrida sobre el bloque $i$, $(i=1,2,3;\ j=1,2,3)$
con pesos dados por $\boldsymbol{\xi}(i,(\mathrm{x}_{1ij},\mathrm{x}_{2ij}))=\frac{1}{3}\xi_{i}(\mathrm{x}_{1ij},\mathrm{x}_{2ij}).$
Adem�s este dise�o $D$-�ptimal por bloques no depende del cociente
de varianza $d$. 

\clearemptydoublepage 
\chapter{\textcolor{blue}{Conclusiones y trabajos futuros}}

\label{AB3}

%recuperar numeracion arabica%\global\long\def\thechapter{\arabic{chapter}}



\lettrine[lines=4,loversize=-0.1,lraise=0.1,lhang=.2]{E}{}{n el presente trabajo se desarrolla la matriz de informaci�n de
los criterios fijos en un modelo de regresi�n lineal en la presencia
de efectos de bloques aleatorios como una combinaci�n convexa de las
matrices de informaci�n de los modelos l�mites cuando la varianza
del efecto de bloque es cero o tiende a infinito.\\
}

%recuperar numeracion arabica%\global\long\def\thechapter{\arabic{chapter}}
 Por otro lado, se muestra que los dise�os �ptimos para modelos de
efectos fijos tambi�n son �ptimos para modelos en la presencia de
efectos de bloques aleatorios siempre que los bloques sean uniformes.\\


$D$ y $D_{\boldsymbol{\beta}}$-optimalidad coinciden tambi�n en
modelos en la presencia de efectos de bloques aleatorios, un hecho
ya conocido en los escenarios con efectos fijos.\\


Para trabajos futuros se pueden considerar modelos en la presencia
de efectos de bloques aleatorios donde los par�metros de regresi�n
interact�an con diferentes grupos o tratamientos. 
 \clearemptydoublepage
%
\chapter{CONCLUSIONES Y RECOMENDACIONES}
\markboth{CAP\'ITULO 4\quad CONCLUSIONES Y RECOMENDACIONES}{CAP\'ITULO 4\quad CONCLUSIONES Y RECOMENDACIONES}
\label{AB4}

%recuperar numeracion arabica%
\def\thechapter{\arabic{chapter}}

En este \'ultimo cap\'itulo haremos un resumen de las t\'ecnicas y resultados fundamentales de nuestro trabajo y expondremos problemas abiertos y futuras l\'ineas de investigaci\'on relativos a *****.

\section{Conclusiones}
Podemos decir que el resultado m\'as importante de este trabajo es el relacionado con *****, (ver teorema ****).



\section{Problemas Abiertos y Futuras L\'{\i}neas de Investigaci\'on}

De lo expuesto en la secci\'on precedente se desprende que algunos  problemas interesantes que permitir\'{\i}an  continuar con investigaciones relacionadas, ser\'{\i}an:


\textbf{Problema 1.}
Series de Fourier y representaci\'on integral de los polinomios $\mathcal{Q}_n^{[m-1,\alpha]}(x,b,c;\lambda;u,v)$.

\textbf{Problema 2.}
Estudio de las propiedades de los polinomios $\mathcal{Q}_n^{(\alpha)}(x;\lambda;a,b,c;u,v)$ definidos en (\ref{eqnewpol2}) y relaci\'on con otros polinomios y n\'umeros.

\textbf{Problema 3.}
Series de Fourier y representaci\'on integral de los polinomios $\mathcal{Q}_n^{(\alpha)}(x;\lambda;a,b,c;u,v)$ definidos en (\ref{eqnewpol2}).

\textbf{Problema 4.}
Estudio de la nueva clase de polinomios tipo Apostol generalizados basados en Hermite ${_{H}}\mathcal{Q}_n^{[m-1,\alpha]}(x,b,c;\lambda;u,v)$ que est\'an definidos en (\ref{new787}).

\textbf{Problema 5.}
Estudio de otra nueva clase de polinomios tipo Apostol generalizados basados en Hermite ${_{H}}\mathcal{Q}_n^{(\alpha)}(x;\lambda;a,b,c;u,v)$ que est\'an definidos por la siguiente funci\'on generatriz:
\begin{equation}\label{eqnewpol23}
\displaystyle\left(\frac{2^uz^v}{\lambda b^z+a^z}\right)^{\alpha}c^{xz+(\log c)yz^2} =\sum\limits_{n=0}^{\infty}
{_{H}}\mathcal{Q}_n^{(\alpha)}(x;\lambda;a,b,c;u,v)\frac{z^n}{n!} 
\end{equation}



 \clearemptydoublepage

\include{sections/bibliografia} \clearemptydoublepage

\newgeometry{left=1.5cm,bottom=1cm,top=2.5cm,right=1.5cm}
\protect\thispagestyle{empty}%
\protect\enlargethispage{10cm}%
\pagecolor{ptctitle}
\ \vskip -2.70cm%   
\hspace{-2.0cm}%
\begin{tabular}[t]{@{}c@{}}%
\centering
\colorbox{ptctitle}{% 
\begin{minipage}[b][\thesisHeight][c]{\thesisWidth}% 
\color{white}%

\begin{center}
\begin{minipage}[t][0.965\textheight][c]{0.8\textwidth}%
\color{coverwhite}

\begin{center}
\avantgarboldhuge\mytitle \vskip 2.5ex {\avantgarboldLarge\thyauthor}
 \colorbox{covergreen}{%
\begin{minipage}[c]{0.925\textwidth}%
\setlength{\parindent}{0.5cm}\color{coverwhite}\avantgarsmall\begin{small}
Al interior de los experimentos estad�sticos la teor�a de los dise�os
�ptimos ha sido desarrollada. En general el tema de esta teor�a es
que para un apropiado modelo, si queremos poner �nfasis sobre una
cualidad particular de los par�metros a estimar, entonces la configuraci�n
experimental deber\'{i}a ser elegida de acuerdo a ciertos criterios
con sentido estad�stico. En la literatura relacionada con los dise�os
�ptimos, un prominente autor fue Kiefer (1959), el cu�l present� los
principales conceptos, tales como dise�os aproximados y una variedad
de criterios de �ptimalidad para esta rama de los dise�os de experimentos;
Kiefer, en particular dio el nombre $D-$optimalidad al criterio introducido
por Wald (1943), este criterio es el m�s comunmente aplicado y est�
definido en funci�n del determinante de la matriz de covarianza.\\


M�s recientemente son reconocidos los libros de Atkinson y Donev (1992)
y Pukelsheim (1993), donde los autores hacen una presentaci�n estad�stica
formal de los dise�os �ptimos. El presente trabajo se ha organizado
en tres cap�tulos: el cap�tulo uno (Preliminares) contiene conceptos
generales que sirven de apoyo y base a la teor�a que se desarrolla
en los siguientes dos cap�tulos. El cap�tulo dos trata sobre dise�os
�ptimos en la presencia de efectos de bloques aleatorios, y en el
cap�tulo tres se daran a conocer las conclusiones y una serie de problemas
abiertos para futuras investigaciones relacionadas con el tema central
de este trabajo de investigaci�n. \end{small} %
\end{minipage}}
\par\end{center}

\vskip 2.5ex

\begin{center}
{\avantgarboldLarge Programa de Matem�ticas}\\

\par\end{center}

\begin{center}
{\avantgarLarge Universidad del Atl�ntico}\\[3ex]
\par\end{center}

\begin{center}

\par\end{center}%
\end{minipage}
\par\end{center}

%}%centering


\end{minipage}%
    }% 
\end{tabular}%

\end{document}
